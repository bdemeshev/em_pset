% !TEX root = ../em1_pset_v2.tex
\Opensolutionfile{solution_file}[solutions/sols_120]
% в квадратных скобках фактическое имя файла

\chapter{Временные ряды}


\begin{problem}
Что такое автокорреляция?
\begin{sol}
Наличие ненулевой корреляции между $y_t$ и $y_{t-s}$.
\end{sol}
\end{problem}



\begin{problem}
На графике представлены данные по уровню озера Гур\'{о}н в футах в 1875-1972 годах:


\begin{minted}[mathescape, numbersep=5pt, frame=lines, framesep=2mm]{r}
level <- LakeHuron
df <- tibble(level, obs = 1875:1972)
n <- nrow(df) # used later for answers
v.acf <- acf(level, plot = FALSE)$acf
v.pacf <- pacf(level, plot = FALSE)$acf
acfs.df <- tibble(lag = c(1:15, 1:15),
    acf = c(v.acf[2:16], v.pacf[1:15]),
    acf.type = rep(c("ACF", "PACF"), each = 15))
model <- arima(level, order = c(1, 0, 1))
resids <- model$residuals
resid.acf <- acf(resids, plot = FALSE)$acf
ggplot(df, aes(x = obs, y = level)) + geom_line() +
    labs(x = "Год", y = "Уровень озера (футы)")
\end{minted}

\begin{minipage}{0.6\textwidth}
\begin{center}
\begin{tikzpicture}[scale = 0.025]
% Created by tikzDevice version 0.12 on 2019-05-28 10:04:12
% !TEX encoding = UTF-8 Unicode
\definecolor{fillColor}{RGB}{255,255,255}
\path[use as bounding box,fill=fillColor,fill opacity=0.00] (0,0) rectangle (505.89,505.89);
\begin{scope}
\path[clip] (  0.00,  0.00) rectangle (505.89,505.89);
\definecolor{drawColor}{RGB}{255,255,255}
\definecolor{fillColor}{RGB}{255,255,255}

\path[draw=drawColor,line width= 0.6pt,line join=round,line cap=round,fill=fillColor] (  0.00,  0.00) rectangle (505.89,505.89);
\end{scope}
\begin{scope}
\path[clip] ( 36.04, 31.51) rectangle (500.39,483.46);
\definecolor{fillColor}{gray}{0.92}

\path[fill=fillColor] ( 36.04, 31.51) rectangle (500.39,483.46);
\definecolor{drawColor}{RGB}{255,255,255}

\path[draw=drawColor,line width= 0.3pt,line join=round] ( 36.04,124.47) --
	(500.39,124.47);

\path[draw=drawColor,line width= 0.3pt,line join=round] ( 36.04,263.75) --
	(500.39,263.75);

\path[draw=drawColor,line width= 0.3pt,line join=round] ( 36.04,403.03) --
	(500.39,403.03);

\path[draw=drawColor,line width= 0.3pt,line join=round] (111.54, 31.51) --
	(111.54,483.46);

\path[draw=drawColor,line width= 0.3pt,line join=round] (220.34, 31.51) --
	(220.34,483.46);

\path[draw=drawColor,line width= 0.3pt,line join=round] (329.14, 31.51) --
	(329.14,483.46);

\path[draw=drawColor,line width= 0.3pt,line join=round] (437.94, 31.51) --
	(437.94,483.46);

\path[draw=drawColor,line width= 0.6pt,line join=round] ( 36.04, 54.84) --
	(500.39, 54.84);

\path[draw=drawColor,line width= 0.6pt,line join=round] ( 36.04,194.11) --
	(500.39,194.11);

\path[draw=drawColor,line width= 0.6pt,line join=round] ( 36.04,333.39) --
	(500.39,333.39);

\path[draw=drawColor,line width= 0.6pt,line join=round] ( 36.04,472.67) --
	(500.39,472.67);

\path[draw=drawColor,line width= 0.6pt,line join=round] ( 57.14, 31.51) --
	( 57.14,483.46);

\path[draw=drawColor,line width= 0.6pt,line join=round] (165.94, 31.51) --
	(165.94,483.46);

\path[draw=drawColor,line width= 0.6pt,line join=round] (274.74, 31.51) --
	(274.74,483.46);

\path[draw=drawColor,line width= 0.6pt,line join=round] (383.54, 31.51) --
	(383.54,483.46);

\path[draw=drawColor,line width= 0.6pt,line join=round] (492.34, 31.51) --
	(492.34,483.46);
\definecolor{drawColor}{RGB}{0,0,0}

\path[draw=drawColor,line width= 0.6pt,line join=round] ( 57.14,359.85) --
	( 61.49,462.92) --
	( 65.85,400.94) --
	( 70.20,389.10) --
	( 74.55,318.77) --
	( 78.90,360.55) --
	( 83.25,362.64) --
	( 87.61,390.50) --
	( 91.96,430.89) --
	( 96.31,425.31) --
	(100.66,433.67) --
	(105.01,450.38) --
	(109.37,414.87) --
	(113.72,370.30) --
	(118.07,334.09) --
	(122.42,327.12) --
	(126.77,273.50) --
	(131.13,274.89) --
	(135.48,302.05) --
	(139.83,310.41) --
	(144.18,224.75) --
	(148.53,210.83) --
	(152.89,270.72) --
	(157.24,270.02) --
	(161.59,288.13) --
	(165.94,251.22) --
	(170.29,286.04) --
	(174.65,264.45) --
	(179.00,263.75) --
	(183.35,319.46) --
	(187.70,321.55) --
	(192.05,313.89) --
	(196.41,325.73) --
	(200.76,334.09) --
	(205.11,289.52) --
	(209.46,242.16) --
	(213.81,207.34) --
	(218.17,240.77) --
	(222.52,302.05) --
	(226.87,258.18) --
	(231.22,200.38) --
	(235.57,289.52) --
	(239.93,342.44) --
	(244.28,343.14) --
	(248.63,299.27) --
	(252.98,280.47) --
	(257.33,240.08) --
	(261.68,254.00) --
	(266.04,197.60) --
	(270.39,179.49) --
	(274.74,107.06) --
	(279.09,107.06) --
	(283.44,181.58) --
	(287.80,238.68) --
	(292.15,373.78) --
	(296.50,297.18) --
	(300.85,150.94) --
	(305.20,117.51) --
	(309.56,120.30) --
	(313.91, 71.55) --
	(318.26,113.33) --
	(322.61,114.03) --
	(326.96,117.51) --
	(331.32,179.49) --
	(335.67,206.65) --
	(340.02,159.99) --
	(344.37,140.49) --
	(348.72,223.36) --
	(353.08,306.23) --
	(357.43,267.23) --
	(361.78,281.86) --
	(366.13,279.07) --
	(370.48,290.22) --
	(374.84,270.72) --
	(379.19,190.63) --
	(383.54,202.47) --
	(387.89,315.98) --
	(392.24,392.58) --
	(396.60,361.94) --
	(400.95,330.61) --
	(405.30,306.23) --
	(409.65,247.04) --
	(414.00,206.65) --
	(418.36,139.10) --
	(422.71,133.53) --
	(427.06,270.72) --
	(431.41,211.52) --
	(435.76,187.85) --
	(440.12,116.81) --
	(444.47, 52.05) --
	(448.82,110.55) --
	(453.17,171.83) --
	(457.52,220.58) --
	(461.88,230.33) --
	(466.23,315.29) --
	(470.58,285.34) --
	(474.93,325.73) --
	(479.28,330.61);
\end{scope}
\begin{scope}
\path[clip] (  0.00,  0.00) rectangle (505.89,505.89);
\definecolor{drawColor}{gray}{0.30}

\node[text=drawColor,anchor=base east,inner sep=0pt, outer sep=0pt, scale=  0.73] at ( 31.09, 51.80) {576};

\node[text=drawColor,anchor=base east,inner sep=0pt, outer sep=0pt, scale=  0.73] at ( 31.09,191.08) {578};

\node[text=drawColor,anchor=base east,inner sep=0pt, outer sep=0pt, scale=  0.73] at ( 31.09,330.36) {580};

\node[text=drawColor,anchor=base east,inner sep=0pt, outer sep=0pt, scale=  0.73] at ( 31.09,469.64) {582};
\end{scope}
\begin{scope}
\path[clip] (  0.00,  0.00) rectangle (505.89,505.89);
\definecolor{drawColor}{gray}{0.20}

\path[draw=drawColor,line width= 0.6pt,line join=round] ( 33.29, 54.84) --
	( 36.04, 54.84);

\path[draw=drawColor,line width= 0.6pt,line join=round] ( 33.29,194.11) --
	( 36.04,194.11);

\path[draw=drawColor,line width= 0.6pt,line join=round] ( 33.29,333.39) --
	( 36.04,333.39);

\path[draw=drawColor,line width= 0.6pt,line join=round] ( 33.29,472.67) --
	( 36.04,472.67);
\end{scope}
\begin{scope}
\path[clip] (  0.00,  0.00) rectangle (505.89,505.89);
\definecolor{drawColor}{gray}{0.20}

\path[draw=drawColor,line width= 0.6pt,line join=round] ( 57.14, 28.76) --
	( 57.14, 31.51);

\path[draw=drawColor,line width= 0.6pt,line join=round] (165.94, 28.76) --
	(165.94, 31.51);

\path[draw=drawColor,line width= 0.6pt,line join=round] (274.74, 28.76) --
	(274.74, 31.51);

\path[draw=drawColor,line width= 0.6pt,line join=round] (383.54, 28.76) --
	(383.54, 31.51);

\path[draw=drawColor,line width= 0.6pt,line join=round] (492.34, 28.76) --
	(492.34, 31.51);
\end{scope}
\begin{scope}
\path[clip] (  0.00,  0.00) rectangle (505.89,505.89);
\definecolor{drawColor}{gray}{0.30}

\node[text=drawColor,anchor=base,inner sep=0pt, outer sep=0pt, scale=  0.73] at ( 57.14, 20.49) {1875};

\node[text=drawColor,anchor=base,inner sep=0pt, outer sep=0pt, scale=  0.73] at (165.94, 20.49) {1900};

\node[text=drawColor,anchor=base,inner sep=0pt, outer sep=0pt, scale=  0.73] at (274.74, 20.49) {1925};

\node[text=drawColor,anchor=base,inner sep=0pt, outer sep=0pt, scale=  0.73] at (383.54, 20.49) {1950};

\node[text=drawColor,anchor=base,inner sep=0pt, outer sep=0pt, scale=  0.73] at (492.34, 20.49) {1975};
\end{scope}
\begin{scope}
\path[clip] (  0.00,  0.00) rectangle (505.89,505.89);
\definecolor{drawColor}{RGB}{0,0,0}

\node[text=drawColor,anchor=base,inner sep=0pt, outer sep=0pt, scale=  0.92] at (268.21,  7.83) {Год};
\end{scope}
\begin{scope}
\path[clip] (  0.00,  0.00) rectangle (505.89,505.89);
\definecolor{drawColor}{RGB}{0,0,0}

\node[text=drawColor,rotate= 90.00,anchor=base,inner sep=0pt, outer sep=0pt, scale=  0.92] at ( 13.08,257.48) {Уровень озера (футы)};
\end{scope}
\begin{scope}
\path[clip] (  0.00,  0.00) rectangle (505.89,505.89);
\definecolor{drawColor}{RGB}{0,0,0}

\node[text=drawColor,anchor=base west,inner sep=0pt, outer sep=0pt, scale=  1.10] at ( 36.04,491.30) {Изменение высоты озера Гурон};
\end{scope}

\end{tikzpicture}
\end{center}
\end{minipage}


График автокорреляционной и частной автокорреляционной функций:

\todo[inline]{Здесь виснет чанк кода!}
\begin{minted}[mathescape, numbersep=5pt, frame=lines, framesep=2mm]{r}
ggplot(acfs.df, aes(x = lag, y = acf, fill = acf.type)) +
    geom_bar(position = "dodge", stat = "identity") +
  xlab("Лаг") + ylab("Корреляция") +
  guides(fill = guide_legend(title = NULL))+
  geom_hline(yintercept = 1.96/sqrt(nrow(df)))+
  geom_hline(yintercept = -1.96/sqrt(nrow(df)))
\end{minted}

\begin{minipage}{0.6\textwidth}
\begin{center}
\begin{tikzpicture}[scale = 0.025]
% Created by tikzDevice version 0.12 on 2019-05-26 19:37:53
% !TEX encoding = UTF-8 Unicode
\begin{tikzpicture}[x=1pt,y=1pt]
\definecolor{fillColor}{RGB}{255,255,255}
\path[use as bounding box,fill=fillColor,fill opacity=0.00] (0,0) rectangle (505.89,505.89);
\begin{scope}
\path[clip] (  0.00,  0.00) rectangle (505.89,505.89);
\definecolor{drawColor}{RGB}{255,255,255}
\definecolor{fillColor}{RGB}{255,255,255}

\path[draw=drawColor,line width= 0.6pt,line join=round,line cap=round,fill=fillColor] (  0.00,  0.00) rectangle (505.89,505.89);
\end{scope}
\begin{scope}
\path[clip] ( 36.99, 31.51) rectangle (432.34,483.46);
\definecolor{fillColor}{gray}{0.92}

\path[fill=fillColor] ( 36.99, 31.51) rectangle (432.34,483.46);
\definecolor{drawColor}{RGB}{255,255,255}

\path[draw=drawColor,line width= 0.3pt,line join=round] ( 36.99, 95.71) --
	(432.34, 95.71);

\path[draw=drawColor,line width= 0.3pt,line join=round] ( 36.99,207.90) --
	(432.34,207.90);

\path[draw=drawColor,line width= 0.3pt,line join=round] ( 36.99,320.10) --
	(432.34,320.10);

\path[draw=drawColor,line width= 0.3pt,line join=round] ( 36.99,432.29) --
	(432.34,432.29);

\path[draw=drawColor,line width= 0.3pt,line join=round] (102.00, 31.51) --
	(102.00,483.46);

\path[draw=drawColor,line width= 0.3pt,line join=round] (222.60, 31.51) --
	(222.60,483.46);

\path[draw=drawColor,line width= 0.3pt,line join=round] (343.21, 31.51) --
	(343.21,483.46);

\path[draw=drawColor,line width= 0.6pt,line join=round] ( 36.99, 39.62) --
	(432.34, 39.62);

\path[draw=drawColor,line width= 0.6pt,line join=round] ( 36.99,151.81) --
	(432.34,151.81);

\path[draw=drawColor,line width= 0.6pt,line join=round] ( 36.99,264.00) --
	(432.34,264.00);

\path[draw=drawColor,line width= 0.6pt,line join=round] ( 36.99,376.19) --
	(432.34,376.19);

\path[draw=drawColor,line width= 0.6pt,line join=round] ( 41.70, 31.51) --
	( 41.70,483.46);

\path[draw=drawColor,line width= 0.6pt,line join=round] (162.30, 31.51) --
	(162.30,483.46);

\path[draw=drawColor,line width= 0.6pt,line join=round] (282.91, 31.51) --
	(282.91,483.46);

\path[draw=drawColor,line width= 0.6pt,line join=round] (403.51, 31.51) --
	(403.51,483.46);
\definecolor{fillColor}{RGB}{0,191,196}

\path[fill=fillColor] ( 65.82,151.81) rectangle ( 76.67,462.92);
\definecolor{fillColor}{RGB}{248,118,109}

\path[fill=fillColor] ( 54.96,151.81) rectangle ( 65.82,462.92);
\definecolor{fillColor}{RGB}{0,191,196}

\path[fill=fillColor] ( 89.94, 52.05) rectangle (100.79,151.81);
\definecolor{fillColor}{RGB}{248,118,109}

\path[fill=fillColor] ( 79.08,151.81) rectangle ( 89.94,379.91);
\definecolor{fillColor}{RGB}{0,191,196}

\path[fill=fillColor] (114.06,151.81) rectangle (124.91,200.71);
\definecolor{fillColor}{RGB}{248,118,109}

\path[fill=fillColor] (103.21,151.81) rectangle (114.06,323.18);
\definecolor{fillColor}{RGB}{0,191,196}

\path[fill=fillColor] (138.18,151.81) rectangle (149.03,164.54);
\definecolor{fillColor}{RGB}{248,118,109}

\path[fill=fillColor] (127.33,151.81) rectangle (138.18,290.37);
\definecolor{fillColor}{RGB}{0,191,196}

\path[fill=fillColor] (162.30,151.81) rectangle (173.16,175.03);
\definecolor{fillColor}{RGB}{248,118,109}

\path[fill=fillColor] (151.45,151.81) rectangle (162.30,273.56);
\definecolor{fillColor}{RGB}{0,191,196}

\path[fill=fillColor] (186.42,143.90) rectangle (197.28,151.81);
\definecolor{fillColor}{RGB}{248,118,109}

\path[fill=fillColor] (175.57,151.81) rectangle (186.42,258.34);
\definecolor{fillColor}{RGB}{0,191,196}

\path[fill=fillColor] (210.54,151.81) rectangle (221.40,186.20);
\definecolor{fillColor}{RGB}{248,118,109}

\path[fill=fillColor] (199.69,151.81) rectangle (210.54,250.83);
\definecolor{fillColor}{RGB}{0,191,196}

\path[fill=fillColor] (234.66,151.81) rectangle (245.52,168.82);
\definecolor{fillColor}{RGB}{248,118,109}

\path[fill=fillColor] (223.81,151.81) rectangle (234.66,250.55);
\definecolor{fillColor}{RGB}{0,191,196}

\path[fill=fillColor] (258.79,151.81) rectangle (269.64,152.81);
\definecolor{fillColor}{RGB}{248,118,109}

\path[fill=fillColor] (247.93,151.81) rectangle (258.79,248.18);
\definecolor{fillColor}{RGB}{0,191,196}

\path[fill=fillColor] (282.91, 77.00) rectangle (293.76,151.81);
\definecolor{fillColor}{RGB}{248,118,109}

\path[fill=fillColor] (272.05,151.81) rectangle (282.91,220.15);
\definecolor{fillColor}{RGB}{0,191,196}

\path[fill=fillColor] (307.03,151.81) rectangle (317.88,159.05);
\definecolor{fillColor}{RGB}{248,118,109}

\path[fill=fillColor] (296.17,151.81) rectangle (307.03,187.26);
\definecolor{fillColor}{RGB}{0,191,196}

\path[fill=fillColor] (331.15,151.81) rectangle (342.00,155.34);
\definecolor{fillColor}{RGB}{248,118,109}

\path[fill=fillColor] (320.29,151.81) rectangle (331.15,168.42);
\definecolor{fillColor}{RGB}{0,191,196}

\path[fill=fillColor] (355.27,151.81) rectangle (366.12,156.18);
\definecolor{fillColor}{RGB}{248,118,109}

\path[fill=fillColor] (344.41,151.81) rectangle (355.27,162.73);
\definecolor{fillColor}{RGB}{0,191,196}

\path[fill=fillColor] (379.39,151.81) rectangle (390.24,164.76);
\definecolor{fillColor}{RGB}{248,118,109}

\path[fill=fillColor] (368.54,151.81) rectangle (379.39,167.20);
\definecolor{fillColor}{RGB}{0,191,196}

\path[fill=fillColor] (403.51,146.24) rectangle (414.37,151.81);
\definecolor{fillColor}{RGB}{248,118,109}

\path[fill=fillColor] (392.66,151.81) rectangle (403.51,168.74);
\definecolor{drawColor}{RGB}{0,0,0}

\path[draw=drawColor,line width= 0.6pt,line join=round] ( 36.99,225.85) -- (432.34,225.85);

\path[draw=drawColor,line width= 0.6pt,line join=round] ( 36.99, 77.76) -- (432.34, 77.76);
\end{scope}
\begin{scope}
\path[clip] (  0.00,  0.00) rectangle (505.89,505.89);
\definecolor{drawColor}{gray}{0.30}

\node[text=drawColor,anchor=base east,inner sep=0pt, outer sep=0pt, scale=  0.73] at ( 32.04, 36.58) {-0.3};

\node[text=drawColor,anchor=base east,inner sep=0pt, outer sep=0pt, scale=  0.73] at ( 32.04,148.78) {0.0};

\node[text=drawColor,anchor=base east,inner sep=0pt, outer sep=0pt, scale=  0.73] at ( 32.04,260.97) {0.3};

\node[text=drawColor,anchor=base east,inner sep=0pt, outer sep=0pt, scale=  0.73] at ( 32.04,373.16) {0.6};
\end{scope}
\begin{scope}
\path[clip] (  0.00,  0.00) rectangle (505.89,505.89);
\definecolor{drawColor}{gray}{0.20}

\path[draw=drawColor,line width= 0.6pt,line join=round] ( 34.24, 39.62) --
	( 36.99, 39.62);

\path[draw=drawColor,line width= 0.6pt,line join=round] ( 34.24,151.81) --
	( 36.99,151.81);

\path[draw=drawColor,line width= 0.6pt,line join=round] ( 34.24,264.00) --
	( 36.99,264.00);

\path[draw=drawColor,line width= 0.6pt,line join=round] ( 34.24,376.19) --
	( 36.99,376.19);
\end{scope}
\begin{scope}
\path[clip] (  0.00,  0.00) rectangle (505.89,505.89);
\definecolor{drawColor}{gray}{0.20}

\path[draw=drawColor,line width= 0.6pt,line join=round] ( 41.70, 28.76) --
	( 41.70, 31.51);

\path[draw=drawColor,line width= 0.6pt,line join=round] (162.30, 28.76) --
	(162.30, 31.51);

\path[draw=drawColor,line width= 0.6pt,line join=round] (282.91, 28.76) --
	(282.91, 31.51);

\path[draw=drawColor,line width= 0.6pt,line join=round] (403.51, 28.76) --
	(403.51, 31.51);
\end{scope}
\begin{scope}
\path[clip] (  0.00,  0.00) rectangle (505.89,505.89);
\definecolor{drawColor}{gray}{0.30}

\node[text=drawColor,anchor=base,inner sep=0pt, outer sep=0pt, scale=  0.73] at ( 41.70, 20.49) {0};

\node[text=drawColor,anchor=base,inner sep=0pt, outer sep=0pt, scale=  0.73] at (162.30, 20.49) {5};

\node[text=drawColor,anchor=base,inner sep=0pt, outer sep=0pt, scale=  0.73] at (282.91, 20.49) {10};

\node[text=drawColor,anchor=base,inner sep=0pt, outer sep=0pt, scale=  0.73] at (403.51, 20.49) {15};
\end{scope}
\begin{scope}
\path[clip] (  0.00,  0.00) rectangle (505.89,505.89);
\definecolor{drawColor}{RGB}{0,0,0}

\node[text=drawColor,anchor=base,inner sep=0pt, outer sep=0pt, scale=  0.92] at (234.66,  7.83) {Лаг};
\end{scope}
\begin{scope}
\path[clip] (  0.00,  0.00) rectangle (505.89,505.89);
\definecolor{drawColor}{RGB}{0,0,0}

\node[text=drawColor,rotate= 90.00,anchor=base,inner sep=0pt, outer sep=0pt, scale=  0.92] at ( 13.08,257.48) {Корреляция};
\end{scope}
\begin{scope}
\path[clip] (  0.00,  0.00) rectangle (505.89,505.89);
\definecolor{fillColor}{RGB}{255,255,255}

\path[fill=fillColor] (443.34,231.89) rectangle (500.39,283.08);
\end{scope}
\begin{scope}
\path[clip] (  0.00,  0.00) rectangle (505.89,505.89);
\definecolor{drawColor}{RGB}{255,255,255}
\definecolor{fillColor}{gray}{0.95}

\path[draw=drawColor,line width= 0.6pt,line join=round,line cap=round,fill=fillColor] (448.84,254.73) rectangle (466.18,272.08);
\end{scope}
\begin{scope}
\path[clip] (  0.00,  0.00) rectangle (505.89,505.89);
\definecolor{fillColor}{RGB}{248,118,109}

\path[fill=fillColor] (449.55,255.45) rectangle (465.47,271.37);
\end{scope}
\begin{scope}
\path[clip] (  0.00,  0.00) rectangle (505.89,505.89);
\definecolor{drawColor}{RGB}{255,255,255}
\definecolor{fillColor}{gray}{0.95}

\path[draw=drawColor,line width= 0.6pt,line join=round,line cap=round,fill=fillColor] (448.84,237.39) rectangle (466.18,254.73);
\end{scope}
\begin{scope}
\path[clip] (  0.00,  0.00) rectangle (505.89,505.89);
\definecolor{fillColor}{RGB}{0,191,196}

\path[fill=fillColor] (449.55,238.10) rectangle (465.47,254.02);
\end{scope}
\begin{scope}
\path[clip] (  0.00,  0.00) rectangle (505.89,505.89);
\definecolor{drawColor}{RGB}{0,0,0}

\node[text=drawColor,anchor=base west,inner sep=0pt, outer sep=0pt, scale=  0.73] at (471.68,260.38) {ACF};
\end{scope}
\begin{scope}
\path[clip] (  0.00,  0.00) rectangle (505.89,505.89);
\definecolor{drawColor}{RGB}{0,0,0}

\node[text=drawColor,anchor=base west,inner sep=0pt, outer sep=0pt, scale=  0.73] at (471.68,243.03) {PACF};
\end{scope}
\begin{scope}
\path[clip] (  0.00,  0.00) rectangle (505.89,505.89);
\definecolor{drawColor}{RGB}{0,0,0}

\node[text=drawColor,anchor=base west,inner sep=0pt, outer sep=0pt, scale=  1.10] at ( 36.99,491.30) {Автокорреляционная функция для уровня воды в озере Гурон};
\end{scope}
\end{tikzpicture}

\end{tikzpicture}
\end{center}
\end{minipage}

\begin{enumerate}
\item Судя по графикам, какие модели класса ARMA или ARIMA имеет смысл оценить?
\item По результатам оценки некоей модели ARMA c двумя параметрами, исследователь посчитал оценки автокорреляционной функции для остатков модели. Известно, что для остатков модели первые три выборочные автокорреляции равны соответственно $0.0047$, $-0.0129$, $-0.0630$.
С помощью подходящей статистики проверьте гипотезу о том, что первые три корреляции ошибок модели равны нулю.
\end{enumerate}


\begin{sol}
\begin{enumerate}
\item Процесс $AR(2)$, т.к. две первые частные корреляции значимо отличаются от нуля, а гипотезы о том, что каждая последующая равна нулю не отвергаются.
\item Можно использовать одну из двух статистик
\[
\text{Ljung-Box}=n(n+2)\sum_{k=1}^3\frac{\hat{\rho}_k^2}{n-k}=0.43
\]

\begin{minted}[mathescape, numbersep=5pt, frame=lines, framesep=2mm]{r}
n * (n + 2) * sum(resid.acf[2:4]^2/ ((n - 1) : (n - 3)))
\end{minted}


\[
\text{Box-Pierce}=n\sum_{k=1}^3\hat{\rho}_k^2=0.41
\]

\begin{minted}[mathescape, numbersep=5pt, frame=lines, framesep=2mm]{r}
n * sum(resid.acf[2:4]^2)
\end{minted}

Критическое значение хи-квадрат распределения с 3-мя степенями свободы для $\alpha=0.05$ равно $\chi^2_{3,crit}=7.81$.

\begin{minted}[mathescape, numbersep=5pt, frame=lines, framesep=2mm]{r}
qchisq(0.95, df = 3)
\end{minted}

Вывод: гипотеза $H_0$ об отсутствии корреляции ошибок модели не отвергается.
\end{enumerate}
\end{sol}
\end{problem}






\begin{problem}
Винни-Пух пытается выявить закономерность в количестве придумываемых им каждый день ворчалок.  Винни-Пух решил разобраться, является ли оно стационарным процессом, для этого он оценил регрессию

\[ 
  \Delta \hy_t = \underset{(0.5)}{4.5} - \underset{(0.1)}{0.4}y_{t-1} +\underset{(0.5)}{0.7} \Delta y_{t-1} 
\]

Из-за опилок в голове Винни-Пух забыл, какой тест ему нужно провести, то ли Доктора Ватсона, то ли Дикого Фуллера.

\begin{enumerate}
\item Аккуратно сформулируйте основную и альтернативную гипотезы.
\item Проведите подходящий тест на уровне значимости 5\%.
\item Сделайте вывод о стационарности ряда.
\item Почему Сова не советовала Винни-Пуху пользоваться широко применяемым в Лесу $t$-распределением?
\end{enumerate}


\begin{sol}

\begin{enumerate}
\item $H_0$: ряд содержит единичный корень, $\beta=0$; $H_a$: ряд не содержит единичного корня, $\beta<0$
\item $ADF=-0.4/0.1=-4$, $ADF_{crit}=-2.89$, $H_0$ отвергается.

\begin{minted}[mathescape, numbersep=5pt, frame=lines, framesep=2mm]{r}
qunitroot(0.05, N = 100, trend = "c", statistic = "t")
\end{minted}

\item Ряд стационарен.
\item При верной $H_0$ ряд не стационарен, и  $t$-статистика имеет не $t$-распределение, а распределение Дики-Фуллера.
\end{enumerate}
\end{sol}
\end{problem}




\begin{problem}
Рассматривается модель $y_t=\beta_1+\beta_2 x_{t1}+\ldots+\beta_k x_{tk}+\e_t$. 
Ошибки $\e_t$ гомоскедастичны, но в них возможно присутствует автокорреляция первого порядка, $\e_t=\rho \e_{t-1}+u_t$. 
При известном числе наблюдений $T$ на уровне значимости 5\% 
сделайте статистический вывод о наличии автокорреляции в следующих случаях:
\begin{enumerate}
\item $T=25$, $k=2$, $DW=0.8$;
\item $T=30$, $k=3$, $DW=1.6$;
\item $T=50$, $k=4$, $DW=1.8$;
\item $T=100$, $k=5$, $DW=1.1$.
\end{enumerate}


\begin{sol}
\end{sol}
\end{problem}



\begin{problem}
По 100 наблюдениям была оценена модель линейной регрессии
$y_t=\beta_1+\beta_2 x_t+\e_t$. Оказалось, что $RSS=120$, $\he_1=-1$, $\he_{100}=2$, $\sum_{t=2}^{100} \he_t\he_{t-1}=-50$. 
Найдите $DW$ и $\rho$.


\begin{sol}
\end{sol}
\end{problem}



\begin{problem}
Применима ли статистика Дарбина-Уотсона для выявления автокорреляции в следующих моделях:
\begin{enumerate}
\item $y_t=\beta_1 x_t + \e_t$;
\item $y_t=\beta_1 + \beta_2 x_t + \e_t$;
\item $y_t=\beta_1 + \beta_2 y_{t-1} + \e_t$;
\item $y_t=\beta_1 + \beta_2 t +\beta_3 y_{t-1} + \e_t$;
\item $y_t=\beta_1 t + \beta_2 x_t + \e_t$;
\item $y_t=\beta_1 + \beta_2 t +\beta_3 x_t +\beta_4 x_{t-1} + \e_t$?
\end{enumerate}


\begin{sol}
\end{sol}
\end{problem}



\begin{problem}
По 21 наблюдению была оценена модель линейной регрессии
$\underset{(se)}{\hy}=\underset{(0.3)}{1.2}+\underset{(0.18)}{0.9}\cdot y_{t-1}+\underset{(0.01)}{0.1}\cdot t$, $R^2=0.6$, $DW=1.21$. Протестируйте гипотезу об отсутствии автокорреляции ошибок на уровне значимости 5\%.


\begin{sol}
\end{sol}
\end{problem}




\begin{problem}
По 24 наблюдениям была оценена модель линейной регрессии
$\underset{(se)}{\hy}=\underset{(0.01)}{0.5}+\underset{(0.02)}{2}\cdot t$, $R^2=0.9$, $DW=1.3$. Протестируйте гипотезу об отсутствии автокорреляции ошибок на уровне значимости 5\%.


\begin{sol}
\end{sol}
\end{problem}



\begin{problem}
По 32 наблюдениям была оценена модель линейной регрессии
$\underset{(se)}{\hy}=\underset{(2.5)}{10}+\underset{(0.5)}{2.5}\cdot t- \underset{(0.01)}{0.1}\cdot t^2$, $R^2=0.75$, $DW=1.75$. 
Протестируйте гипотезу об отсутствии автокорреляции ошибок на уровне значимости 5\%.


\begin{sol}
\end{sol}
\end{problem}



\begin{problem}
Рассмотрим модель $y_t=\beta_1+\beta_2 x_{t1}+\ldots+\beta_k x_{tk}+\e_t$, 
где $\e_t$ подчиняются автокорреляционной схеме первого порядка, то есть:
\begin{enumerate}
\item $\e_t=\rho \e_{t-1}+u_t$, $-1<\rho<1$;
\item $\Var(\e_t)=const$, $\E(\e_t)=const$;
\item $\Var(u_t)=\sigma^2$, $\E(u_t)=0$;
\item Величины $u_t$ независимы между собой;
\item Величины $u_t$ и $\e_s$ независимы, если $t\geq s$.
\end{enumerate}
Найдите:
\begin{enumerate}
\item $\E(\e_t)$, $\Var(\e_t)$;
\item $\Cov(\e_t,\e_{t+h})$;
\item $\Corr(\e_t,\e_{t+h})$.
\end{enumerate}


\begin{sol}
\begin{enumerate}
\item $\E(\e_t)=0$, $\Var(\e_t)=\sigma^2/(1-\rho^2)$;
\item $\Cov(\e_t,\e_{t+h})=\rho^h\cdot \sigma^2/(1-\rho^2)$;
\item $\Corr(\e_t,\e_{t+h})=\rho^h$.
\end{enumerate}
\end{sol}
\end{problem}




\begin{problem}
Ошибки в модели $y_t=\beta_1+\beta_2 x_{t}+\e_t$ являются автокоррелированными первого порядка, $\e_t=\rho \e_{t-1}+u_t$. Шаман-эконометрист Ойуун выполняет два камлания-преобразования. Поясните смысл камланий:
\begin{enumerate}
\item Камлание А, при $t\geq 2$, Ойуун преобразует уравнение к виду $y_t-\rho y_{t-1}=\beta_1(1-\rho)+ \beta_2(x_t-\rho x_{t-1})+\e_t-\rho \e_{t-1}$;
\item Камлание Б, при $t=1$, Ойуун преобразует уравнение к виду $\sqrt{1-\rho^2}y_1=\sqrt{1-\rho^2}\beta_1+\sqrt{1-\rho^2}\beta_2 x_1+\sqrt{1-\rho^2}\e_1$.
\end{enumerate}


\begin{sol}
\end{sol}
\end{problem}



\begin{problem}
Пусть $y_{t}$ — стационарный процесс. Верно ли, что стационарны процессы:
\begin{enumerate}
\item $z_{t}=2y_{t}$;
\item $z_{t}=y_{t}+1$;
\item $z_{t}=\Delta y_{t}$;
\item $z_{t}=2y_{t}+3y_{t-1}$?
\end{enumerate}


\begin{sol}
Все линейные комбинации стационарны.
\end{sol}
\end{problem}






\begin{problem}
Известно, что временной ряд $y_{t}$ порожден стационарным процессом, задаваемым соотношением $y_{t}=1+0.5y_{t-1}+\varepsilon_{t}$. Имеется 1000 наблюдений. Вася построил регрессию $y_{t}$ на константу и $y_{t-1}$. Петя построил регрессию на константу и $y_{t+1}$. Какие примерно оценки коэффициентов они получат?


\begin{sol}
Они будут примерно одинаковы. Оценка наклона определяется автоковариационной функцией.
\end{sol}
\end{problem}




%%%%%%%%%%%%%%%%%% ARCH-GARCH

\begin{problem}
Рассмотрим следующий AR(1)-ARCH(1) процесс:
\[
\begin{cases}
  y_{t}=1+0.5y_{t-1}+\varepsilon_{t} \\
  \varepsilon_{t}=\nu_{t}\cdot \sigma_{t} \\
  \sigma^{2}_{t}=1+0.8\varepsilon^{2}_{t-1} \\
  \nu_t \sim \cN(0;1). \\
\end{cases}
\]

Также известно, что $y_{100}=2$, $y_{99}=1.7$
\begin{enumerate}
\item Найдите $\E(\varepsilon^{2}_{101} \mid \mathcal{F}_{100})$, \
$\E(\varepsilon^{2}_{102} \mid \mathcal{F}_{100})$, $\E(\varepsilon^{2}_{103} \mid \mathcal{F}_{100})$, 
$\E(\varepsilon^{2}_{t})$.
\item $\Var(y_{t})$, $\Var(y_{t}|\mathcal{F}_{t-1})$.
\item Постройте доверительный интервал для $y_{101}$:
\begin{enumerate}
\item проигнорировав условную гетероскедастичность;
\item учтя условную гетерескедастичность.
\end{enumerate}
\end{enumerate}


\begin{sol}
\end{sol}
\end{problem}



%%%%%%%%%%%%%%%%% Оператор лага

\begin{problem}
Пусть $x_{t}$, $t=0,1,2,\ldots$ - случайный процесс и $y_{t}=(1+\L )^{t}x_{t}$.
Выразите $x_{t}$ с помощью $y_{t}$ и оператора лага $\L $.


\begin{sol}
$x_{t}=(1-\L )^{t}y_{t}$
\end{sol}
\end{problem}



\begin{problem}
Пусть $ F_{n} $ — последовательность чисел Фибоначчи. Упростите величину
\[ F_{1}+C^{1}_{5}F_{2}+C^{2}_{5}F_{3}+C^{3}_{5}F_{4}+C^{4}_{5}F_{5}+C^{5}_{5}F_{6} \]


\begin{sol}
$ F_{n}=\L (1+\L )F_{n} $, значит $ F_{n}=\L ^{k}(1+\L )^{k}F_{n} $ или $ F_{n+k}=(1+\L )^{k}F_{n} $
\end{sol}
\end{problem}




\begin{problem}
Пусть $y_{t}$, $t=\ldots -2, -1, 0, 1, 2, \ldots$ - случайный процесс. И $y_{t}=x_{-t}$. Являются ли верными рассуждения:
\begin{enumerate}
\item $\L y_{t}=\L x_{-t}=x_{-t-1}$;
\item $\L y_{t}=y_{t-1}=x_{-t+1}$?
\end{enumerate}


\begin{sol}
а - неверно, б - верно.
\end{sol}
\end{problem}




%%%%%%%%%%%%%%%% состояние-наблюдение, фильтр Калмана


\begin{problem}
Представьте процесс AR(1),
$y_{t}=0.9y_{t-1}-0.2y_{t-2}+\varepsilon_{t}$, где $\e_t$ —  белый шум с единичной дисперсией,
в виде модели состояние-наблюдение.
\begin{enumerate}
\item  Выбрав в качестве состояний вектор $\left(%
\begin{array}{c}
  y_{t} \\
  y_{t-1} \\
\end{array}%
\right)$ \\
\item Выбрав в качестве состояний вектор $\left(%
\begin{array}{c}
  y_{t} \\
  \hy_{t,1} \\
\end{array}%
\right)$
\end{enumerate}
Найдите дисперсии ошибок состояний.


\begin{sol}
\end{sol}
\end{problem}


\begin{problem}
Представьте процесс MA(1),
$y_{t}=\varepsilon_{t}+0.5\varepsilon_{t-1}$, где $\e_t$ —   белый шум с единичной дисперсией,
	в виде модели состояние-наблюдение.
\begin{enumerate}
\item $\left(%
\begin{array}{c}
  \varepsilon_{t} \\
  \varepsilon_{t-1} \\
\end{array}%
\right)$ \\
\item $\left(%
\begin{array}{c}
  \varepsilon_{t}+0.5\varepsilon_{t-1} \\
  0.5\varepsilon_{t}
\end{array}%
\right)$
\end{enumerate}


\begin{sol}
\end{sol}
\end{problem}


\begin{problem}
Представьте процесс ARMA(1,1),
$y_{t}=0.5y_{t-1}+\varepsilon_{t}+\varepsilon_{t-1}$,
где процесс $\varepsilon_t$ —  белый шум с единичной дисперсией, в
виде модели состояние-наблюдение.

Вектор состояний имеет вид $x_{t},x_{t-1}$, где
$x_{t}=\frac{1}{1-0.5L}\varepsilon_{t}$.


\begin{sol}
\end{sol}
\end{problem}



\begin{problem}
Рекурсивные коэффициенты
\begin{enumerate}
\item Oцените модель вида $y_{t}=a+b_{t}x_{t}+\varepsilon_{t}$,
где $b_{t}=b_{t-1}$.
\item Сравните графики filtered state и smoothed state.
\item Сравните финальное состояние $b_{T}$ с коэффициентом в
обычной модели линейной регрессии, $y_{t}=a+bx_{t}+\varepsilon_{t}$.
\end{enumerate}

\todo[inline]{доделать state-space задачу}

\begin{sol}
\end{sol}
\end{problem}



\begin{problem}
Пусть $u_t$ — независимые нормальные случайные величины с
математическим ожиданием $0$ и дисперсией $\sigma^2$. Известно, что $\e_1=u_1$, $\e_t=u_1+u_2+\ldots+u_t$. Рассмотрим модель $y_t=\beta_1+\beta_2 x_t + \e_t$.

\begin{enumerate}
\item Найдите $\Var(\e_t)$, $\Cov(\e_t,\e_s)$, $\Var(\e)$.
\item Являются ли ошибки $\e_t$ гетероскедастичными?
\item Являются ли ошибки $\e_t$ автокоррелированными?
\item Предложите более эффективную оценку вектора коэффициентов регрессии по сравнению МНК-оценкой.
\item Результаты предыдущего пункта подтвердите симуляциями Монте-Карло на компьютере.
\end{enumerate}


\begin{sol}
\end{sol}
\end{problem}



\begin{problem}
Найдите безусловную дисперсию GARCH-процессов
\begin{enumerate}
\item $\e_t=\sigma_t \cdot \z_t$, $\sigma^2_t=0.1+0.8\sigma^2_{t-1}+0.1\e^2_{t-1}$;
\item $\e_t=\sigma_t \cdot \z_t$, $\sigma^2_t=0.4+0.7\sigma^2_{t-1}+0.1\e^2_{t-1}$;
\item $\e_t=\sigma_t \cdot \z_t$, $\sigma^2_t=0.2+0.8\sigma^2_{t-1}+0.1\e^2_{t-1}$.
\end{enumerate}


\begin{sol}
$1$, $2$, $2$
\end{sol}
\end{problem}



\begin{problem}
Являются ли верными следующие утверждения?
\begin{enumerate}
\item GARCH-процесс является процессом белого шума, условная дисперсия которого
изменяется во времени.
\item Модель GARCH(1,1) предназначена для прогнозирования меры изменчивости цены
финансового инструмента, а не для прогнозирования самой цены инструмента.
\item При помощи GARCH-процесса можно устранять гетероскедастичность.
\item Безусловная дисперсия GARCH-процесса изменяется во времени.
\item Модель GARCH(1,1) может быть использована для прогнозирования
волатильности финансовых инструментов на несколько торговых недель вперёд.
\end{enumerate}


\begin{sol}
\end{sol}
\end{problem}



\begin{problem}
Рассмотрим GARCH-процесс $\e_t=\sigma_t \cdot \z_t$, $\sigma^2_t=k+g_1\sigma^2_{t-1}+a_1\e^2_{t-1}$. Найдите
\begin{enumerate}
\item $\E(\z_t)$, $\E(\z_t^2)$, $\E(\e_t)$, $\E(\e_t^2)$;
\item $\Var(\z_t)$, $\Var(\e_t)$, $\Var(\e_t \mid \mathcal{F}_{t-1})$;
\item $\E(\e_t \mid \mathcal{F}_{t-1})$, $\E(\e_t^2 \mid \mathcal{F}_{t-1})$, $\E(\sigma^2_t \mid \mathcal{F}_{t-1})$;
\item $\E(\z_t\z_{t-1})$, $\E(\z_t^2\z_{t-1}^2)$, $\Cov(\e_t,\e_{t-1})$, $\Cov(\e_t^2,\e_{t-1}^2)$;
\item $\lim_{h\to\infty} \E(\sigma^2_{t+h} \mid \mathcal{F}_t)$.
\end{enumerate}


\begin{sol}
\end{sol}
\end{problem}



\begin{problem}
Используя 500 наблюдений дневных логарифмических доходностей $y_t$ ,
была оценена GARCH(1,1)-модель: $\hy_t=-0.000708+\he_t$, $\e_t=\sigma_t \cdot \z_t$, $\sigma^2_t=0.000455+0.6424\sigma^2_{t-1}+0.2509\e^2_{t-1}$. Также известно, что $\hs^2_{499}=0.002568$, $\he^2_{499}=0.000014$, $\he^2_{500}=0.002178$.
Найдите
\begin{enumerate}
\item  $\hs^2_{500}$, $\hs^2_{501}$, $\hs^2_{502}$;
\item Волатильность в годовом выражении в процентах, соответствующую
наблюдению с номером $t = 500$.
\end{enumerate}


\begin{sol}
\end{sol}
\end{problem}



\begin{problem}
Докажите, что в условиях автокорреляции ошибок МНК-оценки остаются несмещёнными.
\begin{sol}
Несмещённость доказывается с помощью подсчёта $\E(\hb)$, а предпосылка о $\E(\e)=0$ при автокорреляции ошибок не нарушена.
\end{sol}
\end{problem}



\begin{problem}
Продавец мороженного оценил динамическую модель объёмов продаж:
\[
\ln \hat{Q}_t=26.7 + 0.2\ln \hat{Q}_{t-1}-0.6\ln P_t
\]
Здесь $Q_t$ — число проданных в день $t$ вафельных стаканчиков, а $P_t$ — цена одного стаканчика в рублях. Продавец также рассчитал остатки $\hat{e}_t$.
\begin{enumerate}
\item Чему, согласно полученным оценкам, равна долгосрочная эластичность объёма продаж по цене?
\item Предположим, что продавец решил проверить наличие автокорреляции первого порядка с помощью теста Бройша-Годфри. Выпишите уравнение регрессии, которое он должен оценить.
\end{enumerate}


\begin{sol}
\end{sol}
\end{problem}


\begin{problem}
Рассматривается модель $y_t = \mu + \varepsilon_t$, $t = 1,\ldots,T$, где $\varepsilon_t = \rho \varepsilon_{t-1} + u_t$, случайные величины $\varepsilon_0, u_1,\dots,u_T$ независимы, причём $\varepsilon_0 \sim \cN(0,\sigma^2/(1 - \rho^2))$, $u_t \sim \cN(0,\sigma^2)$. Имеются наблюдения $y' = (1, 2, 0, 0, 1)$.
\begin{enumerate}
  \item Выпишите функцию правдоподобия
  \[
  \mathrm{L}(\mu, \rho, \sigma^2) = f_{Y_1}(y_1)\prod_{t=2}^{T}f_{Y_t|Y_{t-1}}(y_t|y_{t-1}).
  \]
  \item Найдите оценки неизвестных параметров модели максимизируя условную функцию правдоподобия
  \[
  \mathrm{L}(\mu, \rho, \sigma^2|Y_1 = y_1) = \prod_{t=2}^{T}f_{Y_t|Y_{t-1}}(y_t|y_{t-1}).
  \]
\end{enumerate}


\begin{sol}
\begin{enumerate}
\item Поскольку имеют место соотношения $\varepsilon_1 = \rho \varepsilon_0 + u_1$ и $Y_1 =\mu + \varepsilon_1$, то из условия задачи получаем, что $\varepsilon_1 \sim \cN(0,\sigma^2 / (1 - \rho^2))$
и $Y_1 \sim \cN\mu,\sigma^2 / (1 - \rho^2))$. Поэтому
\[
f_{Y_1}(y_1) = \frac{1}{\sqrt{2\pi\sigma^2/(1-\rho^2)}}\exp{\left(-\frac{(y_1 - \mu)^2}{2\sigma^2/(1 - \rho^2)}\right)}.
\]

Далее, найдем $f_{Y_2|Y_1}(y_2|y_1)$. Учитывая, что $Y_2 = \rho Y_1 + (1- \rho) \mu + u_2$, получаем $Y_2|\{Y_1 = y_1\} \sim \cN\rho y_1 + (1- \rho) \mu, \sigma^2)$. Значит,
\[
f_{Y_2|Y_1}(y_2|y_1) = \frac{1}{\sqrt{2\pi\sigma^2}}\exp{\left(-\frac{(y_2 - \rho y_1 - (1- \rho) \mu)^2}{2\sigma^2}\right)}.
\]

Действуя аналогично, получаем, что для всех $t \geq 2$ справедлива формула
\[
f_{Y_{t}|Y_{t-1}}(y_{t}|y_{t-1}) = \frac{1}{\sqrt{2\pi\sigma^2}}\exp{\left(-\frac{(y_{t} - \rho y_{t-1} - (1- \rho) \mu)^2}{2\sigma^2}\right)}.
\]

Таким образом, находим функцию правдоподобия
\[
\mathrm{L}(\mu, \rho, \sigma^2) = f_{Y_T,\ldots,Y_1}(y_T,\dots,y_1) = f_{Y_1}(y_1)\prod_{t=2}^{T}f_{Y_t|Y_{t-1}}(y_t|y_{t-1}) \text{,}
\]
где $f_{Y_1}(y_1)$ и $f_{Y_t|Y_{t-1}}(y_t|y_{t-1})$ получены выше.

\item Для нахождения неизвестных параметров модели запишем логарифмическую условную функцию правдоподобия:
\[
\ell(\mu, \rho, \sigma^2|Y_1 = y_1) = \sum_{t=2}^{T}\log{f_{Y_t|Y_{t-1}}(y_t|y_{t-1})} =
\]
\[
=-\frac{T-1}{2} \log(2 \pi) - \frac{T-1}{2} \log{\sigma^2} - \frac{1}{2\sigma^2} \sum_{t=2}^{T}(y_t - \rho y_{t-1} - (1 - \rho) \mu)^2 \text{.}
\]

Найдем производные функции $\ell(\mu, \rho, \sigma^2|Y_1 = y_1)$ по неизвестным параметрам:
\[
\frac{\partial \ell}{\partial \mu} = -\frac{1}{2\sigma^2} \sum_{t=2}^{T} 2(y_t - \rho y_{t-1} - (1 - \rho) \mu) \cdot (\rho - 1) \text{,}
\]
\[
\frac{\partial \ell}{\partial \rho} = -\frac{1}{2\sigma^2} \sum_{t=2}^{T} 2(y_t - \rho y_{t-1} - (1 - \rho) \mu) \cdot (\mu - y_{t-1}) \text{,}
\]
\[
\frac{\partial \ell}{\partial {\sigma^2}} =  - \frac{T-1}{2\sigma^2} + \frac{1}{2\sigma^4} \sum_{t=2}^{T}(y_t - \rho y_{t-1} - (1 - \rho) \mu)^2 \text{.}
\]

Оценки неизвестных параметров модели могут быть получены как решение следующей системы уравнений:
\[
\left\{
  \begin{aligned}
    \frac{\partial \ell}{\partial \mu} = 0 \text{,} \\
    \frac{\partial \ell}{\partial \rho} = 0 \text{,} \\
    \frac{\partial \ell}{\partial {\sigma^2}} = 0 \text{.}
  \end{aligned}
\right.
\]

Из первого уравнения системы получаем, что
\[
\sum_{t=2}^{T}y_{t} - \hat{\rho} \sum_{t=2}^{T}y_{t-1} = (T - 1) (1- \hat{\rho}) \hat{\mu} \text{,}
\]
откуда
\[
\hat{\mu} = \frac{\sum_{t=2}^{T}y_{t} - \hat{\rho} \sum_{t=2}^{T}y_{t-1}}{(T - 1) (1- \hat{\rho})} = \frac{3 - \hat{\rho} \cdot 3}{4\cdot(1-\hat{\rho})} = \frac{3}{4} \text{.}
\]

Далее, если второе уравнение системы переписать в виде
\[
\sum_{t=2}^{T}(y_t - \hat{\mu} - \hat{\rho} (y_{t-1} - \hat{\mu}))(y_{t-1} - \hat{\mu}) = 0 \text{,}
\]
то легко видеть, что
\[
\hat{\rho} = \frac{\sum_{t=2}^{T}(y_t - \hat{\mu})(y_{t-1} - \hat{\mu})}{\sum_{t=2}^{T}(y_{t-1} - \hat{\mu})^2} \text{.}
\]
Следовательно, $\hat{\rho} =-1/11= -0.0909$.

Наконец, из третьего уравнения системы
\[
\hs^2 =\frac{1}{T-1} \sum_{t=2}^{T}(y_t - \hat{\rho} y_{t-1} - (1 - \hat{\rho}) \hat{\mu})^2 \text{.}
\]
Значит, $\hs^2 = 165/242= 0.6818$. Ответы: $\hat{\mu} = 3/4= 0.75$, $\hat{\rho} = -1/11=-0.0909$, $\hs^2 =165/242=0.6818$.
\end{enumerate}
\end{sol}
\end{problem}



\begin{problem}
Была оценена AR(2) модель
\[
\hy_t=2.3+0.8 y_{t-1}-0.2 y_{t-2}.
\]
Дополнительно известно, что $se(\hb_{y_{t-1}})=0.3$ и $\hat{\rho}_1=0.7$. Найдите $se(\hb_{y_{t-2}})$ и $\hCov(\hb_{y_{t-2}},\hb_{y_{t-1}})$.


\begin{sol}
Рассмотрим модель без константы. Тогда ковариационная матрица коэффициентов пропорциональна матрице
\[
\begin{pmatrix}
1 & -\hat{\rho}_1 \\
-\hat{\rho}_1 & 1
\end{pmatrix}
\]
\end{sol}
\end{problem}




\begin{problem}
Рассмотрите следующие два утверждения:
\begin{itemize}
  \item GARCH-процесс является слабо стационарным процессом;
  \item GARCH-процесс является процессом с изменяющейся во времени условной дисперсией.
\end{itemize}
Поясните смысл каждого из них. Объясните, почему между ними нет противоречия.
\begin{sol}
\end{sol}
\end{problem}




\begin{problem}
Предложите способ, при помощи которого из моделей GARCH(1,1) и GARCH(2,1) можно выбрать лучшую.
\begin{sol}
\end{sol}
\end{problem}


\begin{problem}
Опишите тест, при помощи которого можно выявить необходимость использовать GARCH-модель.
\begin{sol}
\end{sol}
\end{problem}




\begin{problem}
Рассматривается GARCH(1,1)-процесс $\sigma_t^2 = 1 + 0.8 \cdot \sigma_{t-1}^2 + 0.1 \cdot \varepsilon_{t-1}^2$. Известно, что $\sigma_T^2 = 9$, $\varepsilon_T = -2$. Найдите
\begin{itemize}
  \item $\E(\sigma_{T+1}^2|\mathcal{F}_T$);
  \item $\E(\sigma_{T+2}^2|\mathcal{F}_T$);
  \item $\E(\sigma_{T+3}^2|\mathcal{F}_T$).
\end{itemize}


\begin{sol}
\end{sol}
\end{problem}



\begin{problem}
Рассмотрите два ряда цен интересующих вас финансовых инструментов, действующих в одной отрасли. Примером могут выступать цены обыкновенных акций Сбербанка и ВТБ. По данным для выбранных инструментов, содержащим не менее 250 наблюдений (за одни и тот же промежуток времени), рассчитайте при помощи GARCH-модели историческую волатильность в годовом выражении в процентах.
\begin{itemize}
  \item В одних координатных осях постройте графики полученных волатильностей.
  \item На основании графика, построенного в пункте (a), сделайте качественный вывод относительно риска каждого финансового инструмента.
  \item Для каждого из выбранных инструментов постройте прогноз волатильности (в годовом выражении в процентах) на три торговых дня вперед.
\end{itemize}


\begin{sol}
\end{sol}
\end{problem}


%%%%%%%%%%%%%%%%
%%%%%%%%%%%%%%%%
%%% Здесь начинаются новые задачи
%%%%%%%%%%%%%%%%%
%%%%%%%%%%%%%%%%%


\begin{problem}
Имеются данные $y=(1,\, 2,\, 0,\,  0,\, 2,\, 1)$. Предполагая модель с автокоррелированной ошибкой, $y_t=\mu+\e_t$, где $\e_t=\rho \e_{t-1}+u_t$ с помощью трёх тестов, LM, LR и Вальда, проверьте гипотезы:
\begin{enumerate}
\item $H_0$: $\rho=0$;
\item $H_0$: $\mu=0$;
\item $H_0$: $\begin{cases}
\rho=0 \\
\mu = 0 \\
\sigma^2=1
\end{cases}.$
\end{enumerate}



\begin{sol}

Для простоты закроем глаза на малое количество наблюдений и как индейцы пираха будем считать, что пять — это много.

\end{sol}
\end{problem}



\begin{problem}
Процесс $x_t$ — это процесс $y_t$, наблюдаемый с ошибкой, т.е. $x_t=y_t+\nu_t$. Ошибки $\nu_t$ являются белым шумом и не коррелированы с $y_t$.
\begin{enumerate}
\item Является ли процесс $x_t$ MA(1) процессом, если $y_t$ —  MA(1) процесс? Если да, то как связаны их автокорреляционные функциии?
\item Является ли процесс $x_t$ стационарным AR(1) процессом, если $y_t$ —  стационарный AR(1) процесс? Если да, то как связаны их автокорреляционные функциии?
\end{enumerate}


\begin{sol}

\end{sol}
\end{problem}


\begin{problem}
Пусть $\e_t$ — белый шум. Рассмотрим процесс $y_t=2+0.5y_{t-1}+\e_t$ с различными начальными условиями, указанными ниже.

\begin{enumerate}
\item Найдите $\E(y_t)$, $\Var(y_t)$ и определите, является ли процесс  стационарным, если:
\begin{enumerate}
\item $y_1=0$;
\item $y_1=4$;
\item $y_1=4+\e_1$;
\item $y_1=4+\frac{2}{\sqrt{3}}\e_1$.
\end{enumerate}
\item Как точно следует понимать фразу «процесс $y_t=2+0.5y_{t-1}+\e_t$ является стационарным»?
\end{enumerate}




\begin{sol}
Процесс стационарен только при $y_1=4+\frac{2}{\sqrt{3}}\e_1$. Фразу нужно понимать как «у стохастического разностного уравнения $y_t=2+0.5y_{t-1}+\e_t$ есть стационарное решение».
\end{sol}
\end{problem}



\begin{problem}
Рассмотрим модель $y_t=\beta x_t +\e_t$, где $\e_1=u_1$ и $\e_t=u_t+u_{t-1}$ при $t\geq 2$. Случайные величины $u_i$ независимы с $\E(u_i)=0$ и $\Var(u_i)=\sigma^2$.
\begin{enumerate}
\item Найдите $\Var(\e_t)$.
\item Являются ли ошибки $\e_t$ гетероскедастичными?
\item Найдите $\Cov(\e_i,\e_j)$.
\item Являются ли ошибки $\e_t$ автокоррелированными?
\item Как выглядит матрица $\Var(\e)$?
\item Рассмотрим оценку
\[
\hb=\frac{\sum x_i y_i}{\sum x_i^2}.
\]
Является ли она несмещённой для $\beta$? Является ли она эффективной в классе линейных по $y$ несмещённых оценок?
\item Если приведенная $\hb$ не является эффективной, то приведите формулу для эффективной оценки.
\end{enumerate}



\begin{sol}
\begin{enumerate}
\item $\E(\e_t)=0$, $\Var(\e_1)=\sigma^2$, $\Var(\e_t)=2\sigma^2$ при $t\geq 2$.
Ошибки гетероскедастичные.
\item $\Cov(\e_t,\e_{t+1})=\sigma^2$. Ошибки автокоррелированы.
\item $\hb$ — несмещённая, неэффективная
\item Более эффективной будет $\hb_{gls}=(X'V^{-1}X)^{-1}X'V^{-1}y$, где
\[
X=\begin{pmatrix}
x_1 \\
x_2 \\
\vdots \\
x_n
\end{pmatrix}
\]

Матрица $V$ известна с точностью до константы $\sigma^2$, но в формуле для $\hb_{gls}$ неизвестная $\sigma^2$ сократится.

Другой способ построить эффективную оценку — применить МНК к преобразованным наблюдениям, т.е. $\hb_{gls}=\frac{\sum x'_i y'_i}{\sum x_i^{\prime 2}}$, где $y'_1=y_1$, $x'_1=x_1$, $y'_t=y_t-y_{t-1}$, $x'_t=x_t-x_{t-1}$ при $t\geq 2$.
\end{enumerate}
\end{sol}
\end{problem}



\begin{problem}
Верно ли, что при удалении из стационарного ряда каждого второго наблюдения получается стационарный ряд?
\begin{sol}
Да, стационарный.
\end{sol}
\end{problem}



\begin{problem}
У эконометрессы Ефросиньи был стационарный ряд.
Ей было скучно и она подбрасывала неправильную монетку, выпадающую орлом с вероятностью $0.7$.
Если выпадал орёл, она оставляла очередной $y_t$, если решка — то зачёркивала.
Получается ли у Ефросиньи стационарный ряд?
\begin{sol}
Да, получается.
\end{sol}
\end{problem}


\begin{problem}
Имеется временной ряд, $\e_1$, $\e_2$, \ldots, $\e_{101}$. Величины $\e_t$ нормально распределены, $\cN(0,\sigma^2)$, и независимы. Построим график этого процесса.
\begin{enumerate}
\item Является ли этот процесс белым шумом?
\item Сколько в среднем раз график пересекает ось абсцисс?
\item Оцените вероятность того, что график пересечет ось абсцисс более 60 раз.
\end{enumerate}



\begin{sol}
Да, процесс $\e_t$ — белый шум. Количество пересечений оси, величина $N$, распределена биномиально,
$Bin(n = 100, p = 1/2)$, $\E(N) = 50$.
\end{sol}
\end{problem}


\begin{problem}
Рассмотрим стационарный AR(1) процесс $y_t=\rho y_{t-1} + \e_t$, где $\e_t \sim \cN(0,1)$. Иммется ряд $y_1$, $y_2$, \ldots, $y_{101}$. Построен график этого процесса. Как от $\rho$ зависит математическое ожидание количества пересечений графика с осью абсцисс?

\begin{sol}
Среднее количество пересечений равно 50 помножить на вероятность того, что два соседних $y_t$ разного знака. Найдём вдвое меньшу вероятность, $\P(y_1>0, y_2 <0)$.
\end{sol}
\end{problem}



\begin{problem}
Рассмотрим процессы:

\begin{enumerate}
\item Процесс скользящего среднего:
\[
y_t=\e_t+2\e_{t-1}+3
\]

\item Процесс случайного блуждания со смещением:
\[
\begin{cases}
z_t=\e_t+z_{t-1}+3 \\
z_0=0
\end{cases}
\]

\item Процесс с трендом:
\[
w_t=2+3t+\e_t
\]

\item Еще один процесс:
\[
r_t=\begin{cases}
1, \; \text{при четных t} \\
-1, \; \text{при нечетных t}
\end{cases}
\]

\item Приращение случайного блуждания
\[
s_t=\Delta z_t
\]

\item Приращение процесса с трендом
\[
d_t=\Delta w_t
\]
\end{enumerate}

Для каждого из процессов найдите $\E(y_t)$, $\Var(y_t)$, $\Corr(y_t,y_{t-k})$.

Являются ли процессы стационарными?

\begin{sol}
\end{sol}
\end{problem}





\begin{problem}
Эконометресса Антуанетта построила график автоковариационной функции временного ряда и распечатала его:

\todo[inline]{Здесь виснет чанк кода!}
\begin{minted}[mathescape, numbersep=5pt, frame=lines, framesep=2mm]{r}
acf.df <- data.frame(lag = 1:10, acf = c(11, 5, 3, 1, -2, 2, 1, 2, 0, 1))
ggplot(acf.df, aes(x = lag, y = acf)) +
    geom_bar(stat = "identity") +
  xlab("Лаг") + ylab("ACF") +
  guides(fill = guide_legend(title = NULL))
\end{minted}

\begin{minipage}{0.6\textwidth}
\begin{center}
\begin{tikzpicture}[scale = 0.025]
% Created by tikzDevice version 0.12 on 2019-05-28 10:05:15
% !TEX encoding = UTF-8 Unicode
\definecolor{fillColor}{RGB}{255,255,255}
\path[use as bounding box,fill=fillColor,fill opacity=0.00] (0,0) rectangle (505.89,505.89);
\begin{scope}
\path[clip] (  0.00,  0.00) rectangle (505.89,505.89);
\definecolor{drawColor}{RGB}{255,255,255}
\definecolor{fillColor}{RGB}{255,255,255}

\path[draw=drawColor,line width= 0.6pt,line join=round,line cap=round,fill=fillColor] (  0.00,  0.00) rectangle (505.89,505.89);
\end{scope}
\begin{scope}
\path[clip] ( 27.42, 31.51) rectangle (500.39,500.39);
\definecolor{fillColor}{gray}{0.92}

\path[fill=fillColor] ( 27.42, 31.51) rectangle (500.39,500.39);
\definecolor{drawColor}{RGB}{255,255,255}

\path[draw=drawColor,line width= 0.3pt,line join=round] ( 27.42, 52.82) --
	(500.39, 52.82);

\path[draw=drawColor,line width= 0.3pt,line join=round] ( 27.42,183.98) --
	(500.39,183.98);

\path[draw=drawColor,line width= 0.3pt,line join=round] ( 27.42,315.13) --
	(500.39,315.13);

\path[draw=drawColor,line width= 0.3pt,line join=round] ( 27.42,446.29) --
	(500.39,446.29);

\path[draw=drawColor,line width= 0.3pt,line join=round] ( 79.32, 31.51) --
	( 79.32,500.39);

\path[draw=drawColor,line width= 0.3pt,line join=round] (187.90, 31.51) --
	(187.90,500.39);

\path[draw=drawColor,line width= 0.3pt,line join=round] (296.48, 31.51) --
	(296.48,500.39);

\path[draw=drawColor,line width= 0.3pt,line join=round] (405.06, 31.51) --
	(405.06,500.39);

\path[draw=drawColor,line width= 0.6pt,line join=round] ( 27.42,118.40) --
	(500.39,118.40);

\path[draw=drawColor,line width= 0.6pt,line join=round] ( 27.42,249.55) --
	(500.39,249.55);

\path[draw=drawColor,line width= 0.6pt,line join=round] ( 27.42,380.71) --
	(500.39,380.71);

\path[draw=drawColor,line width= 0.6pt,line join=round] (133.61, 31.51) --
	(133.61,500.39);

\path[draw=drawColor,line width= 0.6pt,line join=round] (242.19, 31.51) --
	(242.19,500.39);

\path[draw=drawColor,line width= 0.6pt,line join=round] (350.77, 31.51) --
	(350.77,500.39);

\path[draw=drawColor,line width= 0.6pt,line join=round] (459.35, 31.51) --
	(459.35,500.39);
\definecolor{fillColor}{gray}{0.35}

\path[fill=fillColor] (222.64, 52.82) rectangle (261.73,118.40);

\path[fill=fillColor] ( 48.92,118.40) rectangle ( 88.01,479.08);

\path[fill=fillColor] ( 92.35,118.40) rectangle (131.44,282.34);

\path[fill=fillColor] (135.78,118.40) rectangle (174.87,216.76);

\path[fill=fillColor] (179.21,118.40) rectangle (218.30,151.19);

\path[fill=fillColor] (266.08,118.40) rectangle (305.16,183.98);

\path[fill=fillColor] (309.51,118.40) rectangle (348.60,151.19);

\path[fill=fillColor] (352.94,118.40) rectangle (392.03,183.98);

\path[fill=fillColor] (396.37,118.40) rectangle (435.46,118.40);

\path[fill=fillColor] (439.80,118.40) rectangle (478.89,151.19);
\end{scope}
\begin{scope}
\path[clip] (  0.00,  0.00) rectangle (505.89,505.89);
\definecolor{drawColor}{gray}{0.30}

\node[text=drawColor,anchor=base east,inner sep=0pt, outer sep=0pt, scale=  0.73] at ( 22.47,115.37) {0};

\node[text=drawColor,anchor=base east,inner sep=0pt, outer sep=0pt, scale=  0.73] at ( 22.47,246.52) {4};

\node[text=drawColor,anchor=base east,inner sep=0pt, outer sep=0pt, scale=  0.73] at ( 22.47,377.68) {8};
\end{scope}
\begin{scope}
\path[clip] (  0.00,  0.00) rectangle (505.89,505.89);
\definecolor{drawColor}{gray}{0.20}

\path[draw=drawColor,line width= 0.6pt,line join=round] ( 24.67,118.40) --
	( 27.42,118.40);

\path[draw=drawColor,line width= 0.6pt,line join=round] ( 24.67,249.55) --
	( 27.42,249.55);

\path[draw=drawColor,line width= 0.6pt,line join=round] ( 24.67,380.71) --
	( 27.42,380.71);
\end{scope}
\begin{scope}
\path[clip] (  0.00,  0.00) rectangle (505.89,505.89);
\definecolor{drawColor}{gray}{0.20}

\path[draw=drawColor,line width= 0.6pt,line join=round] (133.61, 28.76) --
	(133.61, 31.51);

\path[draw=drawColor,line width= 0.6pt,line join=round] (242.19, 28.76) --
	(242.19, 31.51);

\path[draw=drawColor,line width= 0.6pt,line join=round] (350.77, 28.76) --
	(350.77, 31.51);

\path[draw=drawColor,line width= 0.6pt,line join=round] (459.35, 28.76) --
	(459.35, 31.51);
\end{scope}
\begin{scope}
\path[clip] (  0.00,  0.00) rectangle (505.89,505.89);
\definecolor{drawColor}{gray}{0.30}

\node[text=drawColor,anchor=base,inner sep=0pt, outer sep=0pt, scale=  0.73] at (133.61, 20.49) {2.5};

\node[text=drawColor,anchor=base,inner sep=0pt, outer sep=0pt, scale=  0.73] at (242.19, 20.49) {5.0};

\node[text=drawColor,anchor=base,inner sep=0pt, outer sep=0pt, scale=  0.73] at (350.77, 20.49) {7.5};

\node[text=drawColor,anchor=base,inner sep=0pt, outer sep=0pt, scale=  0.73] at (459.35, 20.49) {10.0};
\end{scope}
\begin{scope}
\path[clip] (  0.00,  0.00) rectangle (505.89,505.89);
\definecolor{drawColor}{RGB}{0,0,0}

\node[text=drawColor,anchor=base,inner sep=0pt, outer sep=0pt, scale=  0.92] at (263.90,  7.83) {Лаг};
\end{scope}
\begin{scope}
\path[clip] (  0.00,  0.00) rectangle (505.89,505.89);
\definecolor{drawColor}{RGB}{0,0,0}

\node[text=drawColor,rotate= 90.00,anchor=base,inner sep=0pt, outer sep=0pt, scale=  0.92] at ( 13.08,265.95) {ACF};
\end{scope}

\end{tikzpicture}
\end{center}
\end{minipage}


Потом она с ужасом обнаружила, что до презентации исследования остается совсем мало времени, а распечатать надо было график автокорреляционной функции. Что надо исправить Антуанетте на графике, чтобы успеть еще сделать причёску и макияж (это очень важно для презентации)?


\begin{sol}
Приписать нолики и точки на вертикальной оси.
\end{sol}
\end{problem}


\begin{problem}
Билл Гейтс оценил модель $y_t=\beta_1 + \beta_2 t + \beta_3 y_{t-1} + \e_t$ с помощью МНК. Значение статистики Дарбина-Уотсона оказалось равно $DW=0.55$. Какой из этого следует вывод об автокорреляции ошибок первого порядка?


\begin{sol}
В данном случае статистика $DW$ не применима, так как есть лаг $y_{t-1}$ среди регрессоров.
\end{sol}
\end{problem}

\Closesolutionfile{solution_file}

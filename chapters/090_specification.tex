\Opensolutionfile{solution_file}[solutions/sols_090]
% в квадратных скобках фактическое имя файла

\chapter{Ошибки спецификации}


\begin{problem}
По $25$ наблюдениям при помощи метода наименьших квадратов оценена
модель $\hy=\hb_1+\hb_2 x +\hb_3 z$, для которой $RSS = 73$. При помощи вспомогательной регрессии $\hat{\hy}=\hat{\gamma}_1+\hat{\gamma}_2 x +\hat{\gamma}_3 z+\hat{\gamma}_4 \hy^2$, для которой $RSS = 70$, выполните тест Рамсея на уровне значимости 5\%.


\begin{sol}
Змаетим, что модель $\hy=\hb_1+\hb_2 x +\hb_3 z$ — ограниченная, $RSS_R = 73$, а модель  $\hat{\hy}=\hat{\gamma}_1+\hat{\gamma}_2 x +\hat{\gamma}_3 z+\hat{\gamma}_4 \hy^2$ — неограниченная, $RSS_{UR} = 70$.
Проверяем гипотезу
\[
H_0: \gamma_4 = 0
\]
Посчитаем F-статистику, которая при верной $H_0$ имеет распределение $F_{1, 21}$:
\[
F_{obs} = \frac{(RSS_{R} - RSS_{UR})/q}{RSS_{UR}/(n-k_{UR})} = \frac{(73-70)/1}{70/(25-4)} = 0.9
\]
Поскольку $F_{obs} < F_{crit} = 4.32$, оснований отвергать $H_0$ нет.
\end{sol}
\end{problem}



\begin{problem}
По $20$ наблюдениям при помощи метода наименьших квадратов оценена
модель $\hy=\hb_1+\hb_2 x +\hb_3 z$, для которой $R^2 = 0.7$. При помощи вспомогательной регрессии $\hat{\hy}=\hat{\gamma}_1+\hat{\gamma}_2 x +\hat{\gamma}_3 z+\hat{\gamma}_4 \hy^2$, для которой $R^2 = 0.75$, выполните тест Рамсея на уровне значимости 5\%.

\begin{sol}
Аналогично предыдущей задаче:
\[
R: \hy=\hb_1+\hb_2 x +\hb_3 z, R^2_R = 0.7
\]
\[
UR: \hat{\hy}=\hat{\gamma}_1+\hat{\gamma}_2 x +\hat{\gamma}_3 z+\hat{\gamma}_4 \hy^2, R^2_{UR} = 0.75
\]
Проверяем гипотезу
\[
H_0: \gamma_4 = 0
\]
Посчитаем F-статистику, которая при верной $H_0$ имеет распределение $F_{1, 16}$:
\[
F_{obs} = \frac{(RSS_{R} - RSS_{UR})/q}{RSS_{UR}/(n-k_{UR})} = \frac{(R^2_{UR} - R^2_{R})/q}{(1-R^2_{UR})/(n-k_{UR})} = \frac{(0.75-0.7)/1}{(1-0.75)/(20-4)} = 3.2
\]
Поскольку $F_{obs} < F_{crit} = 4.49$, оснований отвергать $H_0$ нет.
\end{sol}
\end{problem}



\begin{problem}
По $30$ наблюдениям при помощи метода наименьших квадратов оценена
модель $\hy=\hb_1+\hb_2 x +\hb_3 z$, для которой $RSS = 150$. При помощи вспомогательной регрессии $\hat{\hy}=\hat{\gamma}_1+\hat{\gamma}_2 x +\hat{\gamma}_3 z+\hat{\gamma}_4 \hy^2+\hat{\gamma}_5 \hy^3$, для которой $RSS = 120$, выполните тест Рамсея на уровне значимости 5\%.

\begin{sol}
Задача решается аналогично 9.1, но теперь проверяется гипотеза с двумя ограничениями:
\[
H_0:
\begin{cases}
\gamma_4 = 0 \\
\gamma_5 = 0
\end{cases}
\]
Посчитаем F-статистику, которая при верной $H_0$ имеет распределение $F_{2, 25}$:
\[
F_{obs} = \frac{(150-120)/2}{120/(35-5)} = 6.25
\]
Поскольку $F_{obs} > F_{crit} = 3.39$ основная гипотеза отвергается.
\end{sol}
\end{problem}


\begin{problem}
По $35$ наблюдениям при помощи метода наименьших квадратов оценена
модель $\hy=\hb_1+\hb_2 x +\hb_3 z$, для которой $R^2 = 0.7$. При помощи вспомогательной регрессии $\hat{\hy}=\hat{\gamma}_1+\hat{\gamma}_2 x +\hat{\gamma}_3 z+\hat{\gamma}_4 \hy^2+\hat{\gamma}_5 \hy^3$, для которой $R^2 = 0.8$, выполните тест Рамсея на уровне значимости 5\%.


\begin{sol}
Аналогично задаче 9.2. Проверяемая гипотеза:
\[
H_0:
\begin{cases}
\gamma_4 = 0 \\
\gamma_5 = 0
\end{cases}
\]
F-статистика, которая при верной $H_0$ имеет распределение $F_{2, 20}$:
\[
F_{obs} = \frac{(11.5-9.5)/2}{9.5/20} = 2.1
\]
Так как $F_{obs} < F_{crit} = 3.49$, нет оснований отвергать $H_0$.
\end{sol}
\end{problem}


\begin{problem}
Используя 80 наблюдений, исследователь оценил две конкурирующие модели: $\hy=\hb_1+\hb_2 x +\hb_3 z$, в которой $RSS_1=36875$ и $\widehat{\ln y}=\hb_1+\hb_2 x +\hb_3 z$, в которой $RSS_2=122$.

Выполнив преобразование $y^*_i=y_i/\sqrt[n]{\prod y_i}$, исследователь также оценил две вспомогательные регрессии: $\hy^*=\hb_1+\hb_2 x +\hb_3 z$, в которой $RSS^*_1=239$ и $\widehat{\ln y^*}=\hb_1+\hb_2 x +\hb_3 z$, в которой $RSS^*_2=121$.

Завершите тест Бокса-Кокса на уровне значимости 5\%.


\begin{sol}
$H_0:$ модели (1) и (2) имеют одинаковое качество,
$H_1:$ модели (1) и (2) имеют разное качество, то есть одна из моделей лучше,
\begin{enumerate}
\item Тестовая статистика: \[T=\frac{n}{2}\left| \ln \frac{RSS_{2}^{*}}{RSS_{1}^{*}} \right|\].
\item Распределение тестовой статистики: \[T\underset{H_0, asy}{\sim} \chi^2_1 \].
\item Наблюдаемое значение тестовой статистики \[{{T}_{obs}}=\frac{80}{2}\left| \ln \frac{121}{239} \right|\approx \text{27.23}\].
\item Область, в которой $H_0$ не отвергается: \[[0;\ {{T}_{cr}}]=[0;\ qchisq(0.95,\ df=1)]=[0;\ \text{3.84}]\].
\item Статистический вывод: поскольку \[{{T}_{obs}}\notin [0;\ {{T}_{cr}}]\], гипотеза $H_0$ отвергается в пользу гипотезы $H_1$. Стало быть, из того, что \[RSS_{1}^{*}>RSS_{2}^{*}\] следует, что модель (1*) хуже модели (2*), а значит, модель (1) хуже модели (2). Таким образом, тест Бокса–Кокса говорит о том, что предпочтительнее модель с логарифмом — модель (2).
\end{enumerate}

\end{sol}
\end{problem}


\begin{problem}
Используя 40 наблюдений, исследователь оценил две конкурирующие модели: $\hy=\hb_1+\hb_2 x +\hb_3 z$, в которой $RSS_1=250$ и $\widehat{\ln y}=\hb_1+\hb_2 x +\hb_3 z$, в которой $RSS_2=12$.

Выполнив преобразование $y^*_i=y_i/\sqrt[n]{\prod y_i}$, исследователь также оценил две вспомогательные регрессии: $\hy^*=\hb_1+\hb_2 x +\hb_3 z$, в которой $RSS^*_1=20$ и $\widehat{\ln y^*}=\hb_1+\hb_2 x +\hb_3 z$, в которой $RSS^*_2=25$.

Завершите тест Бокса-Кокса на уровне значимости 5\%.


\begin{sol}
  $H_0:$ модели (1) и (2) имеют одинаковое качество,
  $H_1:$ модели (1) и (2) имеют разное качество, то есть одна из моделей лучше,
\begin{enumerate}
\item Тестовая статистика: \[T=\frac{n}{2}\left| \ln \frac{RSS_{2}^{*}}{RSS_{1}^{*}} \right|\].
\item Распределение тестовой статистики: \[T\underset{H_0, asy}{\sim} \chi^2_1 \].
\item Наблюдаемое значение тестовой статистики \[{{T}_{obs}}=\frac{40}{2}\left| \ln \frac{25}{20} \right|\approx \text{4.46}\].
\item Область, в которой $H_0$ не отвергается: \[[0;\ {{T}_{cr}}]=[0;\ qchisq(0.95,\ df=1)]=[0;\ \text{3.84}]\].
\item Статистический вывод: поскольку \[{{T}_{obs}}\notin [0;\ {{T}_{cr}}]\], гипотеза $H_0$ отвергается в пользу гипотезы $H_1$. Стало быть, из того, что \[RSS_{1}^{*}<RSS_{2}^{*}\] следует, что модель (1*) лучше модели (2*), а значит, модель (1) лучше модели (2). Таким образом, тест Бокса–Кокса говорит о том, что предпочтительнее модель без логарифма — модель (1).
\end{enumerate}

\end{sol}
\end{problem}


\begin{problem}
Почему при реализации теста Бокса-Кокса на компьютере предпочтительнее использовать формулу $y^*_i=\exp(\ln y_i - \sum \ln y_i /n) $, а не формулу $y^*_i=y_i/\sqrt[n]{\prod y_i}$?


\begin{sol}
Чтобы избежать переполнения при подсчёте произведения всех $y_i$
\end{sol}
\end{problem}



\begin{problem}
Обследовав выборку из 27 домохозяйств, исследователь оценил уравнение регрессии:
\[
\frac{\widehat{Exp}_i}{Size_i}=926+235\frac{1}{Size_i}+0.3\frac{Income_i}{Size_i}
\]
где $Exp_i$ — месячные затраты $i$-го домохозяйства на питание в рублях, $Income_i$ — месячный доход домохозяйства (также в рублях),  $Size_i$ — число членов домохозяйства. Известен коэффициент детерминации, $R^2=0.3$.

\begin{enumerate}
\item Каково, согласно оценённой модели, ожидаемое различие в затратах на питание между двумя домохозяйствами с одинаковым доходом, первое из которых больше второго на одного человека?
\item Известно, что нормировка переменных модели на размер семьи $Size_i$ была проведена с целью устранения гетероскедастичности в модели $Exp_i=\beta_1+\beta_2 Size_i+\beta_3 Income_i+\varepsilon_i$. Какое предположение сделал исследователь о виде гетероскедастичности?
\item Для проверки правильности выбранной спецификации было оценено ещё одно уравнение:
\[
\frac{\widehat{\widehat{Exp}}_i}{Size_i}=513+1499\frac{1}{Size_i}+0.5\frac{Income_i}{Size_i}-0.001\left(\frac{\widehat{Exp}_i}{Size_i}\right)^2
\]
Известно, что $R^2=0.4$. Даёт ли эта проверка основание считать модель исследователя неверно специфицированной? Используйте уровень значимости 1\%
\end{enumerate}


\begin{sol}
\end{sol}
\end{problem}







\begin{problem}
Мартовский Заяц и Безумный Шляпник почти всё время пьют чай. Известно, что количество выпитого за день чая (в чашках) зависит от количества пирожных (в штуках) и печенья (в штуках).
Алиса, гостившая у героев в течение 25 дней, заметила, что если оценить зависимость выпитого чая от закуски для Мартовского Зайца и Шляпника, то получится регрессия с $RSS=11.5$:
\[
\widehat{Tea}_i=6+0.5Biscuit_i+1.5Cake_i
\]
Чтобы понять, удачную ли модель она построила,  Алиса оценила ещё одну регрессию с $RSS=9.5$:
\[
\widehat{\widehat{Tea}}_i=12.7+0.65Biscuit_i-0.8Cake_i-0.59\widehat{Tea}^2_i+0.03\widehat{Tea}^3_i
\]
Помогите Алисе понять, верную ли спецификацию модели она выбрала:
\begin{enumerate}
\item Проведите подходящий тест.
\item Сформулируйте основную и альтернативную гипотезы.
\item Алиса решила проверить первоначальную короткую модель на наличие гетероскедастичности с помощью теста Уайта. Выпишите уравнение регрессии, которое она должна оценить.
\end{enumerate}


\begin{sol}
\end{sol}
\end{problem}




\Closesolutionfile{solution_file}

% !TEX root = ../em1_pset_v2.tex
\Opensolutionfile{solution_file}[solutions/sols_110]
% в квадратных скобках фактическое имя файла

\chapter{Обобщённый метод моментов и бутстрэп}




\begin{problem}
Величины $X_i$ равномерны на отрезке $[-a; 3a]$ и независимы. Есть несколько наблюдений, $X_1=0.5$, $X_2=0.7$, $X_3=-0.1$.

\begin{enumerate}
\item Найдите $\E(X_i)$ и $\E(|X_i|)$.
\item Постройте оценку метода моментов, используя $\E(X_i)$.
\item Постройте оценку метода моментов, используя $\E(|X_i|)$.
\item Постройте оценку обобщёного метода моментов используя моменты $\E(X_i)$, $\E(|X_i|)$ и взвешивающую матрицу.
\[
W=\begin{pmatrix}
2 & 0 \\
0 & 1 \\
\end{pmatrix}
\]
\item Найдите оптимальную теоретическую взвешивающую матрицу для обобщённого метода моментов
\item Постройте двухшаговую оценку обобщённого метода моментов, начав со взвешивающей матрицы $W$
\item С помощью полученных оценок постройте 95\%-ый доверительный интервал для неизвестного параметра $a$

\end{enumerate}

\begin{sol}
\end{sol}
\end{problem}


\begin{problem}
Величины $X_i$ имеют Пуассоновское распределение с параметром $\lambda$ и независимы. Есть несколько наблюдений, $X_1=5$, $X_2=7$, $X_3=1$.

\begin{enumerate}
\item Найдите $\E(X_i)$ и $\E((X_i-\bar X)^2)$.
\item Постройте оценку метода моментов, используя $\E(X_i)$.
\item Постройте оценку метода моментов, используя $\E((X_i-\bar X)^2)$.
\item Постройте оценку обобщёного метода моментов используя моменты $\E(X_i)$, $\E((X_i-\bar X)^2)$ и взвешивающую матрицу.
\[
W=\begin{pmatrix}
3 & 0 \\
0 & 1 \\
\end{pmatrix}
\]
\item Найдите оптимальную теоретическую взвешивающую матрицу для обобщённого метода моментов.
\item Постройте двухшаговую оценку обобщённого метода моментов, начав со взвешивающей матрицы $W$.

%\todo[inline]{Корректно ли в ММ брать $\bar X$? Да, но может выйти адская формула с суммами}

\end{enumerate}

\begin{sol}
\end{sol}
\end{problem}


\begin{problem}
Винни-Пух и Пятачок оценивают неизвестный параметр правильности пчёл $\theta$. Когда Винни-Пух проводит очередное измерение параметра правильности, он получает значение $X_i$ нормально распределенное вокруг неизвестного параметра, $X_i \sim \cN(\theta, 1)$. Когда Пятачок проводит измерение параметра правильности, он получает значение $Y_i$, также нормально распределенное вокруг $\theta$, но имеющее большую дисперсию, $Y_i \sim \cN(\theta, 4)$. Различные измерения независимы между собой.
\begin{enumerate}
\item Найдите $\E(X_i)$ и постройте соответствующую оценку метода моментов.
\item Найдите $\E(Y_i)$ и постройте соответствующую оценку метода моментов.
\item Используя два указанных момента найдите обобщённую оценку метода моментов для взвешивающей матрицы
\[
W = \begin{pmatrix}
4 & 0 \\
0 & 9
\end{pmatrix}.
\]
\item Найдите оптимальную взвешивающую матрицу $W$.
\end{enumerate}
\begin{sol}
\end{sol}
\end{problem}


\begin{problem}
Начинающий футболист делает независимые удары по воротам. С вероятностью $\theta$ он попадает левее ворот, с вероятностью $2\theta$ — правее ворот и попадает с вероятностью $1-3\theta$. Из $n$ ударов он попал $N_L$ раз левее ворот и $N_R$ раз — правее.

\begin{enumerate}
\item Найдите $\E(N_L)$ и постройте соответствующую оценку $\theta$ методом моментов.
\item Найдите $\E(N_R)$ и постройте соответствующую оценку $\theta$ методом моментов.
\item Используя два указанных момента постройте оценку обобщённого метода моментов со  взвешивающей матрицей
\[
W = \begin{pmatrix}
4 & 0 \\
0 & 9
\end{pmatrix}.
\]
\item Найдите оптимальную теоретическую взвешивающую матрицу.
\item Для каждой из найденных оценок постройте 95\%-ый доверительный интервал, если $N_L=10$, $N_R=30$, $n=200$.
\end{enumerate}
\begin{sol}
\end{sol}
\end{problem}



\begin{problem}
Можно ли получить МНК-оценки в классической задаче регрессии как оценки обобщённого метода моментов? Можно ли получить оценки метода максимального правдоподобия как оценки обобщённого метода моментов?
\begin{sol}
да, да
\end{sol}
\end{problem}

\begin{problem}
Для распределения

\begin{tabular}{c|ccc}
\toprule
$Z$ & $1$ & $3$ & $7$ \\
$\P(\cdot)$ & $0.1$ & $0.3$ & $0.6$ \\
\bottomrule
\end{tabular}

найдите квантили уровней $0.05$, $0.1$, $0.3$, $0.4$, $0.95$.

\begin{sol}
$q_{0.05} = 1$, $q_{0.1} = 3$, $q_{0.3} = 3$, $q_{0.4} = 7$, $q_{0.95}=7$.
\end{sol}
\end{problem}


\begin{problem}
Рассмотрим модель $y_i = \beta x_i + \e_i$, где $\E(\e_i)=0$, $\Var(\e_i)=\sigma^2$ и $\Cov(x_i, y_i)=0$ при $i \neq j$.

Имеется всего два наблюдения:

\begin{tabular}{cc}
\toprule
$y$ & $x$ \\
\midrule
 3 & 1 \\
 4 & 2 \\
\bottomrule
\end{tabular}

\begin{enumerate}
\item Постройте таблицу распределения бутстрэповской оценки $\beta_*$.
\item Постройте график функции распределения бутстрэповской оценки $\beta_*$.
\item Для бутстрэповской оценки найдите квантили уровней $0.025$, $0.1$, $0.3$, $0.7$, $0.8$, $0.95$.
\item Постройте 95\%-ый бутстрэповский интервал для параметра $\beta$.
\item При помощи построенного бутстрэповского интервала проверьте гипотезу $H_0$: $\beta = 0$ на уровне значимости $\alpha = 0.05$.
\end{enumerate}

\begin{sol}
Случай 1:

\begin{tabular}{cc}
\toprule
$y$ & $x$ \\
\midrule
 3 & 1 \\
 3 & 1 \\
\bottomrule
\end{tabular}

Здесь $\beta_* = 3$.

Случай 2:

\begin{tabular}{cc}
\toprule
$y$ & $x$ \\
\midrule
 3 & 1 \\
 4 & 2 \\
\bottomrule
\end{tabular}

Здесь $\beta_* = 11/5$.

Случай 3:

\begin{tabular}{cc}
\toprule
$y$ & $x$ \\
\midrule
 4 & 2 \\
 3 & 1 \\
\bottomrule
\end{tabular}

Здесь $\beta_* = 11/5$.

Случай 4:

\begin{tabular}{cc}
\toprule
$y$ & $x$ \\
\midrule
 4 & 2 \\
 4 & 2 \\
\bottomrule
\end{tabular}

Здесь $\beta_* = 2$.

Таблица распределения:

\begin{tabular}{c|ccc}
\toprule
$\beta_*$ & $2$ & $11/5$ & $3$ \\
$\P(\cdot)$ & $1/4$ & $2/4$ & $1/4$ \\
\bottomrule
\end{tabular}

По определению квантили, $q_a = \min \{x | F(x) > a \}$. Получаем: $q_{0.025} = 2$, $q_{0.1} = 2$, $q_{0.3} = 2$, $q_{0.7} = 11/5$, $q_{0.8} = 3$, $q_{0.975} = 3$.


Получаем 95\%-ый бутстрэповский интервал $[q_{0.025}; q_{0.975}]=[2;3]$.


Замечаем, что $0 \notin [2;3]$, поэтому гипотеза $H_0$: $\beta = 0$ отвергается.

\end{sol}
\end{problem}




\begin{problem}
Рассматривается модель линейной регрессии $y_i = \beta x_i + \e_i$ . Имеются следующие наблюдения $x_1 =3, x_2 =4, y_1 =2, y_2 =1$.
\begin{enumerate}
\item Постройте таблицу распределения бутстрэповской оценки $\hb_*$.
\item Найдите математическое ожидание бутстрэповской оценки $\hb_*$.
\item Постройте функцию распределения бутстрэповской оценки $\hb_*$.
\item Для найденной в предыдущем пункте функции распределения $F_{\hb_*}(x)$ найдите квантили уровней: $0.025, 0.1, 0.3, 0.8, 0.975$.
\item Для неизвестного параметра $\beta$ постройте $95\%$-ый бутстрэповский доверительный интервал.
\item При помощи доверительного интервала, полученного в предыдущем пункте, протестируйте гипотезу о значимости коэффициента $\beta$ на уровне значимости $5\%$.
\end{enumerate}


\begin{sol}
\begin{enumerate}
\item

\begin{tabular}{llll}
\toprule
$\hb_*$ & 1/4 & 2/5 & 2/3 \\
$\P_{\hb_*}$ & 1/4   & 2/5   & 1/4 \\
\bottomrule
\end{tabular}

\item $\E [\hb_*] = 103/240$
\item
\[
F_{\hb_*}(x) =
\left\{\begin{matrix}
0 & , x \leq 1/4 \\
1/4 & ,1/4 < x \leq 2/5 \\
3/4 & , 2/5 < x \leq 2/3\\
1 & 2/3 < x
\end{matrix}\right.
\]
\item $Q_{\hb_*} (0.025) = 1/4$, $Q_{\hb_*} (0.1) = 1/4$, $Q_{\hb_*} (0.3) = 2/5$, $Q_{\hb_*} (0.8) = 2/3$, $Q_{\hb_*} (0.975) = 2/3$
\item $\left[ Q_{\hb_*} (0.025);  Q_{\hb_*} (0.975) \right] \approx \left[1/4; 2/3 \right] $
\item $0 \notin  \left[ Q_{\hb_*} (0.025);  Q_{\hb_*} (0.975) \right]$, следовательно, коэффициент $\beta$ значим
\end{enumerate}
\end{sol}
\end{problem}


\begin{problem}
Рассматривается модель линейной регрессии $y_i = \beta x_i + \e_i$. В следующей таблице приведены наблюдения

\begin{tabular}{ll}
\toprule
x & y \\
\midrule
1 & 2 \\
3 & 4 \\
5 & 6 \\
7 & 8 \\
\bottomrule
\end{tabular}

C помощью компьютера выполните следующие задания:

\begin{enumerate}
\item Найдите приближенно математическое ожидание бутстрэповской оценки $\hb_*$.
\item Для неизвестного параметра $\beta$ постройте $95\%$-ый бутстрэповский доверительный интервал.
\item При помощи доверительного интервала, полученного в предыдущем пункте, протестируйте гипотезу о значимости коэффициента $\beta$ на уровне значимости $5\%$.
\end{enumerate}


\begin{sol}

\begin{minted}[mathescape,
               linenos,
               numbersep=5pt,
               frame=lines,
               framesep=2mm]{r}
X <- c(1, 3, 5, 7)
Y <- c(2, 4, 6, 8)

Z <- cbind(Y, X)

SL <- 0.05

n <- length(X)
k <- 1


S <- 10^4
b_BOOT <- matrix(0, nrow = k, ncol = S)


set.seed(777) # На удачу!!!
for (s in 1:S) {
  Z_boot <- Z[sample(n, replace = TRUE), ]
  Y_boot <- Z_boot[, 1]
  X_boot <- Z_boot[, -1]
  model <- lm(Y_boot ~ 0 + X_boot)
  b_BOOT[, s] <- coef(model)
}

mean_b_boot <- mean(b_BOOT)

CI_b_boot <- matrix(0, nrow = k, ncol = 2)

for (j in 1:k) {
  CI_b_boot[j, 1] <- quantile(b_BOOT[j, ], SL/2)
  CI_b_boot[j, 2] <- quantile(b_BOOT[j, ], 1 - SL/2)
}
\end{minted}

\begin{enumerate}
\item $\E [\hb_*] \approx 1.22 $
\item $\left[ Q_{\hb_*} (0.025);  Q_{\hb_*} (0.975) \right] \approx \left[1.15; 1.40 \right] $
\item $0 \notin  \left[ Q_{\hb_*} (0.025);  Q_{\hb_*} (0.975) \right]$, следовательно, коэффициент $\beta$ значим
\end{enumerate}
\end{sol}
\end{problem}


\begin{problem}
Рассматривается модель линейной регрессии $y_i = \beta x_i + \e_i$. В следующей таблице приведены наблюдения

\begin{tabular}{ll}
\toprule
x & y \\
\midrule
1 & 2 \\
3 & 4 \\
5 & 6 \\
7 & 8 \\
9 & 10\\
\bottomrule
\end{tabular}
C помощью компьютера выполните следующие задания:

\begin{enumerate}
\item Найдите приближенно математическое ожидание бутстрэповской оценки $\hb_*$.
\item Для неизвестного параметра $\beta$ постройте $95\%$-ый бутстрэповский доверительный интервал.
\item При помощи доверительного интервала, полученного в предыдущем пункте, протестируйте гипотезу о значимости коэффициента $\beta$ на уровне значимости $5\%$.
\end{enumerate}


\begin{sol}
\begin{enumerate}
\item $\E [\hb_*] \approx 1.165 $
\item $\left[ Q_{\hb_*} (0.025);  Q_{\hb_*} (0.975) \right] \approx \left[1.12; 1.29 \right] $
\item $0 \notin  \left[ Q_{\hb_*} (0.025);  Q_{\hb_*} (0.975) \right]$, следовательно, коэффициент $\beta$ значим
\end{enumerate}
\end{sol}
\end{problem}

\begin{problem}
  Допустим, что в GMM количество оцениваемых параметров равно количеству моментных условий.
  \begin{enumerate}
    \item Чему равно минимальное значение целевой функции GMM?
    \item Совпадают ли оценки GMM и оценки обычного метода моментов?
  \end{enumerate}

\begin{sol}
Минимум равен $0$. Совпадают, так как моментные условия выполнены как точные равенства.
\end{sol}
\end{problem}


\Closesolutionfile{solution_file}

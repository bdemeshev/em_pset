\Opensolutionfile{solution_file}[solutions/sols_200]
% в квадратных скобках фактическое имя файла

\chapter{Линейная алгебра}

\begin{problem}
Возведите каждую из следующих матриц в каждую из степеней $\frac{1}{2}$, $\frac{1}{3}$, $-\frac{1}{2}$, $-\frac{1}{3}$, $-1$, $100$.
\begin{enumerate}
\item $\begin{pmatrix}{}
    1 &   1 \\ 
    1 &   2 \\ 
  \end{pmatrix}$;
\item $\begin{pmatrix}{}
    4 &   1 \\ 
    1 &   2 \\ 
  \end{pmatrix}$.
\end{enumerate}


\begin{sol}
\begin{minted}[mathescape, numbersep=5pt, frame=lines, framesep=2mm]{r}
ma1 <- matrix(as.integer(c(1, 1, 1, 2)), nrow = 2, ncol = 2,
  byrow = FALSE, dimnames = NULL)
m1 <- matrix(as.integer(c(4, 1, 1, 2)), nrow = 2, ncol = 2,
  byrow = FALSE, dimnames = NULL)
\end{minted}

\end{sol}
\end{problem}



\begin{problem}
Найдите ортогональную проекцию и ортогональную составляющую (перпендикуляр) вектора $u_1$ на линейное подпространство $L = \mathcal{L}(u_2)$, порождённое вектором $u_2$, если
\begin{enumerate}
\item $\begin{matrix} u_1 = (1 & 1 & 1 & 1), u_2 = (1 & 0 & 0 & 1) \end{matrix}$;
\item $\begin{matrix} u_1 = (2 & 2 & 2 & 2), u_2 = (1 & 0 & 0 & 1) \end{matrix}$;
\item $\begin{matrix} u_1 = (1 & 1 & 1 & 1), u_2 = (7 & 0 & 0 & 7) \end{matrix}$.
\end{enumerate}


\begin{sol}
\end{sol}
\end{problem}



\begin{problem}
Найдите обратные матрицы ко всем матрицам, представленным ниже.
\begin{enumerate}
\item $\begin{pmatrix}{}
    1 &   1 &   0 \\ 
    0 &   1 &   1 \\ 
    0 &   0 &   1 \\ 
  \end{pmatrix}$;
\item $\begin{pmatrix}{}
    1 &   0 &   0 \\ 
    1 &   1 &   0 \\ 
    0 &   1 &   1 \\ 
  \end{pmatrix}$;
\item $\begin{pmatrix}{}
    0 &   0 &   1 \\ 
    1 &   0 &   0 \\ 
    0 &   1 &   0 \\ 
  \end{pmatrix}$;
\item $\begin{pmatrix}{}
  0 & 0 & a \\ 
  1 & 0 & 0 \\ 
  0 & 1 & 0 \\ 
  \end{pmatrix}$.
\end{enumerate}


\begin{sol}

\begin{minted}[mathescape, numbersep=5pt, frame=lines, framesep=2mm]{r}
A <- matrix(as.integer(c(1, 0, 0, 1, 1, 0, 0, 1, 1)), nrow = 3, ncol = 3,
  byrow = FALSE, dimnames = NULL)
B <- matrix(as.integer(c(1, 1, 0, 0, 1, 1, 0, 0, 1)), nrow = 3, ncol = 3,
  byrow = FALSE, dimnames = NULL)
C <- matrix(as.integer(c(0, 1, 0, 0, 0, 1, 1, 0, 0)), nrow = 3, ncol = 3,
  byrow = FALSE, dimnames = NULL)
D <- matrix(c(0, 1, 0, 0, 0, 1, "a", 0, 0), nrow = 3, ncol = 3,
  byrow = FALSE, dimnames = NULL)
\end{minted}


\end{sol}
\end{problem}



\begin{problem}
Найдите ранг следующих матриц в зависимости от значений параметра $\lambda$:

\begin{enumerate}
\item $\begin{pmatrix} \lambda & 1 & 1 \\ 1 & \lambda & 1 \\ 1 & 1 & \lambda \end{pmatrix}$;
\item $\begin{pmatrix} 1-\lambda & 1-2\lambda \\ 1+\lambda & 1+3\lambda \end{pmatrix}$;
\item $\begin{pmatrix} 1 & \lambda & -1 & 2 \\ 2 & -1 & \lambda & 5 \\ 1 & 10 & -6 & 1 \end{pmatrix}$;
\item $\begin{pmatrix} \lambda & 1 & -1 & -1 \\ 1 & \lambda & -1 & -1 \\ 1 & 1 & -\lambda & -1
\\ 1 & 1 & -1 & -\lambda \end{pmatrix}$.
\end{enumerate}



\begin{sol}
\end{sol}
\end{problem}




\begin{problem}
Пусть $\v1 = (1,\dots,1)'$ — вектор из $n$ единиц и $\pi=\v1(\v1'\v1)^{-1}\v1'$. Найдите:
\begin{enumerate}
\item $\tr(\pi)$ и $\rk(\pi)$;
\item $\tr(I-\pi)$ и $\rk(I-\pi)$.
\end{enumerate}


\begin{sol}
\end{sol}
\end{problem}


\begin{problem}
Пусть $X$ — матрица размера ${n \times k}$, где $n > k$, и пусть $\rk(X) = k$. Верно ли, что матрица-шляпница $H = X(X'X)^{-1}X'$ симметрична и идемпотентна?


\begin{sol}
Да, $H'=H$ и $H^2=H$.
\end{sol}
\end{problem}



\begin{problem}
Пусть $X$ — матрица размера ${n \times k}$, где $n > k$, и пусть $\rk(X) = k$. Верно ли, что каждый столбец матрицы-шляпницы $H = X(X'X)^{-1}X'$ является собственным вектором матрицы $H$? Если да, то какое собственное число ему соответствует?


\begin{sol}
Поскольку $H\cdot H = H$, то каждый столбец матрицы — это собственный вектор с собственным числом 1.
\end{sol}
\end{problem}



\begin{problem}
Пусть $X$ — матрица размера ${n \times k}$, где $n > k$, пусть $\rk(X) = k$ и $H = X(X'X)^{-1}X'$. Верно ли, что каждый вектор-столбец $u$, такой что $X'u=0$, является собственным вектором матрицы-шляпницы $H$? Если да, то какое собственное число ему соответствует?


\begin{sol}
Да, $Hu=X(X'X)^{-1}X'u=0$.
\end{sol}
\end{problem}



\begin{problem}
Верно ли, что для любых матриц $A$ размера $m\times n$ и матриц $B$ размера
${n \times m}$ выполняется равенство $\tr(AB) = \tr(BA)$?


\begin{sol}
Да.
\end{sol}
\end{problem}


\begin{problem}
Какие собственные значения могут быть у симметричной и идемпотентной матрицы?
\begin{sol}
Только 0 или 1.
\end{sol}
\end{problem}


\begin{problem}
Пусть $H$ — матрица размера ${n \times n}$, $H'= H$, $H^2 = H$. Верно ли, что $\rk(H) = \tr(H)$?

\begin{sol}
Да.
\end{sol}
\end{problem}



\begin{problem}
Верно ли, что для симметричной матрицы собственные векторы, отвечающие различным собственным значениям, ортогональны?
\begin{sol}
Да.
\end{sol}
\end{problem}



\begin{problem}
Найдите собственные значения и собственные векторы матрицы $H = X(X'X)^{-1}X'$, если

\begin{enumerate}
\item $X = \begin{pmatrix} 1 \\ 2 \\ 3 \\ 4 \end{pmatrix}$;
\item $X = \begin{pmatrix} 1 & 1 \\ 1 & 2 \\ 1 & 3 \\ 1 & 4 \end{pmatrix}$;
\item $X = \begin{pmatrix} 1 & 0 & 0 \\ 1 & 0 & 0 \\ 1 & 1 & 0  \\ 1 & 1 & 1 \end{pmatrix}$;
\item $X = \begin{pmatrix} 1 & 0 & 0 & 0 \\ 1 & 1 & 0 & 0 \\ 1 & 1 & 1 & 0  \\ 1 & 1 & 1 & 1 \end{pmatrix}$.
\end{enumerate}


\begin{sol}

\begin{minted}[mathescape, numbersep=5pt, frame=lines, framesep=2mm]{r}
a <- matrix(as.integer(c(1, 2, 3, 4)), nrow = 4, ncol = 1,
    byrow = FALSE, dimnames = NULL)
b <- matrix(as.integer(c(1, 1, 1, 1, 1, 2, 3, 4)), nrow = 4, 
    ncol = 2, byrow = FALSE, dimnames = NULL)
c <- matrix(as.integer(c(1, 1, 1, 1, 0, 0, 1, 1, 0, 0, 0, 1)), nrow = 4, ncol = 3, 
    byrow = FALSE, dimnames = NULL)
d <- matrix(as.integer(c(1, 1, 1, 1, 0, 1, 1, 1, 0, 0, 1, 1, 0, 0, 0, 1)), nrow = 4, 
    ncol=4, byrow = FALSE, dimnames = NULL)
\end{minted}


Матрица $H$ — это матрица-шляпница, проектор. Собственные числа у неё — 0 и 1. Единице соответствуют столбцы матрицы $X$ и их линейные комбинации. Нулю соответствуют вектора, ортогональные одновременно всем столбцам матрицы $X$.
\end{sol}
\end{problem}


\begin{problem}
Приведите пример таких $A$ и $B$, что $\det(AB)\neq \det(BA)$.
\begin{sol}
Например, $A=(1,2,3)$, $B=(1,0, 1)'$.
\end{sol}
\end{problem}


\begin{problem}
Для матриц-проекторов $\pi=\v1(\v1'\v1)^{-1}\v1'$ и $H=X(X'X)^{-1}X'$ найдите $\tr(\pi)$, $\tr(H)$, $\tr(I-\pi)$, $\tr(I-H)$.


\begin{sol}
$\tr(I)=n$, $\tr(\pi)=1$, $\tr(H)=k$
\end{sol}
\end{problem}



\begin{problem}
Выпишите в явном виде матрицы $X'X$, $(X'X)^{-1}$ и $X'y$, если

$y=\left(
\begin{array}{c}
y_1 \\
y_2 \\
\vdots \\
y_n
\end{array}\right)$ и
$X=\left(
\begin{array}{cc}
1 & x_1 \\
1 & x_2 \\
\vdots & \vdots \\
1 & x_n
\end{array}\right)$.


\begin{sol}
\end{sol}
\end{problem}



\begin{problem}
Выпишите в явном виде матрицы $\pi$, $\pi y$, $\pi \e$, $I-\pi$, если $\pi=\v1(\v1'\v1)^{-1}\v1'$.


\begin{sol}
\end{sol}
\end{problem}



\begin{problem}
Формула Фробениуса. Матрицу $A$ размера $(n+m)\times (n+m)$ разрезали на 4 части: $A=\begin{pmatrix}
A_{11} & A_{12} \\
A_{21} & A_{22}
\end{pmatrix}$. Кусок $A_{11}$ имеет размер $n\times n$ и обратим, кусок $A_{22}$ имеет размер $m\times m$. Известно, что $A$ — обратима и $A^{-1}=B$. На аналогичные по размеру и расположению части разрезали матрицу $B=\begin{pmatrix}
B_{11} & B_{12} \\
B_{21} & B_{22}
\end{pmatrix}$.
\begin{enumerate}
\item Каковы размеры кусков $A_{12}$ и $A_{21}$?
\item Чему равно $B_{22}(A_{22}-A_{21}A_{11}^{-1}A_{12})$?
\end{enumerate}


\begin{sol}
$n\times m$, $m\times n$, $I$
\end{sol}
\end{problem}




\begin{problem}
Спектральное разложение. Симметричная матрица $A$ размера $n\times n$ имеет $n$ собственных чисел $\lambda_1$, \ldots, $\lambda_n$ с собственными векторами $u_1$, \ldots, $u_n$. Докажите, что $A$ можно представить в виде $A=\sum \lambda_i u_i u_i'$.


\begin{sol}
\end{sol}
\end{problem}



\begin{problem}
Найдите определитель, след, собственные значения, собственные векторы и число
обусловленности матрицы $A$. Также найдите $A^{-1}$, $A^{-1/2}$ и $A^{1/2}$.
\begin{enumerate}
\item $A=\begin{pmatrix}
0.2 & 0 \\
0 & 0.1
\end{pmatrix}$;
\item $A=\begin{pmatrix}
2 & 1 \\
1 & 2
\end{pmatrix}$;

\item $A=\begin{pmatrix}
4 & 1 \\
1 & 4
\end{pmatrix}$;

\item $A=\begin{pmatrix}
2 & 1 & 1 \\
1 & 2 & 1 \\
1 & 1 & 2
\end{pmatrix}$;

\item $A=\begin{pmatrix}
3 & 2 & 1 \\
2 & 3 & 2 \\
1 & 2 & 3
\end{pmatrix}$;

\item $A=\begin{pmatrix}
0 & 0 & 1 \\
0 & 1 & 0 \\
1 & 0 & 0
\end{pmatrix}$;

\item $A=\begin{pmatrix}
4 & -1 & 1 \\
-1 & 4 & -1 \\
1 & -1 & 4
\end{pmatrix}$;

\item $A=\begin{pmatrix}
6 & -2 & 2 \\
-2 & 5 & 0 \\
2 & 0 & 7
\end{pmatrix}$.
\end{enumerate}


\begin{sol}
\end{sol}
\end{problem}



\begin{problem}
В этом упражнении исследуется связь определителя, следа и собственных значений. Везде имеются ввиду действительные собственные значения с учетом кратности.
\begin{enumerate}
\item Приведите пример матрицы для которой след равен сумме собственных значений.
\item Приведите пример матрицы для которой след не равен сумме собственных значений.
\item Верно ли, что для симметричной матрицы след всегда равен сумме собственных значений?
\item Приведите пример матрицы для которой определитель равен произведению собственных значений.
\item Приведите пример матрицы для которой определитель не равен произведению собственных значений.
\item Верно ли, что для симметричной матрицы определитель всегда равен произведению собственных значений?
\end{enumerate}


\begin{sol}
\end{sol}
\end{problem}



\begin{problem}
Найдите собственные числа и собственные векторы матрицы строевого леса размера $n\times n$. Матрица строевого леса — это матрица, состоящая только из единиц.

\begin{sol}
Собственные векторы — все векторы с нулевой суммой компонент, $\lambda=0$, векторы из одинаковых чисел, $\lambda=n$.
\end{sol}
\end{problem}


\begin{problem}
Найдите собственные числа и собственные векторы матрицы $A=vv'$, где $v$ — вектор-столбец размера $n\times 1$.

\begin{sol}
Собственные векторы — все векторы перпендикулярные $v$, $\lambda=0$, векторы пропорциональные $v$, $\lambda=|v|^2$.
\end{sol}
\end{problem}


\begin{problem}
Пусть $S$ — матрица строевого леса, т.е. матрица размера $n\times n$, состоящая только из единиц, а $y$ — вектор размера $n\times 1$. Найдите $Sy$, $S^2$, $S^{2015}$, $(I-\frac{1}{n}S)^2$, $(I-\frac{1}{n}S)^{2015}$.

\begin{sol}
\end{sol}
\end{problem}

\begin{problem}
Верно ли, что собственные числа матриц $X'X$ и $XX'$ совпадают?  Известно, что $v$ — это собственный вектор матрицы $X'X$. Найдите какой-нибудь собственный вектор матрицы $XX'$.
\begin{sol}
$X'Xv=\lambda v$, следовательно $XX'Xv=Xv$ и $w=Xv$ — собственный вектор матрицы $XX'$.
\end{sol}
\end{problem}

\begin{problem}
Рассмотрим систему уравнений $X\beta = y$. Здесь $y$ — известный вектор размера $n\times 1$, $\beta$ — неизвестный вектор размера $k\times 1$, и $X$ — известная матрица размера $n\times k$ полного ранга.

Цель задачи состоит в том, чтобы понять, как концепция единственного решения системы уравнений распространяется на случай, когда решений вообще нет или когда их бесконечно много.

\begin{enumerate}
\item При каких условиях система имеет одно решение? ни одного? бесконечно много?
\item Решите систему в случае квадратной $X$ полного ранга.
\item Если решений нет, то найдите наилучшее приближение к решению, то есть такое $\beta$ при котором длина $(y-X\beta)$ минимальна.
\item Если решений бесконечно много, то найдите решение с наименьшей длиной.
\end{enumerate}

\begin{sol}
Если матрица $X$ обратима, то $\beta = X^{-1}y$.

Если решений нет, то наилучшее приближение к решению — это $\beta = (X'X)^{-1}X'y$

Если решений бесконечно много, то решение с наименьшей длиной — это $\beta=X(XX')^{-1}y$.
\end{sol}
\end{problem}

\begin{problem}
  Известно, что матрица $M$ идемпотентная. В каком случае она может быть обратимой?
\begin{sol}
  Только если $M=I$. Доказательство. Если $M^2=M$, и $M$ — обратима, то, домножив обе части равенства на $M^{-1}$, получим $M=I$.
\end{sol}
\end{problem}



\Closesolutionfile{solution_file}

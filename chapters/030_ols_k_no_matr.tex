\Opensolutionfile{solution_file}[solutions/sols_030]
% в квадратных скобках фактическое имя файла

\chapter{Многомерный МНК без матриц}



\begin{problem} % 3.1
 Эконометрэсса Ширли зашла в пустую аудиторию, где царил приятный полумрак, и увидела на доске до боли знакомую надпись:
\[
\hy=\underset{(2.37)}{1.1}-\underset{(-0.4)}{0.7}\cdot x_2+\underset{(3.15)}{0.9}\cdot x_3-\underset{(-0.67)}{19}\cdot x_4
\]

Помогите эконометрэссе Ширли определить, что находится в скобках:
\begin{enumerate}
\item $P$-значения;
\item $t$-статистики;
\item стандартные ошибки коэффициентов;
\item $R^2$, скорректированный на номер коэффициента;
\item показатели $VIF$ для каждого коэффициента.
\end{enumerate}


\begin{sol}
$t$-статистики, только они бывают отрицательными из перечисленных вариантов.
\end{sol}
\end{problem}


\begin{problem}  % 3.2
 Для нормальной регрессии с 5-ю факторами (включая свободный член) известны границы симметричного по вероятности 80\% доверительного интервала для дисперсии $\sigma_{\e}^2$: $[45; 87.942]$.

\begin{enumerate}
\item Определите количество наблюдений в выборке.
\item Вычислите $\hs_{\e}^2$.
\end{enumerate}


\begin{sol}
\begin{enumerate}
\item Поскольку $\frac{\hs_{\e}^2(n-k)}{\sigma_{\e}^2} \sim \chi ^2(n-k)$, где $\hs_{\e}^2 = \frac{RSS}{n-k}$, $k$ = 5. $\P(\chi_{l}^2 < \frac{\hs_{\e}^2}{\sigma_{\e}^2} < \chi_{u}^2) = 0.8$. Преобразовав, получим $\P(\frac{\hs_{\e}^2(n-5)}{\chi_{u}^2} < \sigma_{\e}^2 < \frac{\hs_{\e}^2(n-5)}{\chi_{l}^2}) = 0.8$, где $\chi_{u}^2 = \chi_{n-5; 0.1} ^2$, $\chi_{l}^2 = \chi_{n-5; 0.9} ^2$ — соответствующие квантили. По условию $\frac{\hs_{\e}^2(n-5)}{\chi_{l}^2} = A = 45, \frac{\hs_{\e}^2(n-5)}{\chi_{u}^2} = B = 87.942.$ Поделим $B$ на $A$, отсюда следует $\frac{\chi_{u}^2}{\chi_{l}^2} = 1.95426.$ Перебором квантилей в таблице для хи-квадрат распределения мы находим, что $\frac{\chi_{30; 0.1}^2}{\chi_{30; 0.9}^2} = \frac{40.256}{20.599} = 1.95426.$ Значит, $n - 5 = 30$, отсюда следует, что $n = 35.$
\item $\hs_{\e}^2 = 45 \frac{\chi_{u}^2}{n-5} = 45 \frac{40.256}{30} = 60.384$
\end{enumerate}

Решение в R:
\begin{minted}[mathescape,
               numbersep=5pt,
               frame=lines,
               framesep=2mm]{r}
the_grid <- tibble(df = 1:200, left = qchisq(0.1, df),
                   right = qchisq(0.9, df), ratio = right / left,
                   penalty = (ratio - 87.942 / 45)^2)

df_ans <- the_grid$df[which.min(the_grid$penalty)]
\end{minted}

Количество степеней свободы $n-5$ должно быть равно $30$.

\end{sol}
\end{problem}


\begin{problem}  % 3.3
 Рассмотрим следующую регрессионную модель зависимости логарифма заработной платы индивида $\ln W$ от его уровня образования $Edu$, опыта работы $Exp$, $Exp^2$, уровня образования его отца $Fedu$, и уровня образования его матери $Medu$:
\[
\widehat{\ln W}=\hb_1 + \hb_2 Edu + \hb_3 Exp + \hb_4 Exp^2 + \hb_5 Fedu+\hb_6 Medu.
\]

Модель регрессии была отдельно оценена по выборкам из 35 мужчин и 23 женщин, и были получены остаточные суммы квадратов $RSS_1$ = 34.4 и $RSS_2$ = 23.4 соответственно. Остаточная сумма квадратов в регрессии, оценённой по объединённой выборке, равна 70.3. На уровне значимости $5\%$ проверьте гипотезу об отсутствии дискриминации в оплате труда между мужчинами и женщинами.


\begin{sol}

Упорядочим нашу выборку таким образом, чтобы наблюдения с номерами с 1 по 35 относились к мужчинам, а наблюдения с номерами с 36 по 58 относились к женщинам.
Тогда уравнение
\begin{multline*}
\ln W_i=\beta_1+\beta_2 Edu_i+\beta_3 Exp_i+\beta_4 Exp_i^2+\\
\beta_5 Fedu_i+\beta_6 Medu_i+\e_i, i=1, \ldots, 35
\end{multline*}

соответствует регрессии, построенной для подвыборки из мужчин, а уравнение
\begin{multline*}
\ln W_i=\gamma_1+\gamma_2 Edu_i+\gamma_3 Exp_i+\gamma_4 Exp_i^2+\\
\gamma_5 Fedu_i+\gamma_6 Medu_i+\e_i, i=36, \ldots, 58
\end{multline*}
соответствует регрессии, построенной для подвыборки из женщин. Введем следующие переменные:

\[
d_i =
\begin{cases}
    1, & \text{если $i$--ое наблюдение соответствует мужчине,} \\
    0, & \text{в противном случае;}
\end{cases}
\]

\[
dum_i =
\begin{cases}
    1, & \text{если $i$--ое наблюдение соответствует женщине,} \\
    0, & \text{в противном случае.}
\end{cases}
\]

Рассмотрим следующее уравнение регрессии:
\begin{multline*}
\ln W_i=\beta_1 d_i+\gamma_1 dum_i+\beta_2 Edu_i d_i+\gamma_2
Edu_i dum_i+\beta_3 Exp_i d_i+\\
\gamma_3 Exp_i dum_i+\beta_4 Exp_i^2 d_i + \gamma_4 Exp_i^2 dum_i + \beta_5 Fedu_i d_i +\gamma_5 Fedu_i dum_i+\\
\beta_6 Medu_i d_i+\gamma_6 Medu_i dum_i+\e_i, i=1, \ldots, 58
\end{multline*}
Гипотеза, которую требуется проверить в данной задаче, имеет вид

\[
H_0:
  \begin{cases}
    \beta_1 =\gamma_1, \\
    \beta_2 =\gamma_2 , & H_1:|\beta_1-\gamma_1|+|\beta_2-\gamma_2|+\dots+|\beta_6-\gamma_6| > 0.\\
    \dots   \\
    \beta_6=\gamma_6 \\
 \end{cases}
\]

Тогда регрессия
\begin{multline*}
\ln W_i=\beta_1 d_i+\gamma_1 dum_i+\beta_2 Edu_i d_i+\gamma_2 Edu_i dum_i+\beta_3 Exp_i d_i+\\
\gamma_3 Exp_i dum_i+\beta_4 Exp_i^2 d_i+
\gamma_4 Exp_i^2 dum_i+\beta_5 Fedu_i d_i +\\
\gamma_5 Fedu_i dum_i+\beta_6 Medu_i d_i+\gamma_6 Medu_i dum_i+\e_i, i=1, \ldots, 58
\end{multline*}
по отношению к основной гипотезе $H_0$ является регрессией без ограничений, а регрессия
\begin{multline*}
\ln W_i=\beta_1 +\beta_2 Edu_i+\beta_3 Exp_i+\beta_4 Exp_i^2+\\
\beta_5 Fedu_i+\beta_6 Medu_i+\e_i, i=1, \ldots, 58
\end{multline*}

является регрессией с ограничениями.

Кроме того, для решения задачи должен быть известен следующий факт:

$RSS_{UR}=RSS_1+RSS_2$, где $RSS_{UR}$ — это сумма квадратов остатков в модели:
\begin{multline*}
\ln W_i=\beta_1  d_i+\gamma_1 dum_i+\beta_2 Edu_i d_i+\gamma_2 Edu_i dum_i+\beta_3 Exp_i d_i+\\
\gamma_3 Exp_i dum_i+\beta_4 Exp_i^2 d_i
+ \gamma_4 Exp_i^2 dum_i+\beta_5 Fedu_i d _i +\\
\gamma_5 Fedu_i dum_i+\beta_6 Medu_i d_i+\gamma_6 Medu_i dum_i+\e_i, i=1, \ldots, 58
\end{multline*}

$RSS_1$ — это сумма квадратов остатков в модели:
\begin{multline*}
\ln W_i=\beta_1 +\beta_2 Edu_i+\beta_3 Exp_i+\beta_4 Exp_i^2+\\
\beta_5 Fedu_i+\beta_6 Medu_i+\e_i, i=1, \ldots, 35
\end{multline*}

$RSS_2$ — это сумма квадратов остатков в модели:
\begin{multline*}
\ln W_i=\gamma_1 +\gamma_2 Edu_i+\gamma_3 Exp_i+\gamma_4 Exp_i^2+\\
\gamma_5 Fedu_i+\gamma_6 Medu_i+\e_i, i=36, \ldots, 58
\end{multline*}


\begin{enumerate}
\item Тестовая статистика:

\[
T = \frac{(RSS_R-RSS_{UR})/q}{RSS_{UR}/(n-m)},
\]

где $RSS_R$ — сумма квадратов остатков в модели с ограничениями;

$RSS_{UR}$ — сумма квадратов остатков в модели без ограничений;

$q$ — число линейно независимых уравнений в основной гипотезе $H_0$;

$n$ — общее число наблюдений;

$m$ — число коэффициентов в модели без ограничений

\item Распределение тестовой статистики при верной $H_0$:

\[
T \sim F(q, n-m)
\]

\item Наблюдаемое значение тестовой статистики:

\[
T_{obs} = \frac{(70.3-(34.4+23.4))/6}{(34.4+23.4)/(58-12)}=1.66
\]

\item Область, в которой $H_0$ не отвергается:

\[
[0;T_{cr}]=[0;2.3]
\]

\item Статистический вывод:

Поскольку $T_{obs} \in [0;T_{cr}]$, то на основе имеющихся данных мы не можем отвергнуть гипотезу $H_0$ в пользу альтернативной $H_1$. Следовательно, имеющиеся данные не противоречат гипотезе об отсутствии дискриминации на рынке труда между мужчинами и женщинами.

\end{enumerate}
\end{sol}
\end{problem}


\begin{problem}  % 3.4
 Рассмотрим следующую регрессионную модель зависимости логарифма заработной платы $\ln W$ от уровня образования $Edu$, опыта работы $Exp$, $Exp^2$:
\[
\widehat{\ln W}=\hb_1+\hb_2Edu+\hb_3Exp+\hb_4Exp^2.
\]
Модель регрессии была отдельно оценена по выборкам из 20 мужчин и 20 женщин, и были получены остаточные суммы квадратов $RSS_1$ = 49.4 и $RSS_2$ = 44.1 соответственно. Остаточная сумма квадратов в регрессии, оценённой по объединённой выборке, равна 105.5. На уровне $ 5\%$ проверьте гипотезу об отсутствии дискриминации в оплате труда между мужчинами и женщинами.


\begin{sol}
\end{sol}
\end{problem}



\begin{problem}  % 3.5
 Ниже приведены результаты оценивания спроса на молоко для модели $y_i=\beta_1 + \beta_2 I_i + \beta_3 P_i + \e_i$, где $y_i$ -- стоимость молока, купленного $i$--ой семьёй за последние $7$ дней (в руб.), $I_i$ -- месячный доход $i$--ой семьи (в руб.), $P_i$ -- цена 1 литра молока (в руб.). Вычисления для общей выборки, состоящей из $2127$ семей, дали $RSS = 8841601$. Для двух подвыборок, состоящих из $348$ городских и $1779$ сельских семей, соответствующие суммы квадратов остатков оказались следующими: $RSS_1$ = 1720236 и $RSS_2$ = 7099423. Можно ли считать зависимость спроса на молоко от его цены и дохода единой для городской и сельской местности? Ответ обоснуйте подходящим тестом.


\begin{sol}
Для ответа на вопрос задачи, а именно, можно или нет считать зависимость спроса на молоко от его цены и дохода единой для городской и сельской местностей, воспользуемся гипотезой о нескольких ограничениях. Тогда:
\begin{itemize}
\item Ограниченная («короткая») модель, то есть та модель, которая предполагает выполнение нулевой гипотезы, имеет вид :
\[
R: y_i = \beta_1 + \beta_2I_i + \beta_3P_i + \epsilon_i
\]
\[
RSS_R = RSS = 8841601
\]
\item Для того чтобы записать спецификацию неограниченной («длинной») модели, которая пердполагает разные $\beta_i$ для городской и сельской местностей, введем дополнительную переменную $d_i$, такую что:
\[
d_i=
\begin{cases}
1, \text{город;}\\
0, \text{село}\\
\end{cases}
\]
Пусть коэффициенты для городской местности отличаются на некоторое $\Delta_i$, тогда неограниченная модель имеет вид:
\[
UR: y_i = \beta_1+\Delta_1 d_i + (\beta_2+\Delta_2 d_i)I_i + (\beta_3+\Delta_3 d_i)P_i + \epsilon_i
\]
\[
RSS_{UR} = RSS_1 + RSS_2 = 1720236 + 7099423 = 8819659
\]
\item Гипотезы:
\[
H_0=
\begin{cases}
\Delta_1=0\\
\Delta_2=0\\
\Delta_3=0\\
\end{cases} \;
H_a:\Delta_1^2+\Delta_2^2+\Delta_3^2>0
\]
\item Тестовая статистика имеет вид:
\[
F = \frac{(RSS_R-RSS_{UR})/q}{RSS_{UR}/(n-m)}
\]
где $q$ — число линейно независимых уранений в нулевой гипотезе $H_0$;\\
$n$ — общее число наблюдений;\\
$m$ — число коэффициентов в неограниченной модели
\item Распределение тестовой статистики при верной $H_0$:
\[
F_{cr}\sim(F_{\alpha,q,n-m})
\]
\item Расчётное значение тестовой статистики $F_{obs}=1.758$, $F_{cr}\approx 2.61$
\item Так как $F_{obs}<F_{cr}$, гипотеза $H_0$ не отвергается.
\end{itemize}
Вывод: зависимость спроса на молоко от его цены и дохода для городской и сельской местностей можно считать единой.
\end{sol}
\end{problem}



\begin{problem} % 3.6
 По 52 наблюдениям была оценена следующая зависимость цены квадратного метра квартиры $Price$ (в долларах) от площади кухни $K$ (в квадратных метрах), времени в пути пешком до ближайшего метро $M$ (в минутах), расстояния до центра города $C$ (в км) и наличия рядом с домом лесопарковой зоны $P$ (1 — есть, 0 — нет).
\[
\underset{(se)}{\widehat{Price}}=\underset{(3.73)}{16.12}+\underset{(0.14)}{1.7}K-\underset{(0.03)}{0.35}M-\underset{(0.12)}{0.46}C+\underset{(0.98)}{2.22}P
\]
\[
R^2=0.78, \sum_{i=1}^{52} {(Price_i-\overline{Price})^2}=278
\]
Предположим, что все квартиры в выборке можно отнести к двум категориям: квартиры на севере города (28 наблюдений) и квартиры на юге города (24 наблюдения). Модель регрессии была оценена отдельно только по квартирам на севере и только по квартирам на юге. Ниже приведены результаты оценивания.

Для квартир на севере:
\[
\underset{(se)}{\widehat{Price}}=\underset{(3.3)}{14}+\underset{(0.23)}{1.6}K-\underset{(0.04)}{0.33}M-\underset{(0.22)}{0.4}C+\underset{(0.78)}{2.1}P, RSS=21.8
\]
Для квартир на юге:
\[
\underset{(se)}{\widehat{Price}}=\underset{(3.9)}{16.8}+\underset{(0.4)}{1.62}K-\underset{(0.12)}{0.29}M-\underset{(0.23)}{0.51}C+\underset{(1.28)}{1.98}P, RSS=19.2
\]

На уровне значимости $5\%$ проверьте гипотезу о различии в ценообразовании квартир на севере и на юге.


\begin{sol}
\end{sol}
\end{problem}



\begin{problem} % 3.7
 По 52 наблюдениям была оценена следующая зависимость цены квадратного метра квартиры $Price$ (в долларах) от площади кухни $K$ (в квадратных метрах), времени в пути пешком до ближайшего метро $M$ (в минутах), расстояния до центра города $C$ (в км) и наличия рядом с домом лесопарковой зоны $P$ (1 — есть, 0 — нет).
\[
\underset{(se)}{\widehat{Price}}=\underset{(3.73)}{16.12}+\underset{(0.14)}{1.7}K-\underset{(0.03)}{0.35}M-\underset{(0.12)}{0.46}C+\underset{(0.98)}{2.22}P
\]
\[
R^2=0.78, \sum_{i=1}^{52} {(Price_i-\overline{Price})^2}=278
\]
Предположим, что все квартиры в выборке можно отнести к двум категориям: квартиры на севере города (28 наблюдений) и квартиры на юге города (24 наблюдения). Пусть
$S$ — это фиктивная переменная, равная 1 для домов в южной части города и 0 для домов в северной части города. Используя эту переменную, была оценена следующая регрессия:
\begin{multline*}
\underset{(se)}{\widehat{Price}} =\underset{(3.13)}{14.12}+\underset{(0.11)}{0.25}S+\underset{(0.13)}{1.65}K+\underset{(0.14)}{0.17}K\cdot{S}-\underset{(0.039)}{0.37}M+\\
+\underset{(0.0012)}{0.05}M\cdot{S}-\underset{(0.13)}{0.44}C-\underset{(0.18)}{0.06}C\cdot{S}+\underset{(0.88)}{2.27}P-\underset{(0.08)}{0.23}P\cdot{S}
\end{multline*}
\[
R^2 = 0.85
\]
На уровне значимости $5\%$ проверьте гипотезу о различии в ценообразовании квартир на севере и на юге.


\begin{sol}
Задача решается аналогично предыдущем задачам, к примеру, 3.3, 3.5.

Главное отличие заключается в том, что вместо значений $RSS_{R}$ и $RSS_{UR}$ даются значения соответствующих $R^2$, также следует вспомнить, что $\sum_{i=1}^{n=52}(Price_i-\overline{Price})^2=278$ ни что иное, как $TSS$, которое, в свою очередь, не зависит от спецификации модели, то есть $TSS_R=TSS_{UR}=TSS$. Тогда можно выразить$RSS$ моделей:
\[
\begin{cases}
R^2=\frac{ESS}{TSS} \\
TSS=ESS+RSS
\end{cases}
\to
\begin{cases}
RSS_R=TSS(1-R^2_R)=278(1-0.78)\approx 61.16\\
RSS_{UR}=TSS(1-R^2_{UR})=278(1-0.85)\approx 41.7
\end{cases}
\]

Находим расчётное значение $F$-статистики
\[
F_{obs}=\frac{(61.16-41.7)/5}{41.7/(52-10)}\approx 3.92
\]

Находим критическое значение $F$-статистики
\[
F_{cr}\sim F_{0.05,5,42}\approx 2.44
\]

Получаем, что $F_{obs}>F_{cr}$, и, следовательно, $H_0$ отвергается в пользу альтернативной гипотезы на уровне значимости 5\%.


Вывод: гипотеза об одинаковом ценообразовании квартир на севере и на юге отвергается на уровне значимости 5\%.
\end{sol}
\end{problem}



\begin{problem} % 3.8
На основе квартальных данных с 2003 по 2008 год было получено следующее уравнение регрессии, описывающее зависимость цены на товар P от нескольких факторов:
\[
\hat P_t=3.5+0.4X_t+1.1W_t, ESS=70.4, RSS=40.5
\]
Когда в уравнение были добавлены фиктивные переменные, соответствующие первым
трем кварталам года $Q_1, Q_2, Q_3$, оцениваемая модель приобрела вид:
\[
P_t=\beta+\beta_X X_t+\beta_W W_t+\beta_{Q_{1t}} Q_{1t}+\beta_{Q_{2t}} Q_{2t}+\beta_{Q_{3t}} Q_{3t}+\e_t
\]
При этом величина $ESS = \sum (\hat P_t - \bar P)^2$ выросла до 86.4.

\begin{enumerate}
\item Аккуратно сформулируйте гипотезу об отсутствии сезонности.
\item На уровне значимости $5\%$ проверьте гипотезу о наличии сезонности.
\end{enumerate}



\begin{sol}
Для начала найдём число наблюдений. Будем считать, что данные есть с первого квратала 2003 года по четвёртый квартал 2008 года. Тогда
\[
n = (2008 -2003 + 1) \cdot 4 = 24
\]
\begin{enumerate}
\item
\[
H_0: \begin{cases}
\beta_{Q_{1t}} = 0 \\
\beta_{Q_{2t}} = 0 \\
\beta_{Q_{3t}} = 0
\end{cases}
\quad
H_a: \beta_{Q_{1t}}^2 + \beta_{Q_{2t}}^2 + \beta_{Q_{3t}}^2 >0
\]
\item Поскольку ограниченная и неограниченная модели оценивались по одной и той же выборке, $TSS_R = TSS_{UR} = 70.4 + 40.5 = 110.9$.
Тогда можно найти $RSS_{UR} = 110.9-86.4 = 24.5$. Теперь посчитаем F-статистику, которая при верности $H_0$ имеет распределение $F_{3, 18}$:
\[
F_{obs} = \frac{(RSS_{R} - RSS_{UR})/q}{RSS_{UR}/(n-k_{UR})} = \frac{(40.5-24.5)/3}{24.5/(24-6)} \approx 3.91
\]
Так как $F_{obs} > F_{crit} = 3.16$, основная гипотеза отвергается.
\end{enumerate}
\end{sol}
\end{problem}



\begin{problem} % 3.9
 Рассмотрим следующую функцию спроса с сезонными переменными $SPRING$ (весна), $SUMMER$ (лето), $FALL$ (осень):
\[
\widehat{\ln Q}=\hb_1+\hb_2\cdot{\ln P}+\hb_3\cdot{SPRING}+\hb_4\cdot{SUMMER}+\hb_5\cdot{FALL}
\]
\[
R^2=0.37,n=20
\]
Напишите спецификацию регрессии с ограничениями для проверки статистической гипотезы $H_0: \beta_3 = \beta_5$. Дайте интерпретацию проверяемой гипотезе. Пусть для регрессии с ограничениями был вычислен коэффициент $R_{R}^2=0.23$. На уровне значимости $5\%$ проверьте нулевую гипотезу.


\begin{sol}
Спецификация модели :
\[
\widehat{\ln{Q}} = \hb_1+\hb_2 \ln{P}+\hb_3(SPRING+SUMMER)+\hb_5 FALL
\]
Интерпретация: осень так же влияет на логарифм величины спроса, как и весна. Задача решается аналогично задачам 3.7, 3.5
\[
\begin{cases}
R^2=\frac{ESS}{TSS}\\
TSS=ESS+RSS\\
TSS_R=TSS_{UR}=TSS\\
\end{cases}
\]

Находим расчётное и наблюдаемое значение $F$-статистики
\[
\begin{cases}
F_{obs}=\frac{(R_{UR}^2-R_{R}^2)/q}{(1-R^2_{UR})/(n-m)}\approx3.3\\
F_{cr}= F_{0.05,1,15}\approx 4.54
\end{cases}
\]

Следовательно, $F_{obs}<F_{cr}$ и $H_0$ не отвергается на уровне значимости 5\%.

Вывод: гипотеза $H_0$ о равном влиянии осени и весны на логарифм спроса не отвергается на уровне значимости 5\%.
\end{sol}
\end{problem}



\begin{problem} % 3.10
 Рассмотрим следующую функцию спроса с сезонными переменными $SPRING$ (весна), $SUMMER$ (лето), $FALL$ (осень):
\[
\widehat{\ln Q}=\hb_1+\hb_2\cdot{\ln P}+\hb_3\cdot{SPRING}+\hb_4\cdot{SUMMER}+\hb_5\cdot{FALL}
\]
\[
R^2=0.24,n=24
\]
Напишите спецификацию регрессии с ограничениями для проверки статистической гипотезы
$H_0:
  \begin{cases}
    \beta_3=0, \\
    \beta_4=\beta_5
 \end{cases}.$
Дайте интерпретацию проверяемой гипотезе. Пусть для регрессии с ограничениями был вычислен коэффициент $R_{R}^2=0.13$. На уровне значимости $5\%$ проверьте нулевую гипотезу.


\begin{sol}
Смысл гипотезы: летом и осенью одинаковая зависимость и одинаковая зависимость зимой и весной. Ограниченная модель: $\widehat{\ln Q}=\hb_1+\hb_2\cdot{\ln P}+\hb_3 d$, где $d$ равна 1 для лета и осени. Наблюдаемое значение статистики $F_{obs}=1.375$, критическое, $F_{cr}=3.52$.
\begin{minted}[mathescape,
               numbersep=5pt,
               frame=lines,
               framesep=2mm]{r}
qf(0.95, df1 = 2, df2 = 19)
\end{minted}
Гипотеза не отвергается.
\end{sol}
\end{problem}



\begin{problem}
Исследователь собирается по выборке, содержащей месячные данные за 2 года,
построить модель линейной регрессии с константой и 3-мя объясняющими переменными. В модель предполагается ввести 3 фиктивные сезонные переменные $SPRING$ (весна), $SUMMER$ (лето) и $FALL$ (осень) на все коэффициенты регрессии. Однако в процессе оценивания статистический пакет вывел сообщение «insufficient number of observations». Объясните, почему имеющегося числа наблюдений не хватило для оценивания параметров модели.


\begin{sol}
Наблюдений: 12. Коэффициентов: $4 \cdot 4 = 16$.
\end{sol}
\end{problem}



\begin{problem}
По данным для 57 индивидов оценили зависимость длительности обучения индивида $S$ от способностей индивида, описываемых обобщённой переменной $IQ$, и пола индивида, описываемого с помощью фиктивной переменной $MALE$ (равной 1 для мужчин и 0 для женщин), с помощью двух регрессий (в скобках под коэффициентами указаны оценки стандартных отклонений):
\[
\underset{(se)}{\hat{S}}=\underset{(0.44)}{6.12}+\underset{(0.088)}{0.147}\cdot{IQ}, RSS=2758.6
\]
\begin{multline*}
\underset{(se)}{\hat{S}}=\underset{(0.73)}{6.12}+\underset{(0.014)}{0.147}\cdot{IQ}-\underset{(0.933)}{1.035}\cdot{MALE}+\underset{(0.018)}{0.0166}\cdot{(MALE}\cdot{IQ)}
\end{multline*}
Во второй регрессии сумма квадратов остатков равна $RSS=2090.98$
Зависит ли длительность обучения от пола индивида и почему?


\begin{sol}
\end{sol}
\end{problem}



\begin{problem}
По данным, содержащим 30 наблюдений, построена регрессия:
\[
\hy=1.3870+5.2587\cdot{x}+2.6259\cdot{d}+2.5955 \cdot{x} \cdot{d},
\]
где фиктивная переменная $d$ определяется следующим образом:
\[
d_i =
  \begin{cases}
    1 & \text{при $i$ $\in \bigl\{ 1,\dots,20 \bigr\} $}, \\
    0 & \text{при $i$ $\in \bigl\{ 21,\dots,30 \bigr\} $}.
 \end{cases}
\]
Найдите оценки коэффициентов в модели $y_i=\beta_1+\beta_2 x_i+\e_i$, построенной по первым 20-ти наблюдениям, т.е. при $i \in \bigl\{1,\dots,20 \bigr\}$.


\begin{sol}
$\hb_1=1.3870+2.6259=4.0129$, $\hb_2=5.2587+2.5955=7.8542$
\end{sol}
\end{problem}



\begin{problem} % 3.14
 Выборка содержит 30 наблюдений зависимой переменной $y$ и независимой переменной $x$. Ниже приведены результаты оценивания уравнения регрессии $y_i=\beta_1+\beta_2 x_i+\e_i$ по первым 20-ти и последним 10-ти наблюдениям соответственно:
\[
\hy=4.0039+2.6632\cdot{x}
\]
\[
\hy=1.3780+5.2587\cdot{x}
\]
По имеющимся данным найдите оценки коэффициентов  модели, рассчитанной по 30-ти наблюдениям $y_i=\beta_1+\beta_2 x_i+\Delta{\beta_1} \cdot{d_i} + \Delta{\beta_2} \cdot{x_i} \cdot{d_i}+\e_i$, где фиктивная переменная $d$ определяется следующим образом:
\[
d_i =
  \begin{cases}
    1 & \text{при } i \in \bigl\{ 1,\dots,20 \bigr\} , \\
    0 & \text{при } i \in \bigl\{ 21,\dots,30 \bigr\} .
 \end{cases}
\]


\begin{sol}
$y_i=\beta_1+\beta_2(x_{i1}+x_{i2}+x_{i3})+\epsilon_{i}$
\end{sol}
\end{problem}



\begin{problem} % 3.15
 Пусть регрессионная модель имеет вид $y_i=\beta_1+\beta_2 x_{i1}+\beta_3 x_{i2}+\beta_4 x_{i3}+\e_i, i=1,\dots,n.$ Тестируемая гипотеза $H_0: \beta_2=\beta_3=\beta_4.$ Запишите, какой вид имеет ограниченная модель для тестирования указанной гипотезы.

\begin{sol}
$y_i=\beta_1+\beta_2(x_{i1}+x_{i2}+x_{i3})+\epsilon_{i}$
\end{sol}
\end{problem}



\begin{problem}
Пусть регрессионная модель имеет вид $y_i= \beta_1+ \beta_2 x_{i1}+ \beta_3 x_{i2}+ \beta_4 x_{i3}+\e_i, i=1,\dots,n.$ Тестируемая гипотеза $H_0: \beta_3= \beta_4=1.$ Какая модель из приведённых ниже может выступать в качестве ограниченной для тестирования указанной гипотезы? Если ни одна из них, то запишите свою.
\begin{enumerate}
\item $y_i-(x_{i2}+x_{i3}) = \beta_1+ \beta_2 x_{i1}+ \e_i$;
\item $y_i+(x_{i2}-x_{i3}) = \beta_1+ \beta_2 x_{i1}+ \e_i$;
\item $y_i+x_{i2}+x_{i3} = \beta_1+ \beta_2 x_{i1}+ \e_i$;
\item $y_i= \beta_1+ \beta_2 x_{i1}+ \beta_3 + \beta_4 + \e_i$.
\end{enumerate}


\begin{sol}
1
\end{sol}
\end{problem}





\begin{problem}
Пусть регрессионная модель имеет вид $y_i= \beta_1+ \beta_2 x_{i1}+ \beta_3 x_{i2}+ \beta_4 x_{i3}+\e_i, i=1,\dots,n.$ Тестируемая гипотеза
$H_0:
  \begin{cases}
    \beta_2+ \beta_3+ \beta_4=1, \\
    \beta_3+ \beta_4=0.
 \end{cases}$
Какая модель из приведённых ниже может выступать в качестве ограниченной модели для тестирования указанной гипотезы? Если ни одна из них, то запишите свою.
\begin{enumerate}
\item $y_i-x_{i1} = \beta_1+ \beta_3 (x_{i2}-x_{i3})+ \e_i$;
\item $y_i-x_{i1} = \beta_1+ \beta_4 (x_{i3}-x_{i2})+ \e_i$;
\item $y_i+x_{i1} = \beta_1+ \beta_3 (x_{i2}+x_{i3})+ \e_i$;
\item $y_i+x_{i1} = \beta_1+ \beta_3 (x_{i2}-x_{i3})+ \e_i$.
\end{enumerate}


\begin{sol}
1,2
\end{sol}
\end{problem}



\begin{problem} % 3.18
 Пусть регрессионная модель имеет вид $y_i= \beta_1+ \beta_2 x_{i1}+ \beta_3 x_{i2}+ \beta_4 x_{i3}+\e_i, i=1,\dots,n.$ Тестируемая гипотеза
$H_0:
  \begin{cases}
    \beta_2 - \beta_3=0, \\
    \beta_3 + \beta_4=0.
 \end{cases}$
Какая модель из приведённых ниже может выступать в качестве ограниченной модели для тестирования указанной гипотезы? Если ни одна из них, то запишите свою.
\begin{enumerate}
\item $y_i = \beta_1 + \beta_3 (x_{i2}-x_{i1}-x_{i3})+ \e_i$;
\item $y_i-x_{i1} = \beta_1+ \beta_4 (x_{i3}-x_{i2})+ \e_i$;
\item $y_i = \beta_1+ \beta_3 (x_{i1}+x_{i2}+x_{i3})+ \e_i$;
\item $y_i = \beta_1+ \beta_3 (x_{i1}+x_{i2}-x_{i3})+ \e_i$.
\end{enumerate}


\begin{sol}
4
\end{sol}
\end{problem}



\begin{problem} % 3.19
Известно, что $P$-значение для коэффициента регрессии равно 0.087, а уровень значимости 0.1. Является ли значимым данный коэффициент в регрессии?


\begin{sol}
Значим.
\end{sol}
\end{problem}



\begin{problem} % 3.20
Известно, что $P$-значение для коэффициента регрессии равно 0.078, а уровень значимости 0.05. Является ли значимым данный коэффициент в регрессии?


\begin{sol}
Не значим.
\end{sol}
\end{problem}



\begin{problem}
Известно, что $P$-значение для коэффициента регрессии равно 0.09. На каком
уровне значимости данный коэффициент в регрессии будет признан значимым?


\begin{sol}
$\alpha>0.09$
\end{sol}
\end{problem}



\begin{problem}
Ниже приведены результаты оценивания уравнения линейной регрессии зависимости количества смертей в автомобильных катастрофах от различных характеристик:
\[
deaths_i = \beta_1 + \beta_2 drivers_i + \beta_3 popden_i + \beta_4  temp + \beta_5 fuel + \e_i
\]
\begin{minted}[mathescape,
               numbersep=5pt,
               frame=lines,
               framesep=2mm]{r}
model1 <- lm(deaths ~ drivers + popden + temp + fuel, data = MASS::road)
xmodel(model1, below = "se")
report <- summary(model1)
coefficient_table <- report$coefficients
rownames(coefficient_table) <- c("Intercept", "Drivers", "Popden", "Temp", "Fuel")
colnames(coefficient_table) <- c("Estimate", "St.Error", "t value", "P-value")
xtable(coefficient_table)
\end{minted}

\begin{tabular}{rrrrr}
  \hline
 & Estimate & St.Error & t value & P-value \\
  \hline
Intercept & -27.10 & 222.88 & -0.12 & 0.90 \\
  Drivers & 4.64 & 0.38 & 12.30 & 0.00 \\
  Popden & -0.02 & 0.02 & -0.95 & 0.35 \\
  Temp & 5.30 & 4.60 & 1.15 & 0.26 \\
  Fuel & -0.66 & 0.87 & -0.76 & 0.45 \\
   \hline
\end{tabular}


Перечислите, какие из коэффициентов в регрессии значимы на 5\%-ом уровне значимости.

\begin{sol}
Только коэффициент при переменной Drivers.
\end{sol}
\end{problem}



\begin{problem}
Была оценена функция Кобба-Дугласа с учётом человеческого капитала $H$ ($K$ — физический капитал, $L$ — труд):
\[
\widehat{\ln Q} = 1.4 + 0.46\ln L + 0.27\ln H + 0.23\ln K
\]
\[
ESS = 170.4, RSS = 80.3, n = 21
\]
\begin{enumerate}
\item Чему равен коэффициент $R^2$?
\item На уровне значимости $1\%$ проверьте гипотезу о значимости регрессии в целом.
\end{enumerate}


\begin{sol}
Из формул
\[
\begin{cases}
R^2=\frac{ESS}{TSS}\\
TSS=ESS+RSS\\
\end{cases}
\]
получаем $R^2=\frac{170.4}{(170.4+80.3)}\approx=0.68$

Тестируемые гипотезы:
\[
H_0=
\begin{cases}
\beta_2=0\\
\beta_3=0\\
\beta_4=0\\
\end{cases}
\;
H_a:\beta_2^2+\beta_3^2+\beta_4^2>0
\]

Так как по условию задачи проверяем значимость модели в целом, следовательно ограниченная модель — регрессия на константу, таким образом:
\[
\begin{cases}
\widehat{y_i}=\bar{y}\\
RSS_{R}=\sum_{i=1}^{n}(y_i-\widehat{y_i})^2=\sum_{i=1}^{n}(y_i-\overline{y_i})^2=TSS\\
RSS_{UR}=TSS(1-R^2_{UR})\\
TSS_{UR}=TSS_{R}=TSS\\
\end{cases}
\]

Получаем, $F_{obs}=\frac{R_{UR}^2/q}{(1-R^2_{UR})/(n-m)}$

Значения статистик:
\[
\begin{cases}
F_{obs}\approx 12.04\\
F_{cr}=F(0.01,3,17)\approx 5.185
\end{cases}
\]

Отсюда,
$F_{obs}>F_{cr}$, и  $H_0$ отвергается на уровне значимости 1\%.

Вывод: гипотеза $H_0$ отвергается на уровне значимости 1\%,
следовательно модель в целом значима.
\end{sol}
\end{problem}





\begin{problem}
На основе опроса 25 человек была оценена следующая модель зависимости логарифма зарплаты $\ln W$ от уровня образования $Edu$ (в годах) и возраста $Age$:
\[
\widehat{\ln W} = 1.7 + 0.5Edu + 0.06Age - 0.0004Age^2
\]
\[
ESS = 90.3, RSS = 60.4
\]
Когда в модель были введены переменные $Fedu$ и $Medu$, учитывающие уровень образования родителей, величина $ESS$ увеличилась до $110.3.$
\begin{enumerate}
\item Напишите спецификацию уравнения регрессии с учётом образования родителей.
\item Сформулируйте и на уровне значимости $5\%$ проверьте гипотезу о значимом влиянии уровня образования родителей на заработную плату:
\begin{enumerate}
\item Сформулируйте гипотезу.
\item Приведите формулу для тестовой статистики.
\item Укажите распределение тестовой статистики при верной $H_0$.
\item Вычислите наблюдаемое значение тестовой статистики.
\item Укажите границы области, где основная гипотеза не отвергается.
\item Сделайте статистический вывод.
\end{enumerate}
\end{enumerate}


\begin{sol}

Ограниченная модель (Restricted model):
\[
\ln W_i = \beta + \beta_{Edu}Edu_i + \beta_{Age}Age_i + \beta_{Age^2}Age^2_i + \e_i
\]
Неограниченная модель (Unrestricted model):
\begin{multline*}
\ln W_i = \beta + \beta_{Edu}Edu_i + \beta_{Age}Age_i + \beta_{Age^2}Age^2_i + \\
\beta_{Fedu}Fedu_i + \beta_{Medu}Medu_i + \e_i
\end{multline*}

По условию $ESS_R = 90.3$, $RSS_R = 60.4$, $TSS = ESS_R + RSS_R = 90.3 + 60.4 = 150.7.$ Также сказано, что $ESS_{UR} = 110.3$. Значит, $RSS_{UR} = TSS - ESS_{UR} = 150.7 - 110.3 = 40.4$
\begin{enumerate}
\item Cпецификация:
\begin{multline*}
\ln W_i = \beta + \beta_{Edu}Edu_i + \beta_{Age}Age_i + \beta_{Age^2}Age^2_i + \\
\beta_{Fedu}Fedu_i + \beta_{Medu}Medu_i + \e_i
\end{multline*}
\item Проверка гипотезы
\begin{enumerate}
\item $H_0: \begin{cases}
\beta_{Fedu} = 0  \\
\beta_{Medu} = 0
\end{cases}$
$H_a: |\beta_{Fedu}| + |\beta_{Medu}| > 0$
\item $T = \frac{(RSS_{R} - RSS_{UR})/q}{RSS_{UR}/(n - k)}$, где $q = 2$ — число линейно независимых уравнений в основной гипотезе $H_0$, $n = 25$ — число наблюдений, $k = 6$ — число коэффициентов в модели без ограничения
\item $T \sim F(q; n - k)$
\item $T_{obs} = \frac{(RSS_{R} - RSS_{UR})/q}{RSS_{UR}/(n - k)} = \frac{(60.4 - 40.4)/2}{40.4/(25 - 6)} = 4.70$
\item Нижняя граница равна $0$, верхняя граница равна $3.52$
\item Поскольку $T_{obs} = 4.70$, что не принадлежит промежутку от $0$ до $3.52$, то на основе имеющихся данных можно отвергнуть основную гипотезу на уровне значимости $5\%$. Таким образом, образование родителей существенно влияет на заработную плату.
\end{enumerate}
\end{enumerate}

\end{sol}
\end{problem}




\begin{problem}
Рассмотрим следующую модель зависимости цены дома $Price$ (в тысячах долларов) от его площади $Hsize$ (в квадратных метрах), площади участка $Lsize$ (в квадратных метрах), числа ванных комнат $Bath$ и числа спален $BDR$:
\[
\widehat{Price} = \hb_1 + \hb_2 Hsize + \hb_3 Lsize + \hb_4 Bath + \hb_5 BDR
\]
\[
R^2 = 0.218, n = 23
\]
Напишите спецификацию регрессии с ограничениями для проверки статистистической гипотезы $H_0: \beta_3 = 20\beta_4$. Дайте интерпретацию проверяемой гипотезе. Для регрессии с ограничением был вычислен коэффициент $R_{R}^2 = 0.136$. На уровне значимости $5\%$ проверьте нулевую гипотезу.


\begin{sol}
\[
\widehat{Price}= \hb_1+\hb_2 Hsize+20\hb_4 Lsize+\hb_4 Bath + \hb_5 BDR
\]

Размер участка в 20 раз сильнее влияет на цену дома, чем число ванных комнат.
\[
\begin{cases}
R^2=\frac{ESS}{TSS}\\
TSS=ESS+RSS\\
TSS_R=TSS_{UR}=TSS\\
\end{cases}
\]

\[
\begin{cases}
RSS_{R}=TSS(1-R^2_{R})\\
RSS_{UR}=TSS(1-R^2_{UR})\\
\end{cases} \to
\]

\[
\begin{cases}
F_{obs}=\frac{(R_{UR}^2-R_{R}^2)/q}{(1-R^2_{UR})/(n-m)}=\frac{(0.218-0.136)/1}{(1-0.218)/18}\approx 1.887\\
F_{cr}= F_{0.05,1,18}\approx 4.41
\end{cases}
\]

$F_{obs}<F_{cr}$ и, следовательно, $H_0$ не отвергается на уровне значимости 5\%.

Вывод: гипотеза $H_0$ о том, что размер участка в 20 раз сильнее влияет на цену дома, чем число ванных комнат, не отвергается на уровне значимости 5\%.
\end{sol}
\end{problem}



\begin{problem}
Рассмотрим следующую модель зависимости почасовой оплаты труда $W$ от уровня образования $educ$, 
возраста $age$, уровня образования родителей $fathedu$ и $mothedu$:
\[
\widehat{\ln W}_i = \hb_1 + \hb_2 educ_i + \hb_3 age_i + \hb_4 age^2_i+ \hb_5 fathedu_i + \hb_6 mothedu_i
\]
\[
R^2 = 0.341, n = 27
\]
Напишите спецификацию регрессии с ограничениями для проверки статистистической гипотезы $H_0: \beta_5 = 2\beta_4$. 
Дайте интерпретацию проверяемой гипотезе. Для регрессии с ограничением был вычислен коэффициент $R_{R}^2 = 0.296$. 
На уровне значимости $5\%$ проверьте нулевую гипотезу.


\begin{sol}
  \[
    \ln \hat W_i = \hat{\beta}_1 + \hat{\beta}_2 educ_i + \hat{\beta}_3 age_i + \hat{\beta}_4 age^2_i + \hat{\beta}_5 fathedu_i + \hat{\beta}_6 mothedu_i
    \]
    \[
    R^2 = 0.341, n = 27
    \]
    Образование отца в два раза сильнее влияет на уровень зп, чем зарплата матери. 
    \[
    \begin{cases}
    R^2=\frac{ESS}{TSS}\\
    TSS=ESS+RSS\\
    TSS_R=TSS_{UR}=TSS\\
    \end{cases}
    \]
    \[
    \begin{cases}
    RSS_{R}=TSS(1-R^2_{R})\\
    RSS_{UR}=TSS(1-R^2_{UR})\\
    \end{cases} 
    \]
    
    Наблюдаемое значение статистики
    \[
    F_{obs}=\frac{(R_{UR}^2-R_{R}^2)/q}{(1-R^2_{UR})/(n-k)}=\frac{(0.341-0.296)/1}{(1-0.341)/22}\approx 1.503
    \]
    
    Критическое:
    \[
    F_{crit}= F_{0.05,1,22}\approx 4.3
    \]
    
    $F_{obs}<F_{crit}$ и, следовательно, $H_0$ не отвергается на уровне значимости 5\%.
\end{sol}
\end{problem}



\begin{problem} % 3.27
 По данным для 27 фирм исследователь оценил зависимость объёма выпуска $y$ от труда $l$ и капитала $k$ с помощью двух моделей:
\[
\ln y_i = \beta_1 + \beta_2 \ln l_i + \beta_3 \ln k_i + \e_i
\]
\[
\ln y_i = \beta_1 + \beta_2 \ln (l_i \cdot k_i) + \e_i
\]
Он получил для этих двух моделей суммы квадратов остатков $RSS_1 = 0.851$ и $RSS_2 = 0.894$ соответственно. Сформулируйте гипотезу, которую хотел проверить исследователь. На уровне значимости $5\%$ проверьте эту гипотезу и дайте экономическую интерпретацию.


\begin{sol}
$H_0: \beta_2=\beta_3$ — труд и капитал вносят одинаковый вклад в выпуск фирмы.

\[
\begin{cases}
F_{obs}=\frac{(RSS_R-RSS_{UR})/q}{RSS_{UR}/(n-m)}=\frac{(0.894-0.851)/1}{0.851/(27-3)}\approx 1.213\\
F_{cr}= F_{0.05,1,24}\approx 4.26
\end{cases}
\]

Получаем, что $F_{obs}<F_{cr}$, и, следовательно, $H_0$ не отвергается на уровне значимости 5\%

Вывод: гипотеза $H_0$, предполагающая, что труд и капитал вносят одинаковый вклад в выпуск фирмы, не отвергается на уровне значимости 5\%.
\end{sol}
\end{problem}



\begin{problem} % 3.28

 Пусть задана линейная регрессионная модель:
\[
y_i = \beta_1 + \beta_2 x_{i} + \beta_3 z_{i} + \beta_4 w_{i} + \beta_5 q_{i} + \e_i, i = 1, \dots, 20
\]
По имеющимся данным оценены следующие регрессии:
\[
\underset{(se)}{\hy_i} = \underset{(0.15)}{10.01} + \underset{(0.06)}{1.05}x_i + \underset{(0.04)}{2.06}z_i + \underset{(0.06)}{0.49}w_i - \underset{(0.06)}{1.31}q_i, RSS = 6.85
\]
\[
\underset{(se)}{\widehat{y_i- x_i - 2z_i}} = \underset{(0.15)}{10.00} + \underset{(0.07)}{0.50}w_i - \underset{(0.06)}{1.32}q_i, RSS = 8.31
\]
\[
\underset{(se)}{\widehat{y_i + x_i + 2z_i}} = \underset{(3.62)}{9.93} + \underset{(1.48)}{0.56}w_i - \underset{(1.42)}{1.50}q_i, RSS = 4310.62
\]
\[
\underset{(se)}{\widehat{y_i - x_i + 2z_i}} = \underset{(3.26)}{10.71} + \underset{(1.33)}{0.09}w_i - \underset{(1.28)}{1.28}q_i, RSS = 3496.85
\]
\[
\underset{(se)}{\widehat{y_i + x_i - 2z_i}} = \underset{(1.25)}{9.22} + \underset{(0.51)}{0.97}w_i - \underset{(0.49)}{1.54}q_i, RSS = 516.23
\]
На уровне значимости $5\%$ проверьте гипотезу $H_0: \begin{cases} \beta_2 = 1 \\ \beta_3 = 2 \end{cases}$ против альтернативной гипотезы $H_a: |\beta_2 - 1| + |\beta_3 - 2| \not= 0$.


\begin{sol}
Здесь $RSS_{R}=8.31$, $RSS_{UR}=6.85$.
\end{sol}
\end{problem}





\begin{problem} % 3.29
 Рассмотрим следующую модель зависимости расходов на образование на душу населения от дохода на душу населения, доли населения в возрасте до 18 лет, а также доли городского населения:
\[
education_i = \beta_1 + \beta_2 income_i + \beta_3 young_i + \beta_4 urban_i + \e_i
\]
Ниже приведены результаты оценивания уравнения этой линейной регрессии:

\begin{minted}[mathescape,
               numbersep=5pt,
               frame=lines,
               framesep=2mm]{r}
model1 <- lm(education ~ income + young + urban, data = carData::Anscombe)
xmodel(model1, below = "se")
report <- summary(model1)
coefficient_table <- report$coefficients
rownames(coefficient_table) <- c("Intercept", "Income", "Young", "Urban")
colnames(coefficient_table) <- c("Estimate", "St.Error", "t value", "P-value")
xtable(coefficient_table)
\end{minted}

\ensuremath{\widehat{education}_i=-\underset{( 64.9199 )}{ 287 }+\underset{( 0.0093)}{0.0807}\cdot income_i+\underset{( 0.1598)}{0.817}\cdot young_i-\underset{( 0.0343)}{0.106}\cdot urban_i}


\begin{center}
\begin{tabular}{rrrrr}
  \hline
 & Estimate & St.Error & t value & P-value \\
  \hline
Intercept & -286.84 & 64.92 & -4.42 & 0.00 \\
  Income & 0.08 & 0.01 & 8.67 & 0.00 \\
  Young & 0.82 & 0.16 & 5.12 & 0.00 \\
  Urban & -0.11 & 0.03 & -3.09 & 0.00 \\
   \hline
\end{tabular}
\end{center}

Модель оценивается по 51 наблюдению.

\begin{enumerate}
\item Сформулируйте основную и альтернативую гипотезы, которые соответствуют тесту на значимость коэффициента при переменной доход на душу населения в уравнении регрессии.
\item На уровне значимости $10\%$ проверьте гипотезу о значимости коэффициента при переменной доход на душу населения в уравнении регрессии:
\begin{enumerate}
\item Приведите формулу для тестовой статистики.
\item Укажите распределение тестовой статистики при верной $H_0$.
\item Вычислите наблюдаемое значение тестовой статистики.
\item Укажите границы области, где основная гипотеза не отвергается.
\item Сделайте статистический вывод.
\end{enumerate}
\item На уровне значимости $5\%$ проверьте гипотезу $H_0: \beta_1 = 1$ против альтернативной $H_a: \beta_1 > 1$:
\begin{enumerate}
\item Приведите формулу для тестовой статистики.
\item Укажите распределение тестовой статистики при верной $H_0$.
\item Вычислите наблюдаемое значение тестовой статистики.
\item Укажите границы области, где основная гипотеза не отвергается.
\item Сделайте статистический вывод.
\end{enumerate}
\item Сформулируйте основную гипотезу, которая соответствует тесту на значимость регрессии в целом.
\item На уровне значимости $1\%$ проверьте гипотезу о значимости регрессии в целом, если известно, что соответствующая $F$-статистика равна $34.8$:
\begin{enumerate}
\item Приведите формулу для тестовой статистики.
\item Укажите распределение тестовой статистики  при верной $H_0$.
\item Укажите границы области, где основная гипотеза не отвергается.
\item Сделайте статистический вывод.
\end{enumerate}

Далее приведены результаты оценивания уравнения регрессии без переменной, отражающей долю городского населения:

\begin{minted}[mathescape,
               numbersep=5pt,
               frame=lines,
               framesep=2mm]{r}
model2 <- lm(education ~ income + young, data = carData::Anscombe)
xmodel(model2, below = "se")
report2 <- summary(model2)
coefficient_table2 <- report2$coefficients
rownames(coefficient_table2) <- c("Intercept", "Income", "Young")
colnames(coefficient_table2) <- c("Estimate", "St.Error", "t value", "P-value")
xtable(coefficient_table2)
\end{minted}


\begin{center}
\ensuremath{\widehat{education}_i=-\underset{( 70.271 )}{ 301 }+\underset{( 0.00741)}{0.0612}\cdot income_i+\underset{( 0.173)}{0.836}\cdot young_i}

\begin{tabular}{rrrrr}
  \hline
 & Estimate & St.Error & t value & P-value \\
  \hline
Intercept & -301.09 & 70.27 & -4.28 & 0.00 \\
  Income & 0.06 & 0.01 & 8.25 & 0.00 \\
  Young & 0.84 & 0.17 & 4.83 & 0.00 \\
   \hline
\end{tabular}

\end{center}
Также известно, что $RSS$ для первой модели равен $33489$, а для второй модели — $40277$. На уровне значимости $5\%$ проверьте гипотезу $H_0: \beta_4 = 0$ против альтернативной $H_0: \beta_4 \not= 0$:
\begin{enumerate}
\item Приведите формулу для тестовой статистики.
\item Укажите распределение тестовой статистики  при верной $H_0$.
\item Вычислите наблюдаемое значение тестовой статистики.
\item Укажите границы области, где основная гипотеза не отвергается.
\item Сделайте статистический вывод.
\end{enumerate}
\end{enumerate}


\begin{sol}
\begin{enumerate}
\item
\[
\begin{cases}
H_0: \beta_2=0\\
H_a: \beta_2\neq 0
\end{cases}
\]

\item
\begin{enumerate}
\item $t=\frac{\hb_i-\beta_i}{se(\beta_i)}$
\item $t_{\alpha,n-m}=t_{0.05,47}$
\item $t=\frac{0.08-0}{0.0093}\approx 8.67$
\item $[-t_{cr},t_{cr}]$
\item гипотеза $H_0$ отвергается, так как $P$-значение равно нулю; можно честно посчитать $t_{cr}=t_{0.05,47}$ или вспомнить, что при количестве наблюдений больше 30, $t$-распределение похоже на нормальное, для которого квантиль на уровне 5\% примерно равна $1.67$ и $F_{obs}>F_{cr}$. Гипотеза $H_0$ отвергается, следовательно коэффициент $\beta_2$ значим на уровне значимости 10\%.
\end{enumerate}

\item
\begin{enumerate}
\item $t=\frac{\hb_i-\beta_i}{se(\beta_i)}$
\item $t_{\alpha, n-m} = t_{0.05, 47}$
\item $t=\frac{-287-1}{64.92}\approx -4.42$
\item $(-\infty, t_{cr}]$
\item гипотеза $H_0$ отвергается, так как $P$-значение равно нулю; аналогично 2(e) $t_{cr}=t_{0.05,47}\approx 1.67$ и $F_{obs}>F_{cr}$. Гипотеза $H_0$ отвергается, следовательно коэффициент $\beta_1$ значим на уровне значимости 5\%.
\end{enumerate}

\item
\[
H_0=
\begin{cases}
\beta_2=0\\
\beta_3=0\\
\beta_4=0\\
\end{cases}
\;
H_a:\beta_2^2+\beta_3^2+\beta_4^2>0
\]

\item
\begin{enumerate}
\item $F =\frac{(RSS_R-RSS_{UR})/q}{RSS_{UR}/(n-m)}$
\item $F_{\alpha,q,n-m}=F_{0.01,3,47}$
\item $F=34.81$
\item $[0,F_{cr}]$
\item гипотеза $H_0$ отвергается, так как $P$-значение примерно равно $0$, точнее меньше $(5.337\cdot 10^{-12})$; можно вычислить $F_{cr}=F_{0.01,3,47} \approx 4.23$. Следовательно, $F_{obs}>F_{cr}$ и $H_0$ отвергается, и регрессия в целом значима на уровне значимости 1\%.
\end{enumerate}

\item
\begin{enumerate}
\item $F =\frac{(RSS_R-RSS_{UR})/q}{RSS_{UR}/(n-m)}$
\item $F_{\alpha,q,n-m}=F_{0.05,3,47}$
\item $F\approx 9.525$
\item $[0,F_{cr}]$
\item гипотеза $H_0$ отвергается, так как $F_{cr}=F_{0.05,3,47} \approx 4.047$ и $F_{obs}>F_{cr}$, следовательно коэффициент $\beta_4$ значим на уровне значимости 5\%.
\end{enumerate}
\end{enumerate}
\end{sol}
\end{problem}




\begin{problem}
Рассмотрим следующую модель зависимости расходов на образование на душу населения от дохода на душу населения, доли населения в возрасте до 18 лет, а также доли городского населения:
\[
education_i = \beta_1 + \beta_2 income_i + \beta_3 young_i + \beta_4 urban_i + \e_i
\]
Модель оценивается по 51 наблюдению.
Ниже приведены результаты оценивания уравнения этой линейной регрессии:

\begin{minted}[mathescape,
               numbersep=5pt,
               frame=lines,
               framesep=2mm]{r}
model1 <- lm(education ~ income + young + urban, data = carData::Anscombe)
xmodel(model1, below = "se")
report <- summary(model1)
coefficient_table <- report$coefficients
rownames(coefficient_table) <- c("Intercept", "Income", "Young", "Urban")
colnames(coefficient_table) <- c("Estimate", "St.Error", "t value", "P-value")
xtable(coefficient_table)
\end{minted}


\begin{center}
\ensuremath{\widehat{education}_i=-\underset{( 64.9199 )}{ 287 }+\underset{( 0.0093)}{0.0807}\cdot income_i+\underset{( 0.1598)}{0.817}\cdot young_i-\underset{( 0.0343)}{0.106}\cdot urban_i}

\begin{tabular}{rrrrr}
  \hline
 & Estimate & Std. Error & t value & Pr($>$$|$t$|$) \\
  \hline
(Intercept) & -286.8388 & 64.9199 & -4.42 & 0.0001 \\
  income & 0.0807 & 0.0093 & 8.67 & 0.0000 \\
  young & 0.8173 & 0.1598 & 5.12 & 0.0000 \\
  urban & -0.1058 & 0.0343 & -3.09 & 0.0034 \\
   \hline
\end{tabular}
\end{center}

\begin{enumerate}
\item Сформулируйте основную и альтернативую гипотезы, которые соответствуют тесту на значимость коэффициента при переменной доля населения в возрасте до 18 лет в уравнении регрессии
\item На уровне значимости $10\%$ проверьте гипотезу о значимости коэффициента при переменной доля населения в возрасте до 18 лет в уравнении регрессии:
\begin{enumerate}
\item Приведите формулу для тестовой статистики.
\item Укажите распределение тестовой статистики при верной $H_0$.
\item Вычислите наблюдаемое значение тестовой статистики.
\item Укажите границы области, где основная гипотеза не отвергается.
\item Сделайте статистический вывод.
\end{enumerate}


Далее приведены результаты оценивания уравнения регрессии без переменной, отражающей долю населения в возрасте до 18 лет:

\begin{minted}[mathescape, numbersep=5pt, frame=lines, framesep=2mm]{r}
model3 <- lm(education ~ income + urban, data = carData::Anscombe)
xmodel(model3, below = "se")
report3 <- summary(model3)
coefficient_table3 <- report3$coefficients
rownames(coefficient_table3) <- c("Intercept", "Income", "Urban")
colnames(coefficient_table3) <- c("Estimate", "St.Error", "t value", "P-value")
xtable(coefficient_table3)
\end{minted}


\begin{center}

\ensuremath{\widehat{education}_i=\underset{( 27.3827 )}{ 25.3 }+\underset{( 0.0114)}{0.0762}\cdot income_i-\underset{( 0.0423)}{0.112}\cdot urban_i}

\begin{tabular}{rrrrr}
  \hline
 & Estimate & St.Error & t value & P-value \\
  \hline
Intercept & 25.25 & 27.38 & 0.92 & 0.36 \\
  Income & 0.08 & 0.01 & 6.67 & 0.00 \\
  Urban & -0.11 & 0.04 & -2.66 & 0.01 \\
   \hline
\end{tabular}

\end{center}
Также известно, что $RSS$ для первой модели равен $33489.35$, а для второй модели — $52132.29$. 
На уровне значимости $5\%$ проверьте гипотезу $H_0: \beta_3 = 0$ против альтернативной $H_0: \beta_3 \not= 0$:
\begin{enumerate}
\item Приведите формулу для тестовой статистики.
\item Укажите распределение тестовой статистики при верной $H_0$.
\item Вычислите наблюдаемое значение тестовой статистики.
\item Укажите границы области, где основная гипотеза не отвергается.
\item Сделайте статистический вывод.
\end{enumerate}
\end{enumerate}


\begin{sol}
  \begin{enumerate}
    \item Гипотеза о значимости коэффициента при переменной доля населения до 18 лет. 
    \[H_0: \beta_3=0\]
    \[H_a: \beta_3 \neq 0\]
    
    
    \item 
    Формула тестовой статистики 
\[
t_{obs} = \frac{\hat{\beta}_3-\beta_3}{se(\hat{\beta}_3)}= \frac{0.8173}{0.1598}\approx 5.11
\]
при верной гипотезе $H_0$ распределена по Стьюденту с 51-4 степенями свободы. 
    
Гипотеза $H_0$ не отвергается, если $t$-статистика попадает в интервал $[-1.68; 1.68]$.
    
    \textbf{Вывод}: Гипотеза $H_0$ о незначимости коэффициента отвергается, так как значение рассчитанной статистики не входит в интервал. 
\end{enumerate}
\end{sol}
\end{problem}




\begin{problem}
Вася построил регрессию оценки за первую контрольную работу на константу, рост и вес студента, $\widehat{kr1_i}=\hb_1 + \hb_2 height_i + \hb_3 weight_i$. Затем построил регрессию оценки за вторую контрольную работу на те же объясняющие переменные, $\widehat{kr2_i}=\hb_1' + \hb_2' height_i + \hb_3' weight_i$. Накопленная оценка считается по формуле $nak_i=0.25 \cdot kr1_i + 0.75 \cdot kr2_i$. Чему равны оценки коэффициентов в регрессии накопленной оценки на те же объясняющие переменные? Ответ обоснуйте.

\begin{sol}
$0.25\hb_1+0.75\hb_1'$, $0.25\hb_2+0.75\hb_2'$ и $0.25\hb_3+0.75\hb_3'$
\end{sol}
\end{problem}


\begin{problem}
Истинная модель имеет вид $y_i=\beta x_i +\e_i$. Вася оценивает модель $\hy_i=\hb x_i$ по первой части выборки, получает $\hb_a$, по второй части выборки — получает $\hb_b$ и по всей выборке — $\hb_{tot}$. Как связаны между собой $\hb_{a}$, $\hb_{b}$, $\hb_{tot}$? Как связаны между собой дисперсии $\Var(\hb_a)$,  $\Var(\hb_b)$ и  $\Var(\hb_{tot})$?


\begin{sol}
Сами оценки коэффициентов никак детерминистически не связаны, но при большом размере подвыборок примерно равны. А дисперсии связаны соотношением $\Var(\hb_a)^{-1}+\Var(\hb_b)^{-1}=\Var(\hb_{tot})^{-1}$
\end{sol}
\end{problem}


\begin{problem}
Сгенерируйте вектор $y$ из 300 независимых нормальных $\cN(10,1)$ случайных величин.
Сгенерируйте 40 «объясняющих» переменных, по 300 наблюдений в каждой, каждое наблюдение — независимая нормальная $\cN(5,1)$ случайная величина. Постройте регрессию $y$ на все 40 регрессоров и константу.
\begin{enumerate}
\item Сколько регрессоров оказалось значимо на 5\% уровне?
\item Сколько регрессоров в среднем значимо на 5\% уровне?
\item Эконометрист Вовочка всегда использует следующий подход: строит регрессию зависимой переменной на все имеющиеся регрессоры, а затем выкидывает из модели те регрессоры, которые оказались незначимы. Прокомментируйте Вовочкин эконометрический подход.
\end{enumerate}


\begin{sol}
В среднем ложно значимы должны быть 5\% регрессоров, то есть 2 регрессора.
\end{sol}
\end{problem}



\begin{problem}
Мы попытаемся понять, как введение в регрессию лишнего регрессора влияет на оценки уже имеющихся. В регрессии будет 100 наблюдений. Возьмем $\rho=0.5$. Сгенерим выборку совместных нормальных $x_i$ и $z_i$ с корреляцией $\rho$. Настоящий $y_i$ задаётся формулой $y_i=5+6x_i+\e_i$. Однако мы будем оценивать модель $\hy_i=\hb_1+\hb_2 x_i+\hb_3 z_i$.

\begin{enumerate}
\item Повторите указанный эксперимент 500 раз и постройте оценку для функции плотности $\hb_1$.
\item Повторите указанный эксперимент 500 раз для каждого $\rho$ от $-1$ до $1$ с шагом в $0.05$. Каждый раз сохраняйте полученные 500 значений $\hb_1$. В осях $(\rho,\hb_1)$ постройте 95\%-ый предиктивный интервал для $\hb_1$. Прокомментируйте.
\end{enumerate}


\begin{sol}
\end{sol}
\end{problem}


\begin{problem}
Цель задачи — оценить модель CAPM несколькими способами.
\begin{enumerate}
\item Соберите подходящие данные для модели CAPM. Нужно найти три временных ряда: ряд цен любой акции, любой рыночный индекс, безрисковый актив. Переведите цены в доходности.
\item Постройте графики.
\item Оцените модель CAPM без свободного члена по всем наборам данных. Прокомментируйте смысл оцененного коэффициента.
\item Разбейте временной период на два участка и проверьте устойчивость коэффициента бета.
\item Добавьте в классическую модель CAPM свободный член и оцените по всему набору данных. Какие выводы можно сделать?
\item Методом максимального правдоподобия оцените модель с ошибкой измерения $R^m-R^0$, т.е.

истинная зависимость имеет вид
\begin{equation*}
(R^s-R^0)=\b_1+\b_2(R_m^*-R_0^*)+\e
\end{equation*}
величины $R_m^*$ и $R_0^*$ не наблюдаемы, но
\begin{equation*}
R_m-R_0=R_m^*-R_0^*+u
\end{equation*}

\end{enumerate}


\begin{sol}
\end{sol}
\end{problem}








\begin{problem}
По 47 наблюдениям оценивается зависимость доли мужчин занятых в сельском хозяйстве от уровня образованности и доли католического населения по Швейцарским кантонам в 1888 году.

\[
Agriculture_i=\beta_1+\beta_2 Examination_i+\beta_3 Catholic_i+\varepsilon_i
\]

\begin{minted}[mathescape,
               numbersep=5pt,
               frame=lines,
               framesep=2mm]{r}
model1 <- lm(Agriculture ~ Examination + Catholic, data = swiss)
coef_t <- coeftest(model1)
dimnames(coef_t)[[2]] <- c("Оценка", "Ст. ошибка",
  "t-статистика", "P-значение")
coef_t <- coef_t[, -4]
coef_t[1, 1] <- NA
coef_t[2, 2] <- NA
coef_t[3, 3] <- NA
xtable(coef_t)
\end{minted}

\begin{tabular}{rrrr}
  \hline
 & Оценка & Ст. ошибка & t-статистика \\
  \hline
(Intercept) &  & 8.72 & 9.44 \\
  Examination & -1.94 &  & -5.08 \\
  Catholic & 0.01 & 0.07 &  \\
   \hline
\end{tabular}



\begin{enumerate}
\item Заполните пропуски в таблице.
\item Укажите коэффициенты, значимые на 10\% уровне значимости.
\item Постройте 99\%-ый доверительный интервал для коэффициента при переменной Catholic.
\end{enumerate}

\begin{sol}
\end{sol}
\end{problem}




\begin{problem}
Оценивается зависимость уровня фертильности всё тех же швейцарских кантонов в 1888 году от ряда показателей. В таблице представлены результаты оценивания двух моделей.

Модель 1: $Fertility_i=\beta_1+\beta_2 Agriculture_i+\beta_3 Education_i+\beta_4 Examination_i+\beta_5 Catholic_i+\varepsilon_i$

Модель 2: $Fertility_i=\gamma_1+\gamma_2 (Education_i+Examination_i)+\gamma_3 Catholic_i+u_i$

\begin{minted}[mathescape,
               numbersep=5pt,
               frame=lines,
               framesep=2mm]{r}
m1 <- lm(Fertility ~ Agriculture + Education +
  Examination + Catholic, data = swiss)
m2 <- lm(Fertility ~ I(Education + Examination) + Catholic,
  data = swiss)
texreg(list(m1, m2))
\end{minted}


\begin{tabular}{l c c }
\hline
 & Model 1 & Model 2 \\
\hline
(Intercept)                & $91.06^{***}$ & $80.52^{***}$ \\
                           & $(6.95)$      & $(3.31)$      \\
Agriculture                & $-0.22^{**}$  &               \\
                           & $(0.07)$      &               \\
Education                  & $-0.96^{***}$ &               \\
                           & $(0.19)$      &               \\
Examination                & $-0.26$       &               \\
                           & $(0.27)$      &               \\
Catholic                   & $0.12^{**}$   & $0.07^{*}$    \\
                           & $(0.04)$      & $(0.03)$      \\
I(Education + Examination) &               & $-0.48^{***}$ \\
                           &               & $(0.08)$      \\
\hline
R$^2$                      & 0.65          & 0.55          \\
Adj. R$^2$                 & 0.62          & 0.53          \\
Num. obs.                  & 47            & 47            \\
RMSE                       & 7.74          & 8.56          \\
\hline
\multicolumn{3}{l}{\scriptsize{$^{***}p<0.001$, $^{**}p<0.01$, $^*p<0.05$}}
\end{tabular}



\begin{enumerate}
\item Проверьте гипотезу о том, что коэффициент при $Education$ в модели 1 равен  $-0.5$.
\item На 5\% уровне значимости проверьте гипотезу о том, что
переменные $Education$ и $Examination$ оказывают одинаковое влияние на $Fertility$.
\end{enumerate}


\begin{sol}
\end{sol}
\end{problem}




\begin{problem}
По 2040 наблюдениям оценена модель зависимости стоимости квартиры в Москве (в 1000\$) от общего метража и метража жилой площади.

\begin{minted}[mathescape, numbersep=5pt, frame=lines, framesep=2mm]{r}
model1 <- lm(price ~ totsp + livesp, data = flats)
report <- summary(model1)
coef.table <- report$coefficients
rownames(coef.table) <-
  c("Константа", "Общая площадь", "Жилая площадь")
xtable(coef.table)
var.hat <- vcov(model1)
xtable(var.hat, digits = 4)
\end{minted}

\begin{tabular}{rrrrr}
  \hline
 & Estimate & Std. Error & t value & Pr($>$$|$t$|$) \\
  \hline
Константа & -88.81 & 4.37 & -20.34 & 0.00 \\
  Общая площадь & 1.70 & 0.10 & 17.78 & 0.00 \\
  Жилая площадь & 1.99 & 0.18 & 10.89 & 0.00 \\
   \hline
\end{tabular}


Оценка ковариационной матрицы $\Var(\hb)$ имеет вид
\begin{tabular}{rrrr}
  \hline
 & (Intercept) & totsp & livesp \\
  \hline
(Intercept) & 19.0726 & 0.0315 & -0.4498 \\
  totsp & 0.0315 & 0.0091 & -0.0151 \\
  livesp & -0.4498 & -0.0151 & 0.0335 \\
   \hline
\end{tabular}

\begin{enumerate}
\item Проверьте $H_0$: $\beta_{totsp}=\beta_{livesp}$. В чём содержательный смысл этой гипотезы?
\item Постройте доверительный интервал дли $\beta_{totsp}-\beta_{livesp}$. В чём содержательный смысл этого доверительного интервала?
\end{enumerate}



\begin{sol}

\begin{minted}[mathescape,
               numbersep=5pt,
               frame=lines,
               framesep=2mm]{r}
est.se <- sqrt(var.hat[2, 2] + var.hat[3, 3] - 2 * var.hat[2, 3])
hb <- model1$coefficients
est.diff <- hb[2] - hb[3]
est.diff - 1.96 * est.se
est.diff + 1.96 * est.se
\end{minted}

Из оценки ковариационной матрицы находим, что $se(\hb_{totsp} - \hb_{livesp})=0.27$.

Исходя из $Z_{crit}=1.96$ получаем доверительный интервал, $[-0.82; 0.23]$.

Вывод: при уровне значимости 5\% гипотеза о равенстве коэффициентов не отвергается.
\end{sol}
\end{problem}






\begin{problem}
По 2040 наблюдениям оценена модель зависимости стоимости квартиры в Москве (в 1000\$) от общего метража и метража жилой площади.

\begin{minted}[mathescape,
               numbersep=5pt,
               frame=lines,
               framesep=2mm]{r}
model1 <- lm(price ~ totsp + livesp, data = flats)
report <- summary(model1)
coef.table <- report$coefficients
rownames(coef.table) <-
  c("Константа", "Общая площадь", "Жилая площадь")
xtable(coef.table)
deviance(model1)
xtable(vcov(model1), digits = 4)
\end{minted}

\begin{tabular}{rrrrr}
  \hline
 & Estimate & Std. Error & t value & P-value \\
  \hline
Константа & -88.81 & 4.37 & -20.34 & 0.00 \\
  Общая площадь & 1.70 & 0.10 & 17.78 & 0.00 \\
  Жилая площадь & 1.99 & 0.18 & 10.89 & 0.00 \\
   \hline
\end{tabular}



Сумма квадратов остатков равна $RSS=2.2\cdot 10^6$.
Оценка ковариационной матрицы $\Var(\hb)$ имеет вид

\begin{tabular}{rrrr}
  \hline
 & (Intercept) & totsp & livesp \\
  \hline
(Intercept) & 19.0726 & 0.0315 & -0.4498 \\
  totsp & 0.0315 & 0.0091 & -0.0151 \\
  livesp & -0.4498 & -0.0151 & 0.0335 \\
   \hline
\end{tabular}



\begin{enumerate}
\item Постройте 95\%-ый доверительный интервал для ожидаемой стоимости квартиры с жилой площадью $30$ м$^2$ и общей площадью $60$ м$^2$.
\item Постройте 95\%-ый прогнозный интервал для фактической стоимости квартиры с жилой площадью $30$ м$^2$ и общей площадью $60$ м$^2$.
\end{enumerate}


\begin{sol}
\end{sol}
\end{problem}




\begin{problem}
По 2040 наблюдениям оценена модель зависимости стоимости квартиры в Москве (в 1000\$) от общего метража, метража жилой площади и дамми-переменной, равной 1 для кирпичных домов.
\begin{minted}[mathescape,
               numbersep=5pt,
               frame=lines,
               framesep=2mm]{r}
model1 <- lm(price ~ totsp + livesp + brick + brick:totsp +
  brick:livesp, data = flats)
report <- summary(model1)
coef.table <- report$coefficients
xtable(coef.table)
\end{minted}


\begin{tabular}{rrrrr}
  \hline
 & Estimate & Std. Error & t value & P-value \\
  \hline
(Intercept) & -66.03 & 6.07 & -10.89 & 0.00 \\
  totsp & 1.77 & 0.12 & 14.98 & 0.00 \\
  livesp & 1.27 & 0.25 & 5.05 & 0.00 \\
  brick & -19.59 & 9.01 & -2.17 & 0.03 \\
  totsp:brick & 0.42 & 0.20 & 2.10 & 0.04 \\
  livesp:brick & 0.09 & 0.38 & 0.23 & 0.82 \\
   \hline
\end{tabular}



\begin{enumerate}
\item Выпишите отдельно уравнения регрессии для кирпичных домов и для некирпичных домов.
\item Проинтерпретируйте коэффициент при $brick_i \cdot totsp_i$.
\end{enumerate}


\begin{sol}
\end{sol}
\end{problem}



\begin{problem}
По 20 наблюдениям оценивается линейная регрессия $\hy=\hb_1 +\hb_2 x+\hb_3 z$, причём истинная зависимость имеет вид $y=\beta_1 +\beta_2 x+\varepsilon$. Случайная ошибка $\varepsilon_i$ имеет нормальное распределение $\cN(0,\sigma^2)$.

\begin{enumerate}
\item Найдите вероятность $\P(\hb_3>se(\hb_3))$.
\item Найдите вероятность $\P(\hb_3>\sigma_{\hb_3})$.
\end{enumerate}


\begin{sol}

\begin{enumerate}
\item $\P(\hb_3>se(\hb_3))=\P(t_{17}>1)=0.166$
\begin{minted}[mathescape,
               numbersep=5pt,
               frame=lines,
               framesep=2mm]{r}
1 - pt(1, 17)
\end{minted}

\item $\P(\hb_3>\sigma_{\hb_3})=\P(\cN(0,1)>1)=0.159$
\begin{minted}[mathescape,
               numbersep=5pt,
               frame=lines,
               framesep=2mm]{r}
1 - pnorm(1)
\end{minted}
\end{enumerate}
\end{sol}
\end{problem}



\begin{problem}
К эконометристу Вовочке в распоряжение попали данные с результатами контрольной работы студентов по эконометрике. В данных есть результаты по каждой задаче, переменные $p_1$, $p_2$, $p_3$, $p_4$ и $p_5$, и суммарный результат за контрольную, переменная $kr$. Чему будут равны оценки коэффициентов, их стандартные ошибки, $t$-статистики, $P$-значения, $R^2$, $RSS$, если
\begin{enumerate}
\item Вовочка построит регрессию $kr$ на константу, $p_1$, $p_2$, $p_3$, $p_4$ и $p_5$;
\item Вовочка построит регрессию $kr$ на $p_1$, $p_2$, $p_3$, $p_4$ и $p_5$ без константы.
\end{enumerate}


\begin{sol}
В обоих случаях можно так подобрать коэффициенты $\hb$, что $kr_i=\widehat{kr}_i$. А именно, идеальные прогнозы достигаются при  $\hb_{p_1}=1$, $\hb_{p_2}=1$, $\hb_{p_3}=1$, $\hb_{p_4}=1$, $\hb_{p_5}=1$ и (в первой модели) $\hb_1=0$. Отсюда $RSS=0$, $ESS=TSS$, поэтому $R^2=1$ даже в модели без свободного члена. Получаем $\hs^2=0$, поэтому строго говоря $t$ статистики и $P$-значения не существуют из-за деления на ноль.

На практике при численной минимизации $RSS$ оказывается, что $t$-статистики коэффициентов при задачах принимают очень большие значения, а соответствующие $P$-значения крайне близки к нулю. В первой модели особенной на практике будет $t$ статистика свободного члена. В силу неопределенности вида $0/0$ свободный коэффициент на практике может оказаться незначим.
\end{sol}
\end{problem}





\begin{problem}
Сгенерируйте данные так, чтобы при оценке линейной регрессионной модели оказалось, что скорректированный коэффициент детерминации, $R^2_{adj}$, отрицательный.

\begin{sol}

\[ R^2_{adj}=1-(1-R^2)\frac{n-1}{n-k} \]

Следовательно, при $R^2$ близком к 0 и большом количестве регрессоров $k$ может оказаться, что $R^2_{adj}<0$.

Например,

\begin{minted}[mathescape,
               numbersep=5pt,
               frame=lines,
               framesep=2mm]{r}
set.seed(42)
y <- rnorm(200, sd = 15)
X <- matrix(rnorm(2000), nrow = 200)
model <- lm(y ~ X)
report <- summary(model)
report$adj.r.squared
\end{minted}

\end{sol}
\end{problem}



\begin{problem}
Для коэффициентов регрессии $y_i = \beta_1 + \beta_2 x_i + \beta_3 z_i  + \beta_4 w_i + \e_i$ даны 95\%-ые
доверительные интервалы: $\beta_2 \in (0.16;0.66)$, $\beta_3 \in (-0.33;0.93)$ и $\beta_4 \in (-1.01; 0.54)$.

\begin{enumerate}
\item Найдите $\hb_2$, $\hb_3$, $\hb_4$.
\item Определите, какие из переменных в регрессии значимы на уровне значимости 5\%.
\end{enumerate}


\begin{sol}
$\hb_2=0.41$, $\hb_3=0.3$, $\hb_4= -0.235$, переменная $x$ значима.
\end{sol}
\end{problem}



\begin{problem}
Для коэффициентов регрессии $y_i = \beta_1 + \beta_2 x_i + \beta_3 z_i  + \beta_4 w_i + \e_i$ даны 95\%-ые
доверительные интервалы: $\beta_2 \in (-0.15;1.65)$, $\beta_3 \in (0.32;0.93)$ и $\beta_4 \in (0.14; 1.55)$.

\begin{enumerate}
\item Найдите $\hb_2$, $\hb_3$, $\hb_4$.
\item Определите, какие из переменных в регрессии значимы на уровне значимости 5\%.
\end{enumerate}


\begin{sol}
$\hb_2=0.75$, $\hb_3=0.625$, $\hb_4= 0.845$, переменные $z$ и $w$ значимы
\end{sol}
\end{problem}


\begin{problem}
Эконометрэсса Мырли очень суеверна и поэтому оценила три модели:
\begin{enumerate}
\item[M1:] $y_i=\beta_1 + \beta_2 x_i + \beta_3 w_i + \e_i$ по всем наблюдениям.
\item[M2:] $y_i=\beta_1 + \beta_2 x_i + \beta_3 w_i + \beta_4 d_i + \e_i$ по всем наблюдениям, где $d_i$ — дамми-переменная равная $1$ для 13-го наблюдения и нулю иначе.
\item[M3:] $y_i=\beta_1 + \beta_2 x_i + \beta_3 w_i + \e_i$ по всем наблюдениям, кроме 13-го.
\end{enumerate}

\begin{enumerate}
\item Сравните между собой $RSS$ во всех трёх моделях.
\item Есть ли совпадающие оценки коэффициентов в этих трёх моделях? Если есть, то какие?
\item Может ли Мырли не выполняя вычислений узнать ошибку прогноза для 13-го наблюдения при использовании третьей модели? Если да, то как?
\end{enumerate}


\begin{sol}
$RSS_1 > RSS_2 = RSS_3$, в моделях два и три, ошибка прогноза равна $\hb_4$
\end{sol}
\end{problem}


\begin{problem} % 3.47
Рассмотрим модель $y_i = \beta_1+ \beta_2 x_i + \beta_3 w_i +\beta_4 z_i + \e_i$.  При оценке модели по $24$ наблюдениям оказалось, что $RSS=15$, $\sum (y_i-\bar{y}-w_i+\bar{w})^2=20$. На уровне значимости 1\% протестируйте гипотезу
\[
H_0:
\begin{cases}
\beta_2+\beta_3+\beta_4=1 \\
\beta_2=0 \\
\beta_3=1 \\
\beta_4=0
\end{cases}
\]


\begin{sol}
Сначала заметим, что в основной гипотезе есть зависимые ограничения, оставим только независимые:
\[
H_0:
\begin{cases}
\beta_2=0 \\
\beta_3=1 \\
\beta_4=0
\end{cases}
\]
Ограниченная модель имеет вид:
\[
y_i = \beta_1 + w_i + \e_i
\]
Введём замену $\tilde{y}_i = y_i - w_i$ и получим оценку коэффициента $\beta_1$:
\[
\hb_1 = \bar{\tilde{y}}_i = \bar y - \bar w
\]
Теперь можно найти $RSS_R$:
\[
RSS_R = \sum_{i=1}^{24} (y_i - \hat{\tilde{y}}_i)^2 = \sum_{i=1}^{24} (y_i - \bar y + \bar w - w_i)^2 = 20
\]
Осталось найти значение F-статистики, которая при верной $H_0$ имеет распределение $F_{3, 20}$:
\[
F_{obs} = \frac{(RSS_R - RSS_{UR})/q}{RSS_{UR}/(n-k_{UR})} = \frac{(20-15)/3}{15/(24-4)} = 2.(2)
\]
Так как $F_{obs} < F_{crit} = 3.09$, оснований отвергать нулевую гипотезу нет.
\end{sol}
\end{problem}



\begin{problem}
Модель регрессии $y_i = \beta_1 + \beta_2 x_i + \beta_3 z_i + \e_i$, в которой ошибки
$\e_i$ независимы и нормальны $N(0;\sigma^2)$, оценивается по 13 наблюдениям. Найдите $\E(RSS)$, $\Var(RSS)$, $\P(5\sigma^2<RSS<10\sigma^2)$, $\P(5\hs^2<RSS<10\hs^2)$.


\begin{sol}
$RSS/\sigma^2\sim\chi^2_{n-k}$, $\E(RSS)=(n-k)\sigma^2$, $\Var(RSS)=2(n-k)\sigma^4$, $\P(5\sigma^2<RSS<10\sigma^2)\approx 0.451$
\end{sol}
\end{problem}



\begin{problem}
Рассмотрим модель регрессии $y_i=\beta_1+\beta_2 x_i + \beta_3 z_i+\e_i$, в которой
ошибки $\e_i$ независимы и имеют нормальное распределение $N(0,\sigma^2)$. Известно, что выборка в $n = 30$
наблюдений была разбита на три непересекающиеся подвыборки, содержащие $n_1 = 13$, $n_2 = 4$ и $n_3 = 13$ наблюдений. Пусть $\hs_{j}^2$ — это оценка дисперсии случайных ошибок для
регрессии, оцененной по $j$-ой подвыборке. Найдите
\begin{enumerate}
\item  $\P(\hs_3^2>\hs_1^2)$, $\P(\hs_1^2>2\hs_2^2)$;
\item $\E(\hs_2^2/\hs_1^2)$, $\Var(\hs_2^2/\hs_1^2)$.
\end{enumerate}


\begin{sol}
$\P(\hs_3^2>\hs_1^2)=0.5$, $\P(\hs_1^2>2\hs_2^2)=0.5044$, $\E(\hs_2^2/\hs_1^2)=1.25$, $\Var(\hs_2^2/\hs_1^2)=4.6875$
 \end{sol}
\end{problem}


\begin{problem}
Рассмотрим модель регрессии $y_i=\beta_1+\beta_2 x_i + \beta_3 z_i+\e_i$, в которой
ошибки $\e_i$ независимы и имеют нормальное распределение $N(0,\sigma^2)$. Для $n = 13$ наблюдения найдите уровень
доверия следующих доверительных интервалов для неизвестного параметра $\sigma^2$:
\begin{enumerate}
\item $(0;RSS/4.865)$;
\item $(RSS/18.307;RSS/3.940)$;
\item $(RSS/15.987;\infty)$.
\end{enumerate}


\begin{sol}
90\% во всех пунктах
\end{sol}
\end{problem}




\begin{problem}
Пусть $\hb_1$ и $\hb_2$ — МНК-оценки коэффициентов в регрессии $y_i = \beta_1 + \beta_2 x_i + \e_i$, оцененной по наблюдениям $i = 1, \ldots, m$, а $\hat{\mu}$, $\hat{\nu}$, $\hat{\gamma}$ и $\hat{\delta}$ — МНК-коэффициенты в регрессии $y_i = \mu + \nu x_i + \gamma d_i + \delta x_i d_i + \e_i$, оцененной по наблюдениям $i = 1, \ldots, n$, где фиктивная переменная $d$ определяется следующим образом
\[
d_i = \begin{cases}
1 & \text{ при } i \in \{1, \ldots, m\} \\
0 & \text{ при } i \in \{m + 1, \ldots, n\} \\
\end{cases}
\]
Покажите, что $\hb_1 = \hat{\mu} + \hat{\gamma}$ и $\hb_2 = \hat{\nu} + \hat{\delta}$.


\begin{sol}
Поскольку $\hat{\mu}$, $\hat{\nu}$, $\hat{\gamma}$ и $\hat{\delta}$ являются МНК-коэффициентами в регрессии $y_i = \mu + \nu x_i + \gamma d_i + \delta x_i d_i + \e_i$, $i = 1, \ldots, n$, то для любых $\mu$, $\nu$, $\gamma$ и $\delta$ имеет место
\begin{multline}
\label{task1:1}\sum_{i=1}^n (y_i - \hat{\mu} - \hat{\nu} x_i - \hat{\gamma} d_i - \hat{\delta} x_i d_i - \e_i)^2 \leqslant \\
\sum_{i=1}^n (y_i - \mu - \nu x_i - \gamma d_i - \delta x_i d_i - \e_i)^2
\end{multline}
Перепишем неравенство (\ref{task1:1}) в виде
\begin{multline}
\label{task1:2}\sum_{i=1}^m (y_i - (\hat{\mu} + \hat{\gamma}) - (\hat{\nu} + \hat{\delta}) x_i)^2 + \sum_{i= m + 1}^n (y_i - \hat{\mu} - \hat{\nu} x_i)^2 \leqslant \\
\sum_{i=1}^m (y_i - ({\mu} + {\gamma}) - ({\nu} + {\delta}) x_i)^2 + \sum_{i= m + 1}^n (y_i - {\mu} - {\nu} x_i)^2
\end{multline}
Учитывая, что неравенство (\ref{task1:2}) справедливо для всех $\mu$, $\nu$, $\gamma$ и $\delta$, то оно останется верным для $\mu = \hat{\mu}$, $\nu = \hat{\nu}$ и произвольных $\gamma$ и $\delta$. Имеем
\begin{multline*}
\sum_{i=1}^m (y_i - (\hat{\mu} + \hat{\gamma}) - (\hat{\nu} + \hat{\delta}) x_i)^2 + \sum_{i= m + 1}^n (y_i - \hat{\mu} - \hat{\nu} x_i)^2 \leqslant \\
 \sum_{i=1}^m (y_i - (\hat{\mu} + {\gamma}) - (\hat{\nu} + {\delta}) x_i)^2 + \sum_{i= m + 1}^n (y_i - \hat{\mu} - \hat{\nu} x_i)^2
\end{multline*}
Следовательно
\begin{equation*}
\sum_{i=1}^m (y_i - (\hat{\mu} + \hat{\gamma}) - (\hat{\nu} + \hat{\delta}) x_i)^2 \leqslant \sum_{i=1}^m (y_i - (\hat{\mu} + {\gamma}) - (\hat{\nu} + {\delta}) x_i)^2
\end{equation*}
Обозначим $\tilde{\beta_1} := \hat{mu} + \gamma$ и $\tilde{\beta_2} := \hat{\nu} + \delta$. В силу произвольности $\gamma$ и $\delta$ коэффициенты $\tilde{\beta_1}$ и $\tilde{\beta_2}$ также произвольны. тогда для любых $\tilde{\beta_1}$ и $\tilde{\beta_2}$ выполнено неравенство:
\[
\sum_{i=1}^m (y_i - (\hat{\mu} + \hat{\gamma}) - (\hat{\nu} + \hat{\delta}) x_i)^2 \leqslant \sum_{i=1}^m (y_i - \tilde{\beta_1} - \tilde{\beta_2} x_i)^2
\]
которое означает, что $\hat{\mu} + \hat{\gamma}$ и $\hat{\nu} + \hat{\delta}$ являются МНК-оценками коэффициентов $\beta_1$ и $\beta_2$ в регрессии $y_i = \beta_1 + \beta_2 x_i + \e_i$, оцененной по наблюдениям $i = 1, \ldots, m$, то есть $\hb_1 = \hat{\mu} + \hat{\gamma}$ и $\hb_2 = \hat{\nu} + \hat{\delta}$.
\end{sol}
\end{problem}


\begin{problem}
Верно ли, что $R_{adj}^2 = 1 - (1 - R^2)\frac{n-1}{n-k}$ распределен по $F(n-k, n-1)$?
Если да, то объясните, почему, если нет, то тоже объясните, почему.


\begin{sol}
Не верно, поскольку $R_{adj}^2$ может принимать отрицательные значения, а $F(n-k, n-1)$ — не может.
\end{sol}
\end{problem}



\begin{problem}
Сгенерируйте набор данных, обладающий следующим свойством. Если попытаться сразу выкинуть регрессоры $x$ и $z$, то гипотеза о их совместной незначимости отвергается. Если вместо этого попытаться выкинуть отдельно $x$, или отдельно $z$, то гипотеза о незначимости не отвергается.


\begin{sol}
Сгенерируйте сильно коррелированные $x$ и $z$.
\end{sol}
\end{problem}



\begin{problem}
Сгенерируйте набор данных, обладающий следующим свойством. Если попытаться сразу выкинуть регрессоры $x$ и $z$, то гипотеза о их совместной незначимости отвергается. Если вместо сначала выкинуть отдельно $x$, то гипотеза о незначимости не отвергается. Если затем выкинуть $z$, то гипотезы о незначимости тоже не отвергается.


\begin{sol}
\end{sol}
\end{problem}


\begin{problem}
Напишите свою функцию, которая бы оценивала регрессию методом наименьших квадратов. На вход функции должны подаваться вектор зависимых переменных $y$ и матрица регрессоров $X$. На выходе функция должна выдавать список из $\hb$, $\hVar(\hb)$, $\hy$, $\he$, $ESS$, $RSS$ и $TSS$. По возможности функция должна проверять корректность аргументов, например, что в $y$ и $X$ одинаковое число наблюдений и т.д. Использовать \verb|lm| или \verb|glm| запрещается.


\begin{sol}
\end{sol}
\end{problem}



\begin{problem}
Сгенерируйте данные так, чтобы при оценке модели $\hy=\hb_1+\hb_2x+\hb_3z$ оказывалось, что $\hb_2>0$, а при оценке модели $\hy=\hb_1+\hb_2x$ оказывалось, что $\hb_2<0$.


\begin{sol}
\end{sol}
\end{problem}



\begin{problem}
Предложите способ, как построить доверительный интервал для вершины параболы.


\begin{sol}
bootstrap, дельта-метод.
\end{sol}
\end{problem}



\begin{problem}
Скачайте результаты двух контрольных работ по теории вероятностей, \url{https://github.com/bdemeshev/em301/raw/master/datasets/tvims2012_data.csv} с описанием данных, \url{https://github.com/bdemeshev/em301/raw/master/datasets/tvims2012_data_description.txt}. Наша задача попытаться предсказать результат второй контрольной работы зная позадачный результат первой контрольной, пол и группу студента.
\begin{enumerate}
\item Какая задача из первой контрольной работы наиболее существенно влияет на результат второй контрольной?
\item Влияет ли пол на результат второй контрольной?
\item Что можно сказать про влияние группы, в которой учится студент?
\end{enumerate}


\begin{sol}
\end{sol}
\end{problem}


\begin{problem}
Сформулируйте теорему Гаусса-Маркова.
\begin{sol}
\end{sol}
\end{problem}



\begin{problem}
Эконометресса Эвридика хочет оценить модель $y_i=\beta_1 + \beta_2 x_i +\beta_3 z_i + \e_i$. К сожалению, она измеряет зависимую переменную с ошибкой. Вместо $y_i$ Эвридика знает значение $y_i^*=y_i+u_i$ и использует его в качестве зависимой переменной при оценке регрессии. Ошибки измерения $u_i$ некоррелированы между собой и с $\e_i$.
\begin{enumerate}
\item Будут ли оценки Эвридики несмещёнными?
\item Могут ли дисперсии оценок Эвридики быть ниже чем дисперсии МНК оценок при использовании настоящего $y_i$?
\item Могут ли оценки дисперсий оценок Эвридики быть ниже оценок дисперсий МНК оценок при использовании настоящего $y_i$?
\end{enumerate}



\begin{sol}
При наличии ошибок в измерении зависимой переменной оценки остаются несмещёнными, их дисперсия растет. Однако оценка дисперсии может случайно оказаться меньше. Например, могло случиться, что ошибки $u_i$ случайно компенсировали $\e_i$.
\end{sol}
\end{problem}


\begin{problem}
Эконометресса Ефросинья исследует зависимость удоев от возраста и породы коровы. Она оценила модель
\[
\hy_i = \hb_1 + \hb_2 age_i +\hb_3 d_{1i} + \hb_4 d_{2i}
\]
Эконометресса Глафира исследует ту же зависимость:
\[
\hy_i = \hb'_1 + \hb'_2 age_i +\hb'_3 d'_{1i} + \hb'_4 d'_{2i}
\]
но вводит дамми-переменные вводит по-другому:

\begin{tabular}{c|cccc}
\toprule
Порода коровы & $d_1$ & $d_2$ & $d'_1$ & $d'_2$  \\
\midrule
Холмогорская & 0 & 0 & 1 & 1 \\
Тагильская & 1 & 0 & 0 & 1  \\
Ярославская & 0 & 1 & 1 & 0  \\
\bottomrule
\end{tabular}

Выразите оценки коэффициентов Глафиры через оценки коэффициентов Ефросиньи.


\begin{sol}
\end{sol}
\end{problem}


\begin{problem}
%\todo[inline]{предпосылки JB}
Для проверки гипотезы о нормальности ошибок регрессии используют в частности статистику Харке-Бера (Jarque-Bera):
\[
JB=\frac{S^2}{6} + \frac{(K-3)^2}{24},
\]
где $S=\sum_i^n \he_i^3 /\hs^3_{ML}$
Строго говоря статистика Харке-Бера проверяет гипотезу о том, что скошенность равна нулю, а эксцесс равен 3, т.е. $H_0:$ $\E(\e_i^3)=0$, $\E(\e_i^4)=3\sigma^4$. Асимптотически при верной $H_0$ статистика имеет хи-квадрат распределение с двумя степенями свободы.

По аналогии со статистикой Харке-Бера придумайте асимптотические статистики, которые бы проверяли гипотезы:
\begin{enumerate}
\item $H_0:$ $\E(\e_i^3)=0$;
\item $H_0:$ $\E(\e_i^4)=3\sigma^4$;
\item $H_0:$ $\E(\e_i^5)=0$.
\end{enumerate}
Какое асимптотическое распределение при верной $H_0$ будут иметь придуманные статистики?

\begin{sol}
\end{sol}
\end{problem}



\begin{problem}
Рассмотрим классическую модель линейной регрессии. Найдите предел по вероятности $\plim R^2$ при условии, что $\sigma^2 \rightarrow \infty$.

\begin{sol}
$0$
\end{sol}
\end{problem}



\begin{problem}
В отделении буйнопомешанных 30 больных. В инфекционном отделении 20 больных. Эконометрист Василий из департамента НОБ (науко-образного бреда) построил регрессию температуры больного на константу по каждому отделению в отдельности и по обоим отделениям сразу. По инфекционному отделению $RSS=15$, по отделению буйнопомешанных $RSS=25$, по обоим отделениям сразу — $RSS=80$. Эконометрист Василий хочет получить грант на публикацию в зарубежном рецензируемом журнале статьи, описывающей результаты регрессии по обоим отделениям сразу.

Помогите главврачу с помощью теста Чоу проверить корректность действий эконометриста Василия.


\begin{sol}
Проводим тест Чоу.
\end{sol}
\end{problem}



\begin{problem}
Все предпосылки классической теоремы Гаусса-Маркова со стохастическими регрессорами для случая независимой выборки выполнены. Является ли оценка $\hs$ несмещённой и состоятельной? Если оценка смещена, то в большую или меньшую сторону относительно $\sigma$?


\begin{sol}
Несмещённой является $\hs^2$, поэтому $\hs$ смещена, но состоятельна.
\end{sol}
\end{problem}

\begin{problem}
Аккуратно опишите процедуру сравнения с помощью $F$-теста двух вложенных (ограниченной и неограниченной) линейных моделей:
\begin{enumerate}
\item Сформулируйте $H_0$ и $H_a$.
\item Сформулируйте все предпосылки теста.
\item Укажите способ подсчёта тестовой статистики.
\item Укажите закон распределения тестовой статистики при верной $H_0$.
\item Сформулируйте правило, по которому делается вывод об $H_0$.
\end{enumerate}

\begin{sol}
\end{sol}
\end{problem}

\begin{problem} % 3.67
Чтобы не выдать себя, Джеймс Бонд оценивает с помощью МНК только однопараметрические регрессии вида $y_i=\b x_i + \e_i$. Однако он знаком с теоремой Фриша-Вау.
\begin{enumerate}
\item Сколько подобных однопараметрических регрессий ему придется оценить, чтобы получить оценку коэффициента $\beta_3$ в множественной регрессии $y_i = \b_1 + \b_2 x_i + \b_3 z_i + \e_i$?
\item Укажите, какие именно регрессии нужно построить для данной цели.
\end{enumerate}

\begin{sol}
\begin{enumerate}
\item 6
\item Сначала «очищаем» все переменные от вектора констант
\begin{itemize}
  \item $y$ на константу, откуда получим $\tilde{y}$
  \item $x$ на константу, откуда получим $\tilde{x}$
  \item $z$ на константу, откуда получим $\tilde{z}$
\end{itemize}
Затем «очищенные» от констант переменные «очищаем» от $x$:
\begin{itemize}
  \item $\tilde{y}$ на $\tilde{x}$, получаем $\tilde\tilde{y}$
  \item $\tilde{z}$ на $\tilde{x}$, получаем $\tilde\tilde{z}$
\end{itemize}
И из финальной регрессии $\tilde{\tilde{y}}$ на $\tilde{\tilde{z}}$ полуичм $\hb_3$.
\end{enumerate}
\end{sol}
\end{problem}

\begin{problem}
Известно, что $\Corr(y_i, x_i)=0$, $\Corr(y_i, z_i)=0$.
\begin{enumerate}
\item Какое максимальное значение может принять корреляция между $y_i$ и линейной комбинацией $x_i$ и $z_i$?
\item Как изменится ответ, если $\Corr(y_i, z_i)=0.0001$?
\end{enumerate}


\begin{sol}
$0$ и $1$
\end{sol}
\end{problem}


\begin{problem}
У эконометрессы Агнессы есть дамми-переменная $male_i$, равная 1 для мужчин, и дамми-переменная $female_i$, равная 1 для женщин. Зависимая переменная $y_i$ — доход индивида.
\[
A: \hy_i = \hb male_i
\]
\[
B: \hy_i = \hat{\gamma}female_i
\]
\[
C: \hy_i = \hat{\alpha}_1male_i+\hat{\alpha}_2 female_i
\]
\[
D: \hy_i =  \hat{\delta}_1 + \hat{\delta}_2male_i
\]

\begin{enumerate}
\item Проинтерпретируйте оценки коэффициентов во всех регрессиях.
\item Как связаны между собой оценки коэффициентов в регрессиях C и D?
\end{enumerate}
\begin{sol}
\end{sol}
\end{problem}


\begin{problem}
  Рассмотрим множественную регрессию с $n$ наблюдениями, $k$ оцениваемыми коэффициентами регрессии и нормально распределёнными ошибками $u_i$. Исследователь Рустам хочет оценить неизвестный параметр $\sigma^2=\Var(u_i)$.
  \begin{enumerate}
    \item При каком $c$ оценка  $\hs^2=c\cdot RSS$ будет несмещённой для параметра $\sigma^2$?
    \item При каком $c$ оценка  $\hs^2=c\cdot RSS$ будет обладать наименьшей среднеквадратичной ошибкой MSE?
  \end{enumerate}

\begin{sol}
$c=1/(n-k)$.
Заметим, что $RSS/\sigma^2 \sim \chi^2_{n-k}$, поэтому:
\[
MSE = \Var(\hs^2) + bias^2(\hs^2)=\sigma^4 \cdot (c^2 2(n-k) + (c(n-k)-1)^2).
\]
Минимизируя по $c$ получаем $c=1/(n-k+2)$.
\end{sol}
\end{problem}

\begin{problem}
Прогнозируемая переменная не зависит ни от одного регрессора, $y_i = \mu + u_i$, где ошибки $u_i$ нормальны $\cN(0; \sigma^2)$ и независимы. Эконометресса Беатриче строит регрессию $y_i$ на константу и еще 5 регрессоров по 51 наблюдению.

\begin{enumerate}
\item Как связаны между собой бета и гамма распределения?
\item Как связаны между собой гамма и хи-квадрат распределения?
\item Как связаны между собой $R^2$ и $F$-статистика о незначимости регрессии в целом?
\item Как распределена $F$-статистика?
\item Как распределён $R^2$?
\item Найдите $\E(R^2)$ и $\Var(R^2)$.
\end{enumerate}

\begin{sol}
  Коэффициент $R^2$ имеет бета-распределение.
\end{sol}


\end{problem}






\Closesolutionfile{solution_file}

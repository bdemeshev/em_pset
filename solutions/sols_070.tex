\protect \hypertarget {soln:7.1}{}
\begin{solution}{{7.1}}
\end{solution}
\protect \hypertarget {soln:7.2}{}
\begin{solution}{{7.2}}
увеличить количество наблюдений, уменьшить дисперсию случайной ошибки
\end{solution}
\protect \hypertarget {soln:7.3}{}
\begin{solution}{{7.3}}
\end{solution}
\protect \hypertarget {soln:7.4}{}
\begin{solution}{{7.4}}
\end{solution}
\protect \hypertarget {soln:7.5}{}
\begin{solution}{{7.5}}
\end{solution}
\protect \hypertarget {soln:7.6}{}
\begin{solution}{{7.6}}
\end{solution}
\protect \hypertarget {soln:7.7}{}
\begin{solution}{{7.7}}
\end{solution}
\protect \hypertarget {soln:7.8}{}
\begin{solution}{{7.8}}
$r^* = -1/2$
\end{solution}
\protect \hypertarget {soln:7.9}{}
\begin{solution}{{7.9}}
$r^* = -1/3$
\end{solution}
\protect \hypertarget {soln:7.10}{}
\begin{solution}{{7.10}}
$\Var(\hb_2)=\frac{\sigma^2}{RSS_2}=40$, $\Var(\hb_3)=\frac{\sigma^2}{RSS_3}=20$, а $RSS_j=(1-R^2_j) \cdot TSS_j$. При этом в парной регрессии $x$ на $z$ или $z$ на $x$ коэффициенты $R^2$ равны и равны квадрату выборочной корреляции, то есть $R^2_j=0.81$.
\end{solution}
\protect \hypertarget {soln:7.11}{}
\begin{solution}{{7.11}}
$1$, т.к. все главные компоненты ортогональны
\end{solution}
\protect \hypertarget {soln:7.12}{}
\begin{solution}{{7.12}}

\end{solution}
\protect \hypertarget {soln:7.13}{}
\begin{solution}{{7.13}}
$VIF=1/0.19$, $VIF\geq 1/0.19$
\end{solution}
\protect \hypertarget {soln:7.14}{}
\begin{solution}{{7.14}}
Никак, Мишель сделала линейную замену регрессоров, $\tilde X = X \cdot A$, где $A$ — обратима.
\end{solution}
\protect \hypertarget {soln:7.15}{}
\begin{solution}{{7.15}}
\end{solution}
\protect \hypertarget {soln:7.16}{}
\begin{solution}{{7.16}}
\end{solution}

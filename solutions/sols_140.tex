\protect \hypertarget {soln:13.1}{}
\begin{solution}{{13.1}}
\end{solution}
\protect \hypertarget {soln:13.2}{}
\begin{solution}{{13.2}}
$I = 2p(1-p)$, энтропия и индекс Джини максимальны при $p=0.5$.
\end{solution}
\protect \hypertarget {soln:13.3}{}
\begin{solution}{{13.3}}
\end{solution}
\protect \hypertarget {soln:13.4}{}
\begin{solution}{{13.4}}
\end{solution}
\protect \hypertarget {soln:13.5}{}
\begin{solution}{{13.5}}
\end{solution}
\protect \hypertarget {soln:13.6}{}
\begin{solution}{{13.6}}

\end{solution}
\protect \hypertarget {soln:13.7}{}
\begin{solution}{{13.7}}
\end{solution}
\protect \hypertarget {soln:13.8}{}
\begin{solution}{{13.8}}
$100\cdot \left(\frac{99}{100} \right)^{100}\approx 100/e \approx 37$
\end{solution}
\protect \hypertarget {soln:13.9}{}
\begin{solution}{{13.9}}
Сначала делим по $z$, потом по $x$, так как индекс Джини в таком порядке падает сильнее.
\end{solution}
\protect \hypertarget {soln:13.10}{}
\begin{solution}{{13.10}}

\end{solution}
\protect \hypertarget {soln:13.11}{}
\begin{solution}{{13.11}}
Нет, в силу выпуклости функций.
\end{solution}
\protect \hypertarget {soln:13.12}{}
\begin{solution}{{13.12}}
Все $y_i$ одинаковые; поровну $y_i$ двух типов; 1000 разных типов $y_i$, по одному наблюдению каждого типа.
\end{solution}
\protect \hypertarget {soln:13.13}{}
\begin{solution}{{13.13}}
\begin{tabular}{ccc}
\toprule
$y_i$ & $x_i$ & $z_i$ \\
\midrule
$1$ & $1$ & $1$ \\
$1$ & $2$ & $2$ \\
$0$ & $1$ & $2$ \\
$0$ & $2$ & $1$\\
\bottomrule
\end{tabular}
\end{solution}

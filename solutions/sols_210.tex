\protect \hypertarget {soln:15.1}{}
\begin{solution}{{15.1}}
\newpage
\end{solution}
\protect \hypertarget {soln:15.2}{}
\begin{solution}{{15.2}}
\end{solution}
\protect \hypertarget {soln:15.3}{}
\begin{solution}{{15.3}}
\end{solution}
\protect \hypertarget {soln:15.4}{}
\begin{solution}{{15.4}}
\end{solution}
\protect \hypertarget {soln:15.5}{}
\begin{solution}{{15.5}}
\end{solution}
\protect \hypertarget {soln:15.6}{}
\begin{solution}{{15.6}}
\end{solution}
\protect \hypertarget {soln:15.7}{}
\begin{solution}{{15.7}}
\end{solution}
\protect \hypertarget {soln:15.8}{}
\begin{solution}{{15.8}}
\end{solution}
\protect \hypertarget {soln:15.9}{}
\begin{solution}{{15.9}}
По определению ковариационной матрицы:

$\Var (\xi) =  \begin{pmatrix}
\Var (\xi_1) & \Cov (\xi_1, \xi_2) & \Cov (\xi_1, \xi_3) \\
\Cov (\xi_2, \xi_1) & \Var (\xi_2) & \Cov (\xi_2, \xi_3) \\
\Cov (\xi_3, \xi_1) & \Cov (\xi_3, \xi_2) & \Var (\xi_3) \\
\end{pmatrix}  =  \begin{pmatrix}
2 & 1 & -1 \\
1 & 3 & 0 \\
-1 & 0 & 4 \\
\end{pmatrix} $

\begin{multline*}
\Var (\xi_1 + \xi_2 + \xi_3)  = \Var   \begin{pmatrix}
 \begin{pmatrix}
1 & 1 & 1 \\
\end{pmatrix}  &  \begin{pmatrix}
\xi_1 \\
\xi_2 \\
\xi_3 \\
\end{pmatrix}
\end{pmatrix}  = \\
\begin{pmatrix}
1 & 1 & 1 \\
\end{pmatrix}  \Var   \begin{pmatrix}
\xi_1 \\
\xi_2 \\
\xi_3 \\
\end{pmatrix}   \begin{pmatrix}
1 \\
1 \\
1 \\
\end{pmatrix}  = \\
 \begin{pmatrix}
1 & 1 & 1 \\
\end{pmatrix}   \begin{pmatrix}
2 & 1 & -1 \\
1 & 3 & 0 \\
-1 & 0 & 4 \\
\end{pmatrix}   \begin{pmatrix}
1 \\
1 \\
1 \\
\end{pmatrix}  = 9
\end{multline*}
\end{solution}
\protect \hypertarget {soln:15.10}{}
\begin{solution}{{15.10}}
\begin{multline*}
\E (z_1) = \E   \begin{pmatrix}
 \begin{pmatrix}
0 & 0 \\
0 & 1 \\
\end{pmatrix}  &  \begin{pmatrix}
\xi_1 \\
\xi_2 \\
\end{pmatrix}  \\
\end{pmatrix}  =  \begin{pmatrix}
0 & 0 \\
0 & 1 \\
\end{pmatrix}  \E   \begin{pmatrix}
\xi_1 \\
\xi_2 \\
\end{pmatrix}  = \\
 \begin{pmatrix}
0 & 0 \\
0 & 1 \\
\end{pmatrix}   \begin{pmatrix}
1\\
2\\
\end{pmatrix}  =  \begin{pmatrix}
0\\
2\\
\end{pmatrix}
\end{multline*}

$\Var (z_1) = \Var   \begin{pmatrix}
 \begin{pmatrix}
0 & 0 \\
0 & 1 \\
\end{pmatrix}  &  \begin{pmatrix}
\xi_1 \\
\xi_2 \\
\end{pmatrix}  \\
\end{pmatrix}  =  \begin{pmatrix}
0 & 0 \\
0 & 1 \\
\end{pmatrix}  \Var   \begin{pmatrix}
\xi_1 \\
\xi_2 \\
\end{pmatrix}   \begin{pmatrix}
0 & 0 \\
0 & 1 \\
\end{pmatrix} ' =  \begin{pmatrix}
0 & 0 \\
0 & 1 \\
\end{pmatrix}   \begin{pmatrix}
2 & 1 \\
1 & 2 \\
\end{pmatrix}   \begin{pmatrix}
0 & 0 \\
0 & 1 \\
\end{pmatrix} ' = \\
\begin{pmatrix}
0 & 0 \\
0 & 1 \\
\end{pmatrix}   \begin{pmatrix}
0 & 1 \\
0 & 2 \\
\end{pmatrix}  =  \begin{pmatrix}
0 & 0 \\
0 & 2 \\
\end{pmatrix} $
\end{solution}
\protect \hypertarget {soln:15.11}{}
\begin{solution}{{15.11}}
$\E (z_2) = \E   \begin{pmatrix}
 \begin{pmatrix}
0 & 0 \\
0 & 1 \\
\end{pmatrix}  &  \begin{pmatrix}
\xi_1 \\
\xi_2 \\
\end{pmatrix}  & + &  \begin{pmatrix}
1\\
1\\
\end{pmatrix}  \\
\end{pmatrix}  =  \begin{pmatrix}
0 & 0 \\
0 & 1 \\
\end{pmatrix}  \E   \begin{pmatrix}
\xi_1 \\
\xi_2 \\
\end{pmatrix}  +  \begin{pmatrix}
1\\
1\\
\end{pmatrix}  =  \begin{pmatrix}
0 & 0 \\
0 & 1 \\
\end{pmatrix}   \begin{pmatrix}
1\\
2\\
\end{pmatrix}  +  \begin{pmatrix}
1\\
1\\
\end{pmatrix}  =  \begin{pmatrix}
0\\
2\\
\end{pmatrix} +  \begin{pmatrix}
1\\
1\\
\end{pmatrix}  =  \begin{pmatrix}
1\\
3\\
\end{pmatrix} $

Поскольку $z_2 = z_1 +  \begin{pmatrix}
1\\
1\\
\end{pmatrix} $, где $z_1$ — случайный вектор из предыдущей задачи, то $\Var (z_2) = \Var (z_1)$. Сдвиг случайного вектора на вектор-константу не меняет его ковариационную матрицу.
\end{solution}
\protect \hypertarget {soln:15.12}{}
\begin{solution}{{15.12}}
\textit{В данном примере проиллюстрирована процедура центрирования случайного вектора — процедура вычитания из случайного вектора его математического ожидания.}

$\E (z_3) = \E   \begin{pmatrix}
 \begin{pmatrix}
\xi_1 \\
\xi_2 \\
\end{pmatrix}  & - &  \begin{pmatrix}
\E  \xi_1 \\
\E  \xi_2 \\
\end{pmatrix}  \\
\end{pmatrix}  = \E   \begin{pmatrix}
\xi_1 \\
\xi_2 \\
\end{pmatrix}  - \E   \begin{pmatrix}
\E  \xi_1 \\
\E  \xi_2 \\
\end{pmatrix}  =  \begin{pmatrix}
\E  \xi_1 \\
\E  \xi_2 \\
\end{pmatrix}  -  \begin{pmatrix}
\E  \xi_1 \\
\E  \xi_2 \\
\end{pmatrix}  =  \begin{pmatrix}
0\\
0\\
\end{pmatrix} $

Заметим, что вектор $z_3$ отличается от вектора $h$ сдвигом на вектор-константу $ \begin{pmatrix}
\E  \xi_1 \\
\E  \xi_2 \\
\end{pmatrix} $, поэтому $\Var (z_3) = \Var (h)$.
\end{solution}
\protect \hypertarget {soln:15.13}{}
\begin{solution}{{15.13}}
\end{solution}
\protect \hypertarget {soln:15.14}{}
\begin{solution}{{15.14}}
\end{solution}
\protect \hypertarget {soln:15.15}{}
\begin{solution}{{15.15}}
\end{solution}
\protect \hypertarget {soln:15.16}{}
\begin{solution}{{15.16}}
\end{solution}
\protect \hypertarget {soln:15.17}{}
\begin{solution}{{15.17}}
Каждый из вариантов возможен.
\end{solution}
\protect \hypertarget {soln:15.18}{}
\begin{solution}{{15.18}}
\end{solution}
\protect \hypertarget {soln:15.19}{}
\begin{solution}{{15.19}}
Корреляционная матрица должна быть положительно определена. Получаем квадратное неравенство.
\end{solution}

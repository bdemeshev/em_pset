\protect \hypertarget {soln:10.1}{}
\begin{solution}{{10.1}}
Замечаем, что $u_i^2=\e_i^2=1$. Считаем:
\[
\E(\hb \mid x_1, x_2) = \E\left( \frac{x_1^*}{(x_1^*)^2  + (x_2^*)^2}\right) \E(y_1) + \E\left( \frac{x_2^*}{(x_1^*)^2  + (x_2^*)^2}\right) \E(y_2)
\]

В обоих случаях $\E(y_1)=0$. Получаем, $\E(\hb \mid x_1=0, \, x_2=1)=0.2\beta$, $\E(\hb \mid x_1=0, \, x_2=2)=0.8\beta$. Интуитивно объясняем: рисуем прямую по двум точкам. Мы знаем абсциссы точек с точностью $\pm 1$. Если точки близки, то это может сильно менять оценку наклона, если точки далеки, то случайность слабо влияет на наклон.
\end{solution}
\protect \hypertarget {soln:10.2}{}
\begin{solution}{{10.2}}
\end{solution}
\protect \hypertarget {soln:10.3}{}
\begin{solution}{{10.3}}
\end{solution}
\protect \hypertarget {soln:10.4}{}
\begin{solution}{{10.4}}
\end{solution}
\protect \hypertarget {soln:10.5}{}
\begin{solution}{{10.5}}
\end{solution}
\protect \hypertarget {soln:10.6}{}
\begin{solution}{{10.6}}
\end{solution}
\protect \hypertarget {soln:10.7}{}
\begin{solution}{{10.7}}
\end{solution}
\protect \hypertarget {soln:10.8}{}
\begin{solution}{{10.8}}
\end{solution}
\protect \hypertarget {soln:10.9}{}
\begin{solution}{{10.9}}
\end{solution}
\protect \hypertarget {soln:10.10}{}
\begin{solution}{{10.10}}
\end{solution}
\protect \hypertarget {soln:10.11}{}
\begin{solution}{{10.11}}
Оба правы, $\plim \he_1 = \e_1$ и $\plim \hy_1= \b_1 + \b_2 x_i + \b_3 z_i$.
\end{solution}
\protect \hypertarget {soln:10.12}{}
\begin{solution}{{10.12}}
В общем случае $\hb_2 \neq \hat \gamma_2$, однако $\plim \hb_2 = \plim \hat \gamma_2$.
\end{solution}
\protect \hypertarget {soln:10.13}{}
\begin{solution}{{10.13}}
\end{solution}
\protect \hypertarget {soln:10.14}{}
\begin{solution}{{10.14}}
  Оценки будут несмещёнными и состоятельными. Вызвано это тем, что $u_i = \beta_3 x_i^2 + w_i$ некоррелировано с $x_i$.
\end{solution}
\protect \hypertarget {soln:10.15}{}
\begin{solution}{{10.15}}
$\plim \left(\frac{1}{n}x'x\right)^{-1} = 109^{-1}$, $\plim \frac{1}{n}x'u = 0$ и $\plim (x'x)^{-1}x'u = 0$
\end{solution}
\protect \hypertarget {soln:10.16}{}
\begin{solution}{{10.16}}
  Да, например, равномерное распределение $(u_i, x_i)$ на круге или на окружности. Или равновероятное на восьми точках, $(\pm 1, \pm 1)$, $(\pm 2, \pm 2)$.
\end{solution}
\protect \hypertarget {soln:10.17}{}
\begin{solution}{{10.17}}
\end{solution}
\protect \hypertarget {soln:10.18}{}
\begin{solution}{{10.18}}
  Например, можно взять $u_1=x_2$, и все величины $\cN(0;1)$.
\end{solution}
\protect \hypertarget {soln:10.19}{}
\begin{solution}{{10.19}}
  Одинаково распределены: $y_1 \sim y_2 \sim y_3$, $x_1 \sim x_2 \sim x_3$. Независимы переменные с разными номерами.
\end{solution}

\protect \hypertarget {soln:12.1}{}
\begin{solution}{{12.1}}
\end{solution}
\protect \hypertarget {soln:12.2}{}
\begin{solution}{{12.2}}
\end{solution}
\protect \hypertarget {soln:12.3}{}
\begin{solution}{{12.3}}
$f(x_1,x_2)=(x_1^2,x_2^2,\sqrt{2}x_1x_2)$
\end{solution}
\protect \hypertarget {soln:12.4}{}
\begin{solution}{{12.4}}
\end{solution}
\protect \hypertarget {soln:12.5}{}
\begin{solution}{{12.5}}
\end{solution}
\protect \hypertarget {soln:12.6}{}
\begin{solution}{{12.6}}
В исходном пространстве: $|\vec{a}|=\sqrt{3}$, $|\vec{b}|=\sqrt{5}$, $\cos(\vec{a},\vec{b})=\sqrt{0.6}$.

В расширяющем пространстве: $|h(\vec{a})|=1$, $|h(\vec{b})|=1$, $\cos(h(\vec{a}),h(\vec{b}))=e^{-2}$.
\end{solution}
\protect \hypertarget {soln:12.7}{}
\begin{solution}{{12.7}}
Длина равна 1 и не зависит от $\sigma$. При $\sigma \approx 0$ вектора примерно совпадают, при больших $\sigma$ вектора примерно ортогональны.
\end{solution}
\protect \hypertarget {soln:12.8}{}
\begin{solution}{{12.8}}
$C=0$ и $\sigma=+\infty$
\end{solution}
\protect \hypertarget {soln:12.9}{}
\begin{solution}{{12.9}}
\end{solution}
\protect \hypertarget {soln:12.10}{}
\begin{solution}{{12.10}}
\end{solution}
\protect \hypertarget {soln:12.11}{}
\begin{solution}{{12.11}}
\end{solution}
\protect \hypertarget {soln:12.12}{}
\begin{solution}{{12.12}}
\end{solution}

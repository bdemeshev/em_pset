\protect \hypertarget {soln:9.1}{}
\begin{solution}{{9.1}}
\end{solution}
\protect \hypertarget {soln:9.2}{}
\begin{solution}{{9.2}}
\end{solution}
\protect \hypertarget {soln:9.3}{}
\begin{solution}{{9.3}}
\end{solution}
\protect \hypertarget {soln:9.4}{}
\begin{solution}{{9.4}}
\end{solution}
\protect \hypertarget {soln:9.5}{}
\begin{solution}{{9.5}}
$H_0:$ модели (1) и (2) имеют одинаковое качество,
$H_1:$ модели (1) и (2) имеют разное качество, то есть одна из моделей лучше,
\begin{enumerate}
\item Тестовая статистика: \[T=\frac{n}{2}\left| \ln \frac{RSS_{2}^{*}}{RSS_{1}^{*}} \right|\].
\item Распределение тестовой статистики: \[T\underset{H_0, asy}{\sim} \chi^2_1 \].
\item Наблюдаемое значение тестовой статистики \[{{T}_{obs}}=\frac{80}{2}\left| \ln \frac{121}{239} \right|\approx \text{27.23}\].
\item Область, в которой $H_0$ не отвергается: \[[0;\ {{T}_{cr}}]=[0;\ qchisq(0.95,\ df=1)]=[0;\ \text{3.84}]\].
\item Статистический вывод: поскольку \[{{T}_{obs}}\notin [0;\ {{T}_{cr}}]\], гипотеза $H_0$ отвергается в пользу гипотезы $H_1$. Стало быть, из того, что \[RSS_{1}^{*}>RSS_{2}^{*}\] следует, что модель (1*) хуже модели (2*), а значит, модель (1) хуже модели (2). Таким образом, тест Бокса–Кокса говорит о том, что предпочтительнее модель с логарифмом — модель (2).
\end{enumerate}

\end{solution}
\protect \hypertarget {soln:9.6}{}
\begin{solution}{{9.6}}
  $H_0:$ модели (1) и (2) имеют одинаковое качество,
  $H_1:$ модели (1) и (2) имеют разное качество, то есть одна из моделей лучше,
\begin{enumerate}
\item Тестовая статистика: \[T=\frac{n}{2}\left| \ln \frac{RSS_{2}^{*}}{RSS_{1}^{*}} \right|\].
\item Распределение тестовой статистики: \[T\underset{H_0, asy}{\sim} \chi^2_1 \].
\item Наблюдаемое значение тестовой статистики \[{{T}_{obs}}=\frac{40}{2}\left| \ln \frac{25}{20} \right|\approx \text{4.46}\].
\item Область, в которой $H_0$ не отвергается: \[[0;\ {{T}_{cr}}]=[0;\ qchisq(0.95,\ df=1)]=[0;\ \text{3.84}]\].
\item Статистический вывод: поскольку \[{{T}_{obs}}\notin [0;\ {{T}_{cr}}]\], гипотеза $H_0$ отвергается в пользу гипотезы $H_1$. Стало быть, из того, что \[RSS_{1}^{*}<RSS_{2}^{*}\] следует, что модель (1*) лучше модели (2*), а значит, модель (1) лучше модели (2). Таким образом, тест Бокса–Кокса говорит о том, что предпочтительнее модель без логарифма — модель (1).
\end{enumerate}

\end{solution}
\protect \hypertarget {soln:9.7}{}
\begin{solution}{{9.7}}
Чтобы избежать переполнения при подсчёте произведения всех $y_i$
\end{solution}
\protect \hypertarget {soln:9.8}{}
\begin{solution}{{9.8}}
\end{solution}
\protect \hypertarget {soln:9.9}{}
\begin{solution}{{9.9}}
\end{solution}

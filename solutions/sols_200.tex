\protect \hypertarget {soln:14.1}{}
\begin{solution}{{14.1}}
\end{solution}
\protect \hypertarget {soln:14.2}{}
\begin{solution}{{14.2}}
\end{solution}
\protect \hypertarget {soln:14.3}{}
\begin{solution}{{14.3}}
\end{solution}
\protect \hypertarget {soln:14.4}{}
\begin{solution}{{14.4}}
\end{solution}
\protect \hypertarget {soln:14.5}{}
\begin{solution}{{14.5}}
\end{solution}
\protect \hypertarget {soln:14.6}{}
\begin{solution}{{14.6}}
Да, $H'=H$ и $H^2=H$.
\end{solution}
\protect \hypertarget {soln:14.7}{}
\begin{solution}{{14.7}}
Поскольку $H\cdot H = H$, то каждый столбец матрицы — это собственный вектор с собственным числом 1.
\end{solution}
\protect \hypertarget {soln:14.8}{}
\begin{solution}{{14.8}}
Да, $Hu=X(X'X)^{-1}X'u=0$.
\end{solution}
\protect \hypertarget {soln:14.9}{}
\begin{solution}{{14.9}}
Да.
\end{solution}
\protect \hypertarget {soln:14.10}{}
\begin{solution}{{14.10}}
Только 0 или 1.
\end{solution}
\protect \hypertarget {soln:14.11}{}
\begin{solution}{{14.11}}
Да.
\end{solution}
\protect \hypertarget {soln:14.12}{}
\begin{solution}{{14.12}}
Да.
\end{solution}
\protect \hypertarget {soln:14.13}{}
\begin{solution}{{14.13}}
Матрица $H$ — это матрица-шляпница, проектор. Собственные числа у неё — 0 и 1. Единице соответствуют столбцы матрицы $X$ и их линейные комбинации. Нулю соответствуют вектора, ортогональные одновременно всем столбцам матрицы $X$.
\end{solution}
\protect \hypertarget {soln:14.14}{}
\begin{solution}{{14.14}}
Например, $A=(1,2,3)$, $B=(1,0, 1)'$.
\end{solution}
\protect \hypertarget {soln:14.15}{}
\begin{solution}{{14.15}}
$\tr(I)=n$, $\tr(\pi)=1$, $\tr(H)=k$
\end{solution}
\protect \hypertarget {soln:14.16}{}
\begin{solution}{{14.16}}
\end{solution}
\protect \hypertarget {soln:14.17}{}
\begin{solution}{{14.17}}
\end{solution}
\protect \hypertarget {soln:14.18}{}
\begin{solution}{{14.18}}
$n\times m$, $m\times n$, $I$
\end{solution}
\protect \hypertarget {soln:14.19}{}
\begin{solution}{{14.19}}
\end{solution}
\protect \hypertarget {soln:14.20}{}
\begin{solution}{{14.20}}
\end{solution}
\protect \hypertarget {soln:14.21}{}
\begin{solution}{{14.21}}
\end{solution}
\protect \hypertarget {soln:14.22}{}
\begin{solution}{{14.22}}
Собственные векторы — все векторы с нулевой суммой компонент, $\lambda=0$, векторы из одинаковых чисел, $\lambda=n$.
\end{solution}
\protect \hypertarget {soln:14.23}{}
\begin{solution}{{14.23}}
Собственные векторы — все векторы перпендикулярные $v$, $\lambda=0$, векторы пропорциональные $v$, $\lambda=|v|^2$.
\end{solution}
\protect \hypertarget {soln:14.24}{}
\begin{solution}{{14.24}}
\end{solution}
\protect \hypertarget {soln:14.25}{}
\begin{solution}{{14.25}}
$X'Xv=\lambda v$, следовательно $XX'Xv=Xv$ и $w=Xv$ — собственный вектор матрицы $XX'$.
\end{solution}
\protect \hypertarget {soln:14.26}{}
\begin{solution}{{14.26}}
Если матрица $X$ обратима, то $\beta = X^{-1}y$.

Если решений нет, то наилучшее приближение к решению — это $\beta = (X'X)^{-1}X'y$

Если решений бесконечно много, то решение с наименьшей длиной — это $\beta=X(XX')^{-1}y$.
\end{solution}
\protect \hypertarget {soln:14.27}{}
\begin{solution}{{14.27}}
  Только если $M=I$. Доказательство. Если $M^2=M$, и $M$ — обратима, то, домножив обе части равенства на $M^{-1}$, получим $M=I$.
\end{solution}

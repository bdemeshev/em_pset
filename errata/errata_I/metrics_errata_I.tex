\documentclass[12pt]{article} % размер шрифта

\usepackage{tikz} % картинки в tikz
\usepackage{microtype} % свешивание пунктуации

\usepackage{array} % для столбцов фиксированной ширины

\usepackage{indentfirst} % отступ в первом параграфе

\usepackage{sectsty} % для центрирования названий частей
\allsectionsfont{\centering} % приказываем центрировать все sections

\usepackage{amsmath} % куча стандартных математических плюшек

\usepackage[top=2cm, left=1.5cm, right=1.5cm, bottom=2cm]{geometry} % размер текста на странице

\usepackage{lastpage} % чтобы узнать номер последней страницы

\usepackage{enumitem} % дополнительные плюшки для списков
%  например \begin{enumerate}[resume] позволяет продолжить нумерацию в новом списке
\usepackage{caption} % подписи к картинкам без плавающего окружения figure


\usepackage{fancyhdr} % весёлые колонтитулы
\pagestyle{fancy}
\lhead{Эконометрика в задачах и упражнениях, издание I}
\chead{}
\rhead{\today}
\lfoot{Дмитрий Борзых, Борис Демешев}
\cfoot{}
\rfoot{\thepage/\pageref{LastPage}}
\renewcommand{\headrulewidth}{0.4pt}
\renewcommand{\footrulewidth}{0.4pt}



\usepackage{todonotes} % для вставки в документ заметок о том, что осталось сделать
% \todo{Здесь надо коэффициенты исправить}
% \missingfigure{Здесь будет картина Последний день Помпеи}
% команда \listoftodos — печатает все поставленные \todo'шки

\usepackage{booktabs} % красивые таблицы
% заповеди из документации:
% 1. Не используйте вертикальные линии
% 2. Не используйте двойные линии
% 3. Единицы измерения помещайте в шапку таблицы
% 4. Не сокращайте .1 вместо 0.1
% 5. Повторяющееся значение повторяйте, а не говорите "то же"

\usepackage{fontspec} % поддержка разных шрифтов
\usepackage{polyglossia} % поддержка разных языков

\setmainlanguage{russian}
\setotherlanguages{english}

\setmainfont{Linux Libertine O} % выбираем шрифт
% можно также попробовать Helvetica, Arial, Cambria и т.Д.

% чтобы использовать шрифт Linux Libertine на личном компе,
% его надо предварительно скачать по ссылке
% http://www.linuxlibertine.org/index.php?id=91&L=1

\newfontfamily{\cyrillicfonttt}{Linux Libertine O}
% пояснение зачем нужно шаманство с \newfontfamily
% http://tex.stackexchange.com/questions/91507/

\AddEnumerateCounter{\asbuk}{\russian@alph}{щ} % для списков с русскими буквами
\setlist[enumerate, 2]{label=\asbuk*),ref=\asbuk*} % списки уровня 2 будут буквами а) б) ...

%% эконометрические и вероятностные сокращения
\DeclareMathOperator{\Cov}{Cov}
\DeclareMathOperator{\Corr}{Corr}
\DeclareMathOperator{\Var}{Var}
\DeclareMathOperator{\E}{E}
\def \hb{\hat{\beta}}
\def \hs{\hat{\sigma}}
\def \htheta{\hat{\theta}}
\def \s{\sigma}
\def \hy{\hat{y}}
\def \hY{\hat{Y}}
\def \v1{\vec{1}}
\def \e{\varepsilon}
\def \he{\hat{\e}}
\def \z{z}
\def \hVar{\widehat{\Var}}
\def \hCorr{\widehat{\Corr}}
\def \hCov{\widehat{\Cov}}
\def \cN{\mathcal{N}}




\newcommand{\reportedby}[2]{{\small [First reported by #1 on \mbox{#2}.]}}
\newcommand{\erratum}[1]{\subsubsection*{#1}}
% \erroronpage <page> <line info> <contributor> <date>
\newcommand{\erroronpage}[4]{\textbf{Стр. #1, #2} (#3, #4)}


\begin{document}
%\maketitle
%\begin{abstract} Список очепяток
%\end{abstract}

\section*{Список опечаток}


\erroronpage{14}{задача 2.6, пункт 10}{авторы}{15.10.2014}

Должно быть $\hat{\beta} = \frac{1}{2} \frac{y_n - y_1}{x_n - x_1} + \frac{1}{2n}  \left( \frac{y_1}{x_1} + \ldots + \frac{y_n}{x_n} \right) $

\erroronpage{41}{задача 3.24}{Анна Тихонова}{01.11.2014}

Вместо «уведичилась» должно быть «увеличилась»

\erroronpage{44}{задача 3.29}{Ангелина ??}{11.12.2014}

Пропущена фраза: Модель оценивается по 51 наблюдению.

\erroronpage{46}{задача 3.30}{Ангелина ??}{11.12.2014}

Пропущена фраза: Модель оценивается по 51 наблюдению.

\erroronpage{50}{задача 3.36}{авторы}{13.11.2014}

Пропали русские буквы в коде R.

\erroronpage{53}{задача 3.38}{авторы}{13.11.2014}

Пропали русские буквы в коде R. Ковариационную матрицу лучше привести с большим количеством цифр после запятой:

\begin{table}[ht]
\centering
\begin{tabular}{rrrr}
  \hline
 & (Intercept) & totsp & livesp \\
  \hline
(Intercept) & 19.0726 & 0.0315 & -0.4498 \\
  totsp & 0.0315 & 0.0091 & -0.0151 \\
  livesp & -0.4498 & -0.0151 & 0.0335 \\
   \hline
\end{tabular}
\end{table}


\erroronpage{54}{задача 3.39}{авторы}{13.11.2014}

Пропали русские буквы в коде R. Пропущена фраза про $RSS$:
«Сумма квадратов остатков равна $RSS=\ensuremath{2.2217\times 10^{6}}$».
Сильное округление в ковариационной матрице приводит к отрицательной оценке дисперсии, поэтому числа лучше привести с большим количеством цифр после запятой:

\begin{table}[ht]
\centering
\begin{tabular}{rrrr}
  \hline
 & (Intercept) & totsp & livesp \\
  \hline
(Intercept) & 19.0726 & 0.0315 & -0.4498 \\
  totsp & 0.0315 & 0.0091 & -0.0151 \\
  livesp & -0.4498 & -0.0151 & 0.0335 \\
   \hline
\end{tabular}
\end{table}


\erroronpage{55}{задача 3.41}{авторы}{13.11.2014}

Вместо $N(0, 1)$ должно быть $N(0, \sigma^2)$, хотя на решение это не влияет.

\erroronpage{67}{задача 4.10}{Андрей Зубанов}{07.01.2015}

Вместо: «вектора рамзера» следует писать «вектора размера»

\erroronpage{67}{задача 4.11}{Андрей Зубанов}{07.01.2015}

Вместо: «вектора рамзера» следует писать «вектора размера»

\erroronpage{68-76}{задача 4.13}{Александр Левкун}{22.10.2014}

В условии должно быть $y_i=\beta_1+\beta_2 x_{i2} + \beta_3 x_{i3}+\varepsilon_i$

Вместо $\beta_1$ везде должно быть $\beta_2$.

Вместо $\beta_2$ везде должно быть $\beta_3$.

Вместо $x_1$ везде должно быть $x_2$

\erroronpage{???}{задача 4.32}{Андрей Зубанов}{07.01.2015}

Теорема Фриша-Вау, лучше исправить на более популярное, теорема Фриша-Во-Ловелла

\erroronpage{98}{задача 7.5}{авторы}{02.12.2014}

В выражении для $\frac{\partial g}{\partial \theta'}$ вместо $c_1$ и $c_2$ должно быть $g_1$ и $g_2$ соответственно.

\erroronpage{100, 102}{задачи 5.6, 5.7}{авторы}{02.12.2014}

Логарифм вместо курсива, $ln L(\theta)$, должен быть прямым, $\ln L(\theta)$.


\erroronpage{113-114}{задача 7.3}{авторы}{13.11.2014}

Точки вместо русских букв в подписях к графику. Пропали русские буквы в коде R.

\erroronpage{119}{задача 8.2, условие}{авторы}{22.10.2014}

Должно быть: В модели $y_i=\beta_1 + \beta_2 x_i +\varepsilon_i$

\erroronpage{119}{задача 8.3, условие}{авторы}{22.10.2014}

Должно быть: В модели $y_i=\beta_1 + \beta_2 x_i +\varepsilon_i$

\erroronpage{122}{задача 8.8}{Анна Тихонова}{01.11.2014}

Пропущена запятая во фразе «Область, в которой»

\erroronpage{123}{задача 8.9}{Анна Тихонова}{01.11.2014}

Пропущена запятая во фразе «Область, в которой»

\erroronpage{124}{задача 8.10}{Анна Тихонова}{01.11.2014}

Пропущена запятая во фразе «Область, в которой»

\erroronpage{125}{задача 8.11}{Анна Тихонова}{01.11.2014}

Пропущена запятая во фразе «Область, в которой»

\erroronpage{126}{задача 8.12}{Анна Тихонова}{01.11.2014}

Пропущена запятая во фразе «Область, в которой»

\erroronpage{138-139}{задача 11.2}{авторы}{13.11.2014}

Точки вместо русских букв в подписях к графику. Пропали русские буквы в коде R.

\erroronpage{178}{устав проверки гипотез}{Анна Тихонова}{01.11.2014}

Пропущена запятая во фразе «Область, в которой»







\end{document}

\documentclass{article}
\usepackage[utf8]{inputenc}
\usepackage[russian]{babel}
\usepackage[parfill]{parskip}
\date{\today}
\title{Список опечаток к книжке \\
<<Эконометрика в задачах и упражнениях>>}
\author{Дмитрий Борзых, Борис Демешев}

\newcommand{\reportedby}[2]{{\small [First reported by #1 on \mbox{#2}.]}}
\newcommand{\erratum}[1]{\subsubsection*{#1}}
% \erroronpage <page> <line info> <contributor> <date> 
\newcommand{\erroronpage}[4]{\textbf{Стр. #1, #2} (#3, #4)}

\usepackage[left=1.5cm,right=1.5cm,top=1.5cm,bottom=1.5cm]{geometry}


\begin{document}
\maketitle
%\begin{abstract} Список очепяток
%\end{abstract}

\section{Первое издание задачника}

Здесь номера страниц и задач относятся к первому печатному изданию задачника.




\erroronpage{14}{задача 2.6, пункт 10}{авторы}{15.10.2014}

Должно быть $\hat{\beta} = \frac{1}{2} \frac{y_n - y_1}{x_n - x_1} + \frac{1}{2n}  \left( \frac{y_1}{x_1} + \ldots + \frac{y_n}{x_n} \right) $

\erroronpage{41}{задача 3.24}{Анна Тихонова}{01.11.2014}

Вместо <<уведичилась>> должно быть <<увеличилась>>

\erroronpage{44}{задача 3.29}{Ангелина ??}{11.12.2014}

Пропущена фраза: Модель оценивается по 51 наблюдению.

\erroronpage{46}{задача 3.30}{Ангелина ??}{11.12.2014}

Пропущена фраза: Модель оценивается по 51 наблюдению.

\erroronpage{50}{задача 3.36}{авторы}{13.11.2014}

Пропали русские буквы в коде R.

\erroronpage{53}{задача 3.38}{авторы}{13.11.2014}

Пропали русские буквы в коде R. Ковариационную матрицу лучше привести с большим количеством цифр после запятой:

\begin{table}[ht]
\centering
\begin{tabular}{rrrr}
  \hline
 & (Intercept) & totsp & livesp \\ 
  \hline
(Intercept) & 19.0726 & 0.0315 & -0.4498 \\ 
  totsp & 0.0315 & 0.0091 & -0.0151 \\ 
  livesp & -0.4498 & -0.0151 & 0.0335 \\ 
   \hline
\end{tabular}
\end{table}


\erroronpage{54}{задача 3.39}{авторы}{13.11.2014}

Пропали русские буквы в коде R. Пропущена фраза про $RSS$:
<<Сумма квадратов остатков равна $RSS=\ensuremath{2.2217\times 10^{6}}$>>. 
Сильное округление в ковариационной матрице приводит к отрицательной оценке дисперсии, поэтому числа лучше привести с большим количеством цифр после запятой:

\begin{table}[ht]
\centering
\begin{tabular}{rrrr}
  \hline
 & (Intercept) & totsp & livesp \\ 
  \hline
(Intercept) & 19.0726 & 0.0315 & -0.4498 \\ 
  totsp & 0.0315 & 0.0091 & -0.0151 \\ 
  livesp & -0.4498 & -0.0151 & 0.0335 \\ 
   \hline
\end{tabular}
\end{table}


\erroronpage{55}{задача 3.41}{авторы}{13.11.2014}

Вместо $N(0, 1)$ должно быть $N(0, \sigma^2)$, хотя на решение это не влияет.

\erroronpage{67}{задача 4.10}{Андрей Зубанов}{07.01.2015}

Вместо: <<вектора рамзера>> следует писать <<вектора размера>> 

\erroronpage{67}{задача 4.11}{Андрей Зубанов}{07.01.2015}

Вместо: <<вектора рамзера>> следует писать <<вектора размера>> 

\erroronpage{68-76}{задача 4.13}{Александр Левкун}{22.10.2014}

В условии должно быть $y_i=\beta_1+\beta_2 x_{i2} + \beta_3 x_{i3}+\varepsilon_i$

Вместо $\beta_1$ везде должно быть $\beta_2$.

Вместо $\beta_2$ везде должно быть $\beta_3$.

Вместо $x_1$ везде должно быть $x_2$

\erroronpage{???}{задача 4.32}{Андрей Зубанов}{07.01.2015}

Теорема Фриша-Вау, лучше исправить на более популярное, теорема Фриша-Во-Ловелла

\erroronpage{98}{задача 7.5}{авторы}{02.12.2014}

В выражении для $\frac{\partial g}{\partial \theta'}$ вместо $c_1$ и $c_2$ должно быть $g_1$ и $g_2$ соответственно.

\erroronpage{100, 102}{задачи 5.6, 5.7}{авторы}{02.12.2014}

Логарифм вместо курсива, $ln L(\theta)$, должен быть прямым, $\ln L(\theta)$.


\erroronpage{113-114}{задача 7.3}{авторы}{13.11.2014}

Точки вместо русских букв в подписях к графику. Пропали русские буквы в коде R.

\erroronpage{119}{задача 8.2, условие}{авторы}{22.10.2014}

Должно быть: В модели $y_i=\beta_1 + \beta_2 x_i +\varepsilon_i$

\erroronpage{119}{задача 8.3, условие}{авторы}{22.10.2014}

Должно быть: В модели $y_i=\beta_1 + \beta_2 x_i +\varepsilon_i$

\erroronpage{122}{задача 8.8}{Анна Тихонова}{01.11.2014}

Пропущена запятая во фразе <<Область, в которой>>

\erroronpage{123}{задача 8.9}{Анна Тихонова}{01.11.2014}

Пропущена запятая во фразе <<Область, в которой>>

\erroronpage{124}{задача 8.10}{Анна Тихонова}{01.11.2014}

Пропущена запятая во фразе <<Область, в которой>>

\erroronpage{125}{задача 8.11}{Анна Тихонова}{01.11.2014}

Пропущена запятая во фразе <<Область, в которой>>

\erroronpage{126}{задача 8.12}{Анна Тихонова}{01.11.2014}

Пропущена запятая во фразе <<Область, в которой>>

\erroronpage{138-139}{задача 11.2}{авторы}{13.11.2014}

Точки вместо русских букв в подписях к графику. Пропали русские буквы в коде R.

\erroronpage{178}{устав проверки гипотез}{Анна Тихонова}{01.11.2014}

Пропущена запятая во фразе <<Область, в которой>>







\end{document}